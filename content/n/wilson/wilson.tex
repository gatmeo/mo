\qitem{%
    Let $a$ be an integer such that
    \begin{alignat*}{1}
        \dfrac{1}{1}+\dfrac{1}{2}+\dfrac{1}{3}+\cdots+\dfrac{1}{23}=\dfrac{a}{23!}.
    \end{alignat*}
    Find the remainder when $a$ is divided by $13$.
    }{%
    Multiplying both sides of the equation by $23!,$ we obtain 
    \[ \frac{23!}1+\frac{23!}2+\frac{23!}3+\cdots+\frac{23!}{13}+\cdots+\frac{23!}{22}+\frac{23!}{23}=a. \]
    Note that all the terms on the LHS except $\frac{23!}{13}$ are divisible by $13,$ so the amswer os simply the modulo $13$ of $\frac{23!}{13}.$

    Note that $\frac{23!}{13} \equiv 12!10!\equiv \frac12 \equiv 7 \pmod{13},$ so $a\equiv \boxed{7}\pmod{13}.$
    }{%
    <++>
}

\qitem{%
    Suppose $p>3$ is an prime. Show that $(p-3)!\equiv -\frac{p+1}{2}$ mod $p$.
    }{%
    $(p-3)!\equiv (p-1)!(p-1)^{-1}(p-2)^{-1}\equiv -(2)^{-1}$. Since $2(\frac{p+1}{2})\equiv 1(p)$, $(p-3)!\equiv -\frac{p+1}{2}$ mod $p$.
    }{%
    <++>
}

\qitem{%
    Let $p$ be a prime, $a$ and integer. Show that $p|(a^p+(p-1)!a)$.
    }{%
    $(a^p+(p-1)!a)\equiv a+ (-1)a\equiv 0 (p)$
    }{%
    <++>
}

\qitem{%
    Prove that if $p$ and $p+2$ are a pair of twin primes, then
    \begin{alignat*}{1}
        4((p-1)!+1)+p\equiv 0\textnormal{ mod }p(p+2)
    \end{alignat*}
    }{%
    $4((p-1)!+1)+p\equiv 4(-1+1)+0(p),4((p-1)!+1)+p\equiv 4((-2)^{-1}+1)+(-2)\equiv 0 (p+2)$
    }{%
    <++>
}

\qitem{%
    Let $p$ be an odd prime, then
    \begin{alignat*}{1}
        \left(\left(\frac{p-1}{2}\right)!\right)^2\equiv (-1)^{\frac{p+1}{2}} (p)
    \end{alignat*}
    }{%
    \begin{alignat*}{1}
        \left(\left(\frac{p-1}{2}\right)!\right)^2&\equiv \frac{p-1}{2}! (p-1)(p-2)\cdots (p-\frac{p-1}{2})\cdot (-1)^{p-1/2}\\
                                                  &\equiv -1\cdot (-1)^{(p-1)/2}\\
                                                  &\equiv (-1)^{\frac{p+1}{2}} (p) 
    \end{alignat*}
    }{%
    <++>
}

\qitem{%
    Prove that if $p$ is an odd prime
    \begin{alignat*}{1}
        1^2\cdot 3^2\cdots(p-2)^2\equiv 2^2\cdot 4^2\cdots(p-1)^2\equiv (-1)^{\frac{p-1}{2}}\textnormal{ mod }p
    \end{alignat*}
    }{%
    \begin{alignat*}{1}
        1^2\cdot 3^2\cdots(p-2)^2&\equiv 1(p-1)(-1)3(p-3)(-1)\cdots (p-2)(2)(-1)\\
                                 &\equiv (p-1)!(-1)^{p-1/2}
    \end{alignat*}
    }{%
    <++>
}

\qitem{%
    Find the set of all positive integers $n$ with the property that the set
    \begin{alignat*}{1}
        \left\{n,n+1,n+2,n+3,n+4,n+5\right\}
    \end{alignat*}
    can be partitioned into 2 sets such that the product of the numbers in one set equals the product of the numbers in the other set.
    }{%
    Consider mod $7$, if only one element is divisible by $7$, then the product of two sets is not equal. 

    If all relatively prime to $7$, we know 
    \[n(n+1)\cdots (n+6)\equiv 1\cdot 2\cdot \cdots \cdot 6\equiv A\cdot B\equiv -1 (7)\]
    If $A=B$, we have $A^2\equiv -1 (7)$, which is impossible by checking.
    }{%
    <++>
}
