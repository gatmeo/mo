\mysection{Wilson's Theorem}

\begin{mysubsection}{}
    \begin{theorem}[pps:]{}
        If $a^2\equiv 1$ mod $p$, thenn either $a\equiv 1$ mod $p$ or $a\equiv -1$ mod $p$.
    \end{theorem}

    \begin{proof}
        We have $p|a^2-1=(a-1)(a+1)$. Since $p$ is a prime, it must divide at least one of the factors. This proves the lemma.
    \end{proof}

    \begin{theorem}[thm:]{}
        If $p$ is a prime, then $(p-1)!\equiv -1 \textnormal{ mod }p$.
    \end{theorem}

    \begin{proof}
        If $p=2$ or $p=3$, the result follows by direct verification. So assumet that $p>3$. Consider $a$, $2\leq a\leq p-2$. To each such $a$, we associate its unique inverse $\bar{a}$ mod $p$, i.e. $a\bar{a}\equiv 1$ mod $p$. Observe that $a\neq \bar{a}$ since then we would have $a^2\equiv 1$ mod $p$ which violates the preceding lemma as $a\neq 1$, $a\neq p-1$. Thus in multiplying all $a$ in the range $2\leq a\leq p-2$, we pair them of which their inverses, and the net contribution of this product is therefore $1$. In symbols,
        \begin{alignat*}{1}
            2\cdot 3\cdots(p-1)\equiv 1 \textnormal{ mod }p.
        \end{alignat*}
    \end{proof}
    \begin{proof}
        In other words,
        \begin{alignat*}{1}
            (p-1)!\equiv 1\left(\prod_{2\leq a\leq p-2}j\right)\cdot (p-1)&\equiv 1\cdot 1\cdot (p-1)\equiv -1\textnormal{ mod }p.
        \end{alignat*}

        This gives the result.
    \end{proof}
\end{mysubsection}

\begin{shortque}[]{1}
    \qitem{%
        Let $p$ be a prime. Find $(p-2)!$ mod $p$.
        }{%
        $p-1\equiv =-1, (p-1)^{-1}\equiv -1(p)$, $(p-2)!\equiv (p-1)!(p-1)^{-1}\equiv (-1)(-1)\equiv 1(p)$
        }{%
        <++>
    }

    \qitem{%
        Find the remainder of $97!$ when divided by $101$.
        }{%
        $-1\equiv 100!=100\times 99\times 98(97!)\equiv (-1)(-2)(-3)(97!)\textnormal{mod }101$, $(6)(97!)\equiv 1(101)$

        inverse of 6 is 17, 
        $101=6(16)+5, 6=5(1)+1,1=6+5(-1),1=6+[101+6(-16)](-1),1=101(-1)+6(17)$

        $(17)(6)(97!)\equiv (17)(1) (101)$, $97!\equiv 17(101)$.
        }{%
        <++>
    }

    \qitem{%
        Find the remainder of $67!$ when divided by $71$.
        }{%
        $(-1)(-2)(-3)(67!)\equiv -1 (71), 6(67!)\equiv 1$, o
        $71=6(11)+5,6=5(1)+1,1=6+5(-1),1=6+[71+6(-11)](-1),1=71(-1)+6(12)$

        $(12)(6)(67!)\equiv (12)(1), 67!\equiv 12$.
        }{%
        <++>
    }
\end{shortque}

