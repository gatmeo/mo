\mysection{Euclidean Algorithm}

\begin{mysubsection}{}
    \begin{theorem}[thm:]{Euclid}
        For natural numbers a,b, we use the division algorithm to determine a quotient and remainder, $q,r,$ such that $a=bq+r$. Then $gcd(a,b) = gcd(b,r)$.
    \end{theorem}

    \begin{proof}
        If $d$ is a common divisor of $a$ and $b$,then since $d$ divides both $a$ and $b$, $d$ divides all linear combinatinations of $a$ and $b$.  Therefore, $d|a-bq=r$, meaning that $d$ is also a common divisor of $b$ and $r$.

        Conversely, if $d$ is a common divisor of $b$ and $r$, then $d$ is a common divisorof all linear combinations of $b$ and $r$, therefore, $d|bq+r=a$.  Hence, $d$ is also a common divisor of $a$ and $b$.
    \end{proof}
\end{mysubsection}

\begin{mysubsection}{Extended Euclidean Algorithm}
    If $gcd(a,b)=1$, then by euclidean algorithm, we should be able to get something similar to:
    \begin{alignat*}{3}
        a&=b\times q_1 + r_1 &&\qquad \qquad 0<r_1<b\\
        b&=r_1\times q_2 + r_2 &&\qquad \qquad 0<r_2<r_1\\
        r_1&=r_2\times q_3 + 1 &&\qquad \qquad 0<r_3<r_2\\
    \end{alignat*}
    We then can we rewrite the fomular with $1=r_1-r_2\cdot q_3$, then substitute $r_2=b-r_1\cdot q_2$ to it, then sub $r_1=a-b\cdot q_1$, and can get a expression of $1=ma+nb$.

    Hence if we take mod $a$, we get $1\equiv nb\textnormal{ mod }a$, i.e. $n$ is the inverse of $b$. Similarly we get $1\equiv ma\textnormal{ mod }b$.
\end{mysubsection}

\begin{shortque}[]{Finding Inverse}
    \qitem{%
        Find the inverse of 216 mod 811.
        }{%
        \begin{tabular}{r|cc|l}
            3 & 811 & 216 & 1 \\
              &648 & 163 &        \\
              \hline
            3 & 163 & 53 & 13 \\
              & 159 & 52 & \\
              \hline
              & 4 & 1 &  \\
        \end{tabular}
        \begin{alignat*}{1}
            811&= 3\cdot 216+163\\
            216&= 1\cdot 163+53\\
            163&= 3\cdot 53+4\\
            53&= 13\cdot 4+1\\
        \end{alignat*}

        \begin{alignat*}{1}
            1&= 53-13\cdot 4\\
             &= 53 - 13 (163-3\cdot 53)\\
             &= 40 \cdot 53 - 13\cdot 163\\
             &= 40 \cdot (216-163) - 13\cdot 163\\
             &= 40\cdot 216 - 53\cdot 163\\
             &= 40\cdot 216 - 53\cdot (811-3\cdot 216)\\
             &= 199\cdot 216 - 53\cdot 811
        \end{alignat*}
        }{%
        <++>
    }

    \qitem{%
        Find the inverse of 134 mod $673$.
        }{%
        \begin{alignat*}{1}
            673&= 5\cdot 134+3\\
            134&= 44\cdot 3+2\\
            3&= 1\cdot 2+1\\
        \end{alignat*}

        \begin{alignat*}{1}
            1&= 3-1\cdot 2\\
             &= 3-(134-44\cdot 3)\\
             &= 45\cdot 3 - 134\\
             &= 45\cdot (673-5\cdot 134)-134\\
             &= 45\cdot 673 - 226\cdot 134
        \end{alignat*}
        }{%
        <++>
    }
\end{shortque}

\mynewpage
\begin{mysubsection}{Bezout's Identity}
    \begin{theorem}[thm:]{Bezout's Identity}
        For $a,b$ natural, there exist $x,y\in \mathbb{Z} $ such that $ax+by=gcd(a,b)$.
    \end{theorem}

    \begin{proof}
        We can run the Euclidean Algorithm backwards and get:
        \begin{alignat*}{1}
            \textnormal{gcd}(a,b)&= r_{n-2}-r_{n-1}q_n\\
                                 &= r_{n-2}-(r_{n-3}-r_{n-2}q_{n-1})q_n\\
                                 &= r_{n-2}(1+q_nq_{n-1})-r_{n-3}(q_n)\\
                                 &= \dots\\
                                 &= ax+by.
        \end{alignat*}
    \end{proof}
\end{mysubsection}

\begin{shortque}[]{}
    \qitem{%
        Express $10$ as a linear combination of $110$ and $380$.
        }{%
        \begin{alignat*}{1}
            380&= 3\cdot 110 + 50\\
            100&= 2\cdot 50 + 10\\
            50&= 5\cdot 10 
        \end{alignat*}
        \begin{alignat*}{1}
            10&= 110-2\cdot 50\\
              &= 110-2\cdot (380-3\cdot 110)\\
              &= 7\cdot 110-2\cdot 380
        \end{alignat*}
        }{%
        <++>
    }

    \qitem{%
        Express $3$ as a linear combination of $1011$ and $11202$.
        }{%
        \begin{alignat*}{1}
            11202&= 11\cdot 1011+81\\
            1011&= 12\cdot 81+39\\
            81&= 2\cdot 39+3\\
            39&= 13\cdot 3
        \end{alignat*}
        \begin{alignat*}{1}
            3&= 81-2\cdot 39\\
             &= 81-2(1011-81\cdot 12)\\
             &= 81\cdot 25-2\cdot 1011\\
             &= 25(11202-1011\cdot 11)-2\cdot 1011\\
             &= 25\cdot 11202-277\cdot 1011
        \end{alignat*}
        }{%
        <++>
    }
\end{shortque}

\mynewpage

\begin{shortque}[Bezout's Identity on Polynomials]{}
    \qitem{%
        Find polynomials $u(x),v(x)$ such that
        \begin{equation*}
            (x^4-1)u(x)+(x^7-1)v(x)=(x-1).
        \end{equation*}
        }{%
        \begin{alignat*}{1}
            x^7-1&= (x^4-1)x^3+x^3-1\\
            x^4-1&= x(x^3-1)+x-1\\
            x^3-1&= (x-1)(x^2+x+1)
        \end{alignat*}
        \begin{alignat*}{1}
            x-1&= x^4-1-x(x^3-1)\\
               &= x^4-1-x(x^7-1-(x^4-1)x^3)\\
               &= (x^4-1)(x^4+1)-x(x^7-1)
        \end{alignat*}
        Hence $u(x)=x^4+1, v(x)=-x$.
        }{%
        <++>
    }
\end{shortque}
