\mysection{Inverse}

\begin{mysubsection}{}
    \begin{definition}[def:]{}
        $a-b$ is divisible by $m$ if and only if $a\equiv b \mod m$ \qquad for positive integers $m, a \textnormal{ and }b$
    \end{definition}

    \begin{definition}[def:]{Inverse}
        The inverse of a number $x$ is a number $y$ s.t. $xy=1$. We denote the inverse of $x$ as $x^{-1} or 1/x$.
    \end{definition}
\end{mysubsection}

\begin{example}[exp:]{}
    \eitem{%
        Find the inverse of 2 mod 7.
        }{%
        Since $2\times 4=8\equiv 1\quad (\textnormal{mod }7)$, $2^{-1} = 4\quad (\textnormal{mod }7)$, and $4^{-1}=2\quad (\textnormal{mod }7)$.
    }
\end{example}

\begin{shortque}[]{}
    \qitem{%
        Find the inverse of $2,3,4,5$ mod 11.
        }{%
        $2\cdot 5\equiv -1, 3\cdot 4\equiv 1$

        Hence $2:-5, 3:4, 4:3, 5:-2$
        }{%
        <++>
    }

    \qitem{%
        Find the inverse of 34 mod 103.
        }{%
        $34\cdot 3\equiv 102 \equiv -1$.

        Hence $34^{-1}\equiv -3$.
        }{%
        <++>
    }
\end{shortque}

\mynewpage
\begin{mysubsection}{Uniqueness}
    If we find a pair $(a,b)$ s.t. $ab\equiv 1\quad (\textnormal{mod }n)$, then we know their inverses correspond to each other. However, does $a$ have inverse other than $b$, or vise versa? The answer is no, which can be easily proved.

    \myframebreak
    \begin{theorem}[thm:]{}
        Inveres is unique.
    \end{theorem}

    \begin{proof}
        Suppose $a$ has 2 distinct inverses $b$ and $c$ mod $n$.

        \begin{equation*}
            b=b\times 1=b\times (a\times c)=(b\times a)c=1\times c=c\quad (\textnormal{mod }n)
        \end{equation*}

        Contradiction. Hence, inverse of any integers mod $n$ is unique.
    \end{proof}
\end{mysubsection}

\begin{shortque}[]{}
    \qitem{%
        $x\equiv 28\cdot7^{-1}\quad (\textnormal{mod }29)$
        }{%
        4
        }{%
        <++>
    }

    \qitem{%
        $x\equiv 5^{99}\times (40 + 3^{-1})\quad (\textnormal{mod }121)$
        }{%
        $3^{-1}=-40$

        $x\equiv 0$.
        }{%
        <++>
    }

    \qitem{%
        Find the inverse of $9$ mod $82$.
        }{%
        $9^2\equiv -1$

        $9^{-1}\equiv -9$.
        }{%
        <++>
    }

    \nbf{Remark:} Indeed, inverse share a concept similar to division. Observed in the previous example, we know that we can simply get the answer for the first question as $28/7=4$. Yet, how does it work?
\end{shortque}

\mynewpage
\begin{example}[exp:]{}
    \eitem{%
        Find the value of $28\times 7^{-1}$ mod $29$.
        }{%
        We can write $28=4\times 7$, then $x\equiv 28\times 7^{-1}\equiv 4\times 7\times 7^{-1}\equiv 4\quad (\textnormal{mod }29)$. The $7$ and $7^{-1}$ cancel out each other, works as division.
    }

    \eitem{%
        Find the value of $55^{99}\times (40+3^{-1})$ mod $11$.
        }{%
        And most of other operation you do with factions can be done with inverse. We can find the common denominator using that concept. $a+b\times c^{-1}\equiv a\times c\times c^{-1}+b\times c^{-1}\equiv (a\times c+b)c^{-1}\quad (\textnormal{mod }n)$, which is the normal way of combining fractions. Hence, for the second question, we can write
        \begin{alignat*}{1}
            x&\equiv 5^{99}\times (40 + 3^{-1})\equiv 5^{99}\times (40\times 3 + 1)3^{-1}\equiv 5^{99}3^{-1}(121)\\
             &\equiv 0\quad (\textnormal{mod }11)
        \end{alignat*}
    }
\end{example}

\begin{mysubsection}{Existence}
    \begin{theorem}[thm:]{}
        Let $a$ and $m$ be relatively prime positive integers. Let the set of positive integers relatively prime to $m$ and less than $m$ be $R=\left\{a_1,a_2,\dots,a_{\phi (m)}\right\}$. Then $S=\left\{aa_1,aa_2,\dots,aa_{\phi (m)}\right\}$ is the same as $R$ when reduced mod $m$.
    \end{theorem}

    \begin{proof}
        It is sufficient to prove no 2 element in $S$ are the same. Assume contrary,
        \begin{equation*}
            aa_x=aa_y\quad (\textnormal{mod }m)\Rightarrow a(a_x-a_y)\equiv 0\quad (\textnormal{mod }m)\Rightarrow a_x\equiv a_y\quad (\textnormal{mod }m)
        \end{equation*}

        Hence, when $gcd(a,m)=1$, $a$ always has a distinct inverse mod $m$. (one of the element in $\left\{aa_1,aa_2,\dots,aa_{\phi (m)}\right\}$ equals 1) And there always integral solution for $ax\equiv b\quad (\textnormal{mod }m)$. (WHY?)\\

        And for any pair of $(a,m)$ whose gcd isnt 1, $a$'s inverse doesnt exist. For example, consider $3x\equiv 1\quad (\textnormal{mod }15)$, we can write $3x=15k+1$, $x=5k+1/3$, which there's no integral solution.
    \end{proof}
\end{mysubsection}
