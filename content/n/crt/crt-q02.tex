\qitem{%
    Let $f(n)=n^{99}+2n^{98}+3n^{97}+\cdots +99n+100$. Find $f(2)$ mod 100?
    }{%
    It must be $2 \mod 4$ because all terms but $99 \cdot 2 \equiv 2$ are $0 \mod 4$. Now we evaluate it $\mod 25$.

We use the standard formula $\sum_{k=1}^m kr^k = \frac{r(1-(m+1)r^m+mr^{m+1}}{(1-r)^2}$ for $r \neq 1$ (this result is useful in many other problems as well). A proof's provided below; there's a much shorter proof by differentiating the sum of a geometric series.
Proof of formula
We notice that $f(n)=\sum_{k=1}^{100} kn^{100-k}$
$$ \equiv n^{100} \sum{k=1}^{100} k(n^{-1})^k$$Now we substitute $r=n^{-1}, m=100$ in our formula:
$$f(n)=n^{100} \frac{n^{-1}(1-101(n^{-1})^{100}+100(n^{-1})^{101})}{(1-n^{-1})^2}$$Since $\phi(25)=20, a^{20} \equiv 1 (\mod 25)$ for $a$ coprime to $25$ by Euler's totient theorem. $2^{-1} \equiv 13 (\mod 25)$

$$f(2) \equiv (2^{20})^5 \cdot \frac{13(1-1(13^{20})^5)}{(1-13)^2}$$$$\equiv 1^5 \cdot \frac{13(1-1^5)}{144}$$$$\equiv 0 (\mod 25)$$.
We have $f(2) \equiv 0 (\mod 25), f(2) \equiv 2 (\mod 4)$, which gives $f(2) \equiv 50 (\mod 100)$
    }{%
    https://artofproblemsolving.com/community/c4t30973f4h1885343_a_sum_in_chinese_remainder_theoremfunctions__i_m_unable_to_solve
}
