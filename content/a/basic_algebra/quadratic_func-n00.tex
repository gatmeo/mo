\mysection{Quadratic Function}

\begin{mysubsection}{}
    For a quadric function, we can find its maximum and minimum value by doing completing square:
    \begin{alignat*}{1}
        ax^2+bx+c&= a\left(x^2+2\frac{b}{2a}x+\frac{b^2}{4a^2}\right)+c-\frac{b^2}{4a}\\
                  &= a\left(x+\frac{b}{2a}\right)^2+c-\frac{b^2}{4a}.
    \end{alignat*}
    Since $\left(x+\frac{b}{2a}\right)^2\geq 0$, if $a>0$, the minimum value of the quadratic function is $c-\frac{b^2}{4a}$ at $x=-\frac{b}{2a}$. If $a>0$, we get the same result except $c-\frac{b^2}{4a}$ is the maximum value.
\end{mysubsection}

\begin{shortque}[]{}
    \qitem{%
        If $x,y$ are positive numbers with $x+y=2a$, then the product $xy$ is maximal when $x=y=a$.
        }{%
        If $x+y=2a,$ then $y=2a-x$. Hence $xy=x(2a-x)=-x^2+2ax=-(x-a)^2+a^2$ has maximum value when $x=a$, and then $y=x=a$.
        }{%
        <++>
    }

    \qitem{%
        If $x,y$ are positive with $xy=1$, then sum $x+y$ is minimal when $x=y=1$.
        }{%
        If $xy=1$, then $y=\dfrac{1}{x}$. It follows that $x+y=x+\dfrac{1}{x}=\left(\sqrt{x}-\dfrac{1}{\sqrt{x}}\right)^2+2$, and then $x+y$ is minimal when $\sqrt{x}-\dfrac{1}{\sqrt{x}}=0$, that is when $x=1$. Therefore, $x=y=1$.
        }{%
        <++>
    }

    \qitem{%
        For any positive number $x$ we have $x+\frac{1}{x}\geq 2$.
        }{%
        $x+\dfrac{1}{x}=\left(\sqrt{x}-\dfrac{1}{\sqrt{x}}\right)^2+2$
        
        The equality holds if and only if $\sqrt{x}-\dfrac{1}{\sqrt{x }}=0$, that is when $x=1$.
        }{%
        <++>
    }
\end{shortque}

