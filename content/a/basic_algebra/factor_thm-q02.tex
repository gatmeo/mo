\qitem{%
    Let $f(x)=(x+m)(x+n)(3x-1)+2$, where $m$ and $n$ are positive integers and $m>n$. When $f(x)$ is divided by $x-1$, the remainder is $14$. Show that $(m+1)(n+1)=6$, and hence find the value of $m$ and $n$. 

    Let $g(x)=k(x^2+x+2)$, where $k$ is a non-zero constant. It is given that $f(x)-g(x)$ is divisible by $x+1$. Find the value of $k$, and determine whether there is only one of the roots of the equation $f(x)-g(x)=0$ is real.
    }{%
    We have $f(1)=14$, $(1+m)(1+n)(3\cdot 1-1)+2=14$, $(m+1)(n+1)=6$. Since $m$, $n$ both positive integers, $m>n$, we have $m+1=3$, $n+1=2$, $m=2, n=1$.

    We have $f(x)-g(x)=(x+2)(x+1)(3x-1)+2-k(x^2+x+2)$, since divisible by $x+1$, $f(-1)-g(-1)=0$, we hence have $k=1$. Hence $f(x)-g(x)=(x-1)(3x^2+4x-2)$ by long division. 

    Since $\Delta$ of $3x^2+4x-2=4^2-4(3)(-2)=40>0$, the claim is wrong.
    }{%
    <++>
}
