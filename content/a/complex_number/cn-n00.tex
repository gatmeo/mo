\mysection{Complex Numbers}

A point in the complex plane can be represented by a complex number written in cartesian coordinates. A standard representation of a complex number $z$ would be $z=x+iy$. 
Before introducing the main topic, we will fix some notations for complex numbers in general. We denote $r=|z|=\sqrt{x^2+y^2}$ be the \textbf{magnitude} of $z$. And denote $\overline{z}=x-yi$ be the \textbf{conjugate} of the complex number $z$. With a simple calculation, we see that $|z|=\sqrt{z\overline{z}}$.
To introduce a new way of expressing a complex number, we need Euler's Formula:
\\

{
\rightasy[1.5in]{
    import MOgeom;
    pair A=dir(60), B=dir(-60), N=mp(A--B);
    path c=D(circle(o,1));
    D(o--D("z",A));
    MAR("\theta",A,o,(1,0),10);
    D(A--(1/2,0));
    D(o--D("\overline{z}",B));
    MA("-\theta",B,o,(1,0),10);
    D(B--(1/2,0));
    pathpen=black;
    D((-1.4,0)--(1.4,0),ar=EndArrow);
    D((0,-1.4)--(0,1.4),ar=EndArrow);
}
\begin{definition}[def:]{Euler's Formula}
    We denote $e^{i\theta}=\cos \theta+i\sin \theta$.
\end{definition}
}

Now from the figure, we get a more fancy way of expressing $z$ and $\overline{z}$.
\[
    z=x+iy=|z|(\cos \theta+i\sin \theta)=re^{i\theta}.
\]
\[
    z=x-iy=|z|(\cos \theta-i\sin \theta)=re^{-i\theta}.
\]

One interesting consequence of expressing $z$ in such a way is that multiplication between become extremely easy. 
\[
    z_1z_2=r_1e^{i\theta_1}\cdot r_2e^{i\theta_2}=r_1r_2e^{i(\theta_1+\theta_2)}.
\]
For the proof of this multiplication being compatible with the standard multiplication on $(x_1+iy_1)(x_2+iy_2)$. But given such a fact, we have a new way of defining the compound angle formulas. For simplicity sake we assume for $z_1,z_2$ have magnitude $1$, then we have
\begin{alignat*}{1}
    \cos (\theta_1+\theta_2) + i\sin (\theta_1+\theta_2)
    &= z_1z_2\\
    &= (\cos \theta_1+i\sin \theta_1)(\cos \theta_2+i\sigma\theta_2)\\
    &= (\cos \theta_1\cos \theta_2 -\sin \theta_1\sin \theta_2) + i(\cos \theta_1\sin \theta_2+\sin \theta_1\cos \theta_2).
\end{alignat*}
By comparing the two sides, we have the following compound angle formulas. One can try to get the following table by just substituting $\theta_2=-\theta_2$ and some computations:

\noindent\textbf{Compound Angle Formulas}
\begin{alignat*}{3}
    \sin(a + b) &= \sin a \cos b + \cos a \sin b&\qquad 
    \sin(a - b) &= \sin a \cos b - \cos a \sin b\\
    \cos(a + b) &= \cos a \cos b - \sin a \sin b&\qquad 
    \cos(a - b) &= \cos a \cos b + \sin a \sin b\\
    \tan(a + b) &= \frac{\tan a + \tan b}{1 - \tan a \tan b}&
    \tan(a - b) &= \frac{\tan a - \tan b}{1 + \tan a \tan b}\\
\end{alignat*}

\begin{shortque}[]{}
    \qitem{%
        Find $\sin 15^{\circ}$ and $\cos 15^{\circ}$.
        }{%
        $AG=GB\rightarrow OG\perp AB$, we also have $OH=OG$. $OE=OE$,\\$\angle OGE=\angle OHE=90^{\circ}$, hence $\triangle OGE\simeq OHE$.\\And hence $EG=EH$, $BE=DE$.
        }{%
    }
    \qitem{%
        Express $\sin 2a,\cos 2a,\tan 2a$ in terms of $\sin a$ and $\cos a$. And express $\sin ^2a$ and $\cos ^2a$ in terms of $\cos 2a$.
        }{%
        \begin{align*}
            \sin 2a &= 2\sin a \cos a\\
            \cos 2a &= \cos^2a-\sin^2a\\
                    &= 2\cos^2a-1\\
                    &= 1-2\sin^2a\\
            \tan 2a &= \frac{2\tan a}{1 - \tan^2 a}\\
            \sin^2a &= \frac{1}{2}(1 - \cos 2a)\\
            \cos^2a &= \frac{1}{2}(1 + \cos 2a)\\
        \end{align*}
        }{%
    }
\end{shortque}

