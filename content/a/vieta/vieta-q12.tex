\qitem{%
    The polynomial $f(x)=x^3+2x^2+3x+4$ has roots $a,b$ and $c$. The polynomial $g(x)$ has roots $a^2,b^2$ and $c^2$. If $g(0)=8$, compute $g(2)$.
    }{%
    If $g(x)=k(x-a^2)(x-b^2)(x-c^2)$, then $k=\frac{g(0)}{-a^2b^2c^2}=-\frac{1}{2}$
$g(2)=k(2-a^2)(2-b^2)(2-c^2)=k(\sqrt{2}-a)(\sqrt{2}-b)(\sqrt{2}-c)(\sqrt{2}+a)(\sqrt{2}+b)(\sqrt{2}+c)=-kf(\sqrt{2})f(-\sqrt{2})$

$f(\sqrt{2})=8+5\sqrt{2}$
$f(-\sqrt{2})=8-5\sqrt{2}$
$-kf(\sqrt{2})f(-\sqrt{2})=-\frac{1}{2}(64-50)=\boxed{7}$
    }{%
    https://artofproblemsolving.com/community/c4t173f4h2342648_help_on_math_problem
}
