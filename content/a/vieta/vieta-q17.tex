\qitem{%
Let $x_1$ and $x_2$ be the roots of $x^2 + x -m(m+1)(m^2 +1)=0$, where $m$ is an positive integer. Suppose that both $x_1$ and $x_2$ are integers, prove that $x_1^2 + x_2^2 -x_1x_2$ is a product from 2 prime numbers.
    }{%
    I'm assuming it meant that $x_1$ and $x_2$ were the roots of $x^2 + x -m(m+1)(m^2 +1)=0$

We also know that $x_1^2 + x_2^2 -x_1x_2 = (x_1 + x_2)^2 - x_1 x_2$

By Vieta's, $x_1 + x_2 = -1$ and $x_1 x_2 = -m(m+1)(m^2 + 1)$

Then, $(x_1 + x_2)^2 - x_1 x_2 = (-1)^2 + m(m+1)(m^2+1) = m^4 + m^3 + m^2 + m + 1$

$x_1, x_2= \frac{1}{2}\left(-1\pm\sqrt{1+4m(m+1)(m^2+1)}\right)=\frac{1}{2}\left(-1\pm\sqrt{4m^4+4m^3+4m^2+4m+1}\right)$

$\Delta=a^2\Longrightarrow m=0, m=2~~~ (m=0$ not acceptable: see "positive integer")

$x_1, x_2=\frac{1}{2}\left(-1\pm \sqrt{121}\right) \Longrightarrow x_1, x_2 = 5, -6$


Other possible positive values for m, which render "$\Delta$" a square: None because

For $m>2, (2m^2+m)^2=4m^4+4m^3+m^2<4m^4+4m^3+4m^2+4m+1<(2m^2+m+1)^2=4m^4+4m^3+5m^2+2m+1$

Solving $\rightarrow x_1=5, x_2=-6$

$x_1^2 + x_2^2 -x_1x_2=25+36+30=91=7*13$

The discriminant is apparently a square only if $m=0 (*), m=2$, as possibly shown by "entrapment"

(*) to be discarded by the wording "positive integer"
    }{%
    https://artofproblemsolving.com/community/c4t173f4h3049636_vietas_formulas
}
