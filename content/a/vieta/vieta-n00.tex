\mysection{Vieta's Formulas}

\begin{mysubsection}{}
    Vieta's Formulas can be used to relate the sum and product of the roots of a polynomial to its coefficients.

    Similarly, for cubics equation, we have
    \[x^3+ax^2+bx+c=(x-p)(x-q)(x-r).\]
    We will get $a=-(p+q+r)$, $b=pq+qr+pr$, $c=-pqr$.

    Generally, let $P(x)$ be a polynomial of degree $n$, so $P(x)={a_n}x^n+{a_{n-1}}x^{n-1}+\cdots+{a_1}x+a_0$, where the coefficient of $x^{i}$ is ${a}_i$ and $a_n \neq 0$. We can also write $P(x)=a_n(x-r_1)(x-r_2)\cdots(x-r_n)$, where ${r}_i$ are the roots of $P(x)$ by factor theorem. We thus have
    \begin{alignat*}{1}
        &a_nx^n+a_{n-1}x^{n-1}+\cdots+a_1x+a_0\\
        &= a_n(x-r_1)(x-r_2)\cdots(x-r_n)\\
        &= a_nx^n - a_n(r_1+r_2+\!\cdots\!+r_n)x^{n-1} + a_n(r_1r_2 + r_1r_3 +\! \cdots\! + r_{n-1}r_n)x^{n-2} +\! \cdots\! + (-1)^na_n r_1r_2\cdots r_n.
    \end{alignat*}
    And hence we have the following general form:

    \begin{theorem}[thm:]{General Form of Vieta's Formulas}
        For a polynomial of the form $f(x)=a_nx^n+a_{n-1}x^{n-1}+...+a_1x+a_0$ with roots $r_1,r_2,r_3,...r_n$, Vieta's formulas state that: 
        \begin{alignat*}{1}
            r_1+r_2+r_3+...+r_n&=-\frac{a_{n-1}}{a_n} \\ r_1r_2+r_1r_3+..+r_{n-1}r_n&=\frac{a_{n-2}}{a_n} \\ r_1r_2r_3+r_1r_2r_4+...+r_{n-2}r_{n-1}r_n&=-\frac{a_{n-3}}{a_n} \\ &\vdots \\ r_1r_2r_3...r_n&=(-1)^n\frac{a_0}{a_n} 
        \end{alignat*}
    \end{theorem}
\end{mysubsection}


\begin{shortque}[]{3}
    \qitem{%
        If $x_1,x_2$ are the roots of the equation $x^2+5x-3=0$, determine the value of $x_1^2+x_2^2$.
        }{%
        Since $x_1^2+x_2^2=(x_1+x_2)^2-2x_1x_2$, we have $x_1^2+x_2^2=(-5)^2-2\cdot (-3)=31$.
        }{%
        https://www.math10.com/problems/problems-using-vietas-formulas/normal/
    }

    \qitem{%
        If $x_1,x_2$ are the roots of the equation $x^2+9x+33=0$, determine the value of $\frac{1}{x_1}+\frac{1}{x_2}$.
        }{%
        We have $\frac{1}{x_1}+\frac{1}{x_2}=\frac{x_1+x_2}{x_1x_2}=\frac{-9}{33}=-\frac{3}{11}$.
        }{%
        <++>
    }

    \qitem{%
        If $x_1,x_2$ are the roots of the equation $x^2-12x+19=0$, determine the value of $x_1(1-x_1)+x_2(1-x_2)$.
        }{%
        $x_1(1-x_1)+x_2(1-x_2)=x_1+x_2-(x_1^2+x_2^2)=(x_1+x_2)(1-(x_1+x_2))+2x_1x_2=12(1-12)+2\cdot 19=-94$.
        }{%
        <++>
    }

    \qitem{%
        Find the value of $x_1^2+x_2^2+x_3^2$ where $x_1,x_2,x_3$ are the roots of the euqation $3x^3-2x^2+5x-7=0$.
        }{%
        $x_1^2+x_2^2+x_3^2=(x_1+x_2+x_3)^2-2(x_1x_2+x_2x_3+x_3x_1)=(\frac{2}{3})^2-2(\frac{5}{3})=-\frac{26}{9}$.
        }{%
        <++>
    }
\end{shortque}

\begin{mysubsection}{Constructing Polynomials}
    Since Vieta formula works the other way round, we can recreate the equation which satisfies the sum of roots and product of roots.

    \setcounter{qcounter}{0}
    \eitem{%
        Let $a$ and $b$ be the roots of $x^2-3x-1=0$. Find a quadratic equation whose roots are $a^2$ and $b^2$. And compute $\frac{1}{a+1}+\frac{1}{b+1}$. (Hint: find a quadratic equation whose roots are $\frac{1}{a+1}$ and $\frac{1}{b+1}$).
        }{%
        We have $a^2b^2=(ab)^2=(-1)^2=1$, $a^2+b^2=(a+b)^2-2ab=3^2-2(-1)=11$. Hence $x^2-11x+1=0$ is the required equation.

        The eqation $(x-1)^2-3(x-1)-1=0$, or $x^2-5x+3=0$ has roots $a+1$ and $b+1$. Similarly $(\frac{1}{x}^2-5(\frac{1}{x})+3=0$, or $1-5x+3x^2=0$ has roots $\frac{1}{a+1}$ and $\frac{1}{b+1}$. Hence $\frac{1}{a+1}+\frac{1}{b+1}=\frac{5}{3}$.
    }
\end{mysubsection}

