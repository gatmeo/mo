\qitem{%
    The altitudes bisect the angles of the triangle $H_AH_BH_C$ (so $H$ is it's incenter).
    }{%
    \begin{wrapfigure}{r}{2in}
        \asycode{
            import MOgeom;
            size(1.5inch);
            pair A=origin, B=(6,0), C=(2,5), H=oct(A,B,C);
            pair[] Hu=foots(A,B,C);
            draw(cycv(A,B,C));
            drawalt(A,B,C);
            labelpt(tp(A,B,C), strabc, td(W,E,N));
            labelapd(Hu, "H_", strabc, td(E,W,S));
            label("$H$",H,2SSE);
        }
    \end{wrapfigure}
    We have $\angle H_A H_C B=C, \angle H_B H_CA=C$, hence $\angle CH_CH_A=90-C=\angle CH_CH_B$.
    }{%
    <++>
}

\qitem{%
    The reflections of $H$ across $BC, CA, AB$ lie on the circumcircle of $\triangle ABC$.
    }{%
    We have $\angle BCH=90-B, \angle CBH=90-C$ hence we have $\angle CHB=180-A$, and since $\angle CH'B+A=180$, we have that they are concyclic.
    }{%
    <++>
}

