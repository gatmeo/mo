\qitem{%
    Let $ABC$ be a triangle with$\angle B > \angle C$. The angle bisector of $\angle A$ intersects $BC$ at $D$. The perpendicular from $B$ to $AD$ intersects the circumcircle of $\triangle ABD$ again at $E$. Prove that the circumcenter of $\triangle ABC$ lies on the line $AE$.
    }{%
    \myrightasy[2.5in]{
        Let the circumcenter of $\triangle ABC$ be $O$. We have
        $$\angle OAB=\tfrac12(180^\circ-\angle AOB)=\tfrac12(180^\circ-2\angle ACB)=90^\circ-\angle ACB$$Also, we have:
        $$\angle EBA=90^\circ-\angle BAD =90^\circ-\tfrac12\angle BAC$$and
        $$\angle AEB=\angle ADB=180^\circ-\angle BAD-\angle ABC=180^\circ-\tfrac12\angle BAC-\angle ABC$$Therefore,
        \begin{align*} \angle EAB&=180^\circ-\angle EBA-\angle AEB\\ &=180^\circ-(90^\circ-\tfrac12\angle BAC)-(180^\circ-\tfrac12\angle BAC-\angle ABC)\\ &=\angle BAC+\angle ABC-90^\circ\\ &=(180^\circ-\angle ACB)-90^\circ\\ &=90^\circ-\angle ACB\\ &=\angle OAB \end{align*}Therefore, $A,E,O$ all lie on the same line.
        }{
        import MOgeom;
        size(2inch);
        pair A=(0,2),B=(-1,0),C=(2,0);
        pair D = intersectionpoint(B--C,A--(A+(bisectorpoint(B,A,C)-A)*3));
        pair E = intersectionpoint(circumcircle(A,B,D),foot(B,A,D)--(B+(foot(B,A,D)-B)*3));
        draw(A--B--C--cycle);
        draw(circumcircle(A,B,D));
        draw(A--D);draw(B--E--A);
        dot(circumcenter(A,B,C));
        label("$O$",circumcenter(A,B,C),dir(50));
        label("$E$",E,dir(-20));
        label("$A$",A,dir(75));
        label("$D$",D,dir(-60));
        label("$B$",B,SW);
        label("$C$",C,SE);
    }
    }{%
    https://artofproblemsolving.com/community/c4t41805f4h1847981_position_of_circumcenter
}

