\mysection{Triangle Centres}


\begin{mysubsection}{Centroid (G)}
    \begin{wrapfigure}{r}{2in}
        \asycode{
            import MOgeom;
            size(2inch);
            pair A=origin, B=(6,0), C=(2,5), I=ctd(A,B,C);
            pair[] M=mps(A,B,C);
            draw(cycv(A,B,C));
            drawmed(A,B,C);
            labelpt(tp(A,B,C), strabc, td(W,S,N));
            add(pathticks(A--M[2],2,0.5,0,25,red));
            labelapd(M, "M_", strabc, td(E,W,S));
            label("$G$",I,SSW);
        }
    \end{wrapfigure}
    \nbf{Construction: }Intersection of medians

    \nbf{Position: }Always lies within $\triangle ABC$
\end{mysubsection}

\begin{example*}[exp:]{Properties}
    \qitem{%
        The area of the 6 triangles are the same
        }{%
        $GAM_C=GBM_C$, $CBG=ABG$, $CGM_A=BGM_A$.
        \asycode{
            import MOgeom;
            size(2inch);
            pair A=origin, B=(6,0), C=(2,5), I=ctd(A,B,C);
            pair[] M=mps(A,B,C);
            draw(cycv(A,B,C));
            drawmed(A,B,C);
            labelpt(tp(A,B,C), strabc, td(W,S,N));
            add(pathticks(A--M[2],2,0.5,0,25,red));
            labelapd(M, "M_", strabc, td(E,W,S));
            label("$G$",I,SSW);
        }
        }{%
        <++>
    }

    \qitem{%
        $AG:GM_A = BG:GM_B = CG:GM_C = 2:1$
        }{%
        $AG:GM_A=\frac{[CGA]}{[CGM_A]}$.
        }{%
        <++>
    }
\end{example*}

\mynewpage

\begin{mysubsection}{Incentre (I)}
    \begin{wrapfigure}{r}{2in}
        \asycode{
            import MOgeom;
            size(2inch);
            pair A=origin, B=(6,0), C=(2,5), I=incenter(A,B,C);
            path ic=incircle(A,B,C);
            pair[] Iu=itps(A,B,C);
            draw(cycv(A,B,C));
            setpt(tp(A,B,C),I);
            setpt(Iu,I);
            draw(ic);
            labelpt(tp(A,B,C), strabc, td(W,S,N));
            labelapd(Iu, "I_", strabc, td(E,W,S));
            label("$I$",I,SSW);
        }
    \end{wrapfigure}

    \nbf{Construction: }Intersection of angle bisectors

    \nbf{Position: }Always lies within $\triangle ABC$
\end{mysubsection}

\begin{example*}[exp:]{Properties}
    \qitem{%
        Equal distance to \textbf{sides} of $\triangle ABC$
        }{%
        $IBI_A\equiv IBI_C$.
        }{%
        <++>
    }

    \qitem{%
        $|AB'|=|AC'|=s-a, |BA'|=|BC'|=s-b, |CA'|=|CB'|=s-c$.
        }{%
        <++>
        }{%
        <++>
    }

    \qitem{%
        \nbf{In-radius: }$A=\frac{1}{2} pr = sr$
        }{%
        \asycode{
            import MOgeom;
            size(2inch);
            pair A=origin, B=(6,0), C=(0,5), I=incenter(A,B,C);
            path ic=incircle(A,B,C);
            pair[] Iu=itps(A,B,C);
            draw(cycv(A,B,C));
            setpt(tp(A,B,C),I);
            setpt(Iu,I);
            draw(ic);
            labelpt(tp(A,B,C), strabc, td(W,S,N));
            labelapd(Iu, "I_", strabc, td(E,W,S));
            label("$I$",I,SSW);
        }
        }{%
        <++>
    }

    \qitem{%
        For right angled-triangle, in-radius equals $\frac{1}{2} (a+b-c)$.
        }{%
        <++>
        }{%
        <++>
    }

    \qitem{%
        Let the internal angle bisector $A$ intersect $BC$ at $T$. Prove that
        \begin{alignat*}{1}
            \frac{\left|AB\right|}{\left|AC\right|}=\frac{\left|BT\right|}{\left|TC\right|}.
        \end{alignat*}
        This is what we call angle bisector theorem
        }{%
        \begin{figure}[H]
            \asycode{
                import math;
                include graph;
                import olympiad;
                import geometry;
                import CSE5;
                size(2inch);
                pair A,B,C,D,X,I;
                B=origin; C=(30,0); A=(10,20); I=incenter(A,B,C);
                D=(12,0);
                draw(A--B--C--A);
                label("$A$",A,N);
                label("$B$",B,S);
                label("$C$",C,S);
                label("$D$",D,S);
                MA(0,"",10,black,B,A,D,5,black);
                MA(0,"",10,black,D,A,C,5,black);
                draw(A--D);
            }
        \end{figure}
        $\frac{AB}{BD}=\frac{\sin\phi }{\sin\theta }$,
        $\frac{AC}{CD}=\frac{\sin\phi }{\sin\theta }$
        }{%
        <++>
    }
\end{example*}
\mynewpage

\begin{mysubsection}{CircumCentre (O)}
    \begin{wrapfigure}{r}{2in}
        \asycode{
            import MOgeom;
            size(2inch);
            pair A=origin, B=(6,0), C=(2,5), O1=circumcenter(A,B,C);
            path W1=ccc(A,B,C);
            draw(W1);
            pair M[]=mps(A,B,C);
            draw(cycv(A,B,C));
            setpt(tp(A,B,C),O1);
            setpt(M,O1);
            labelpt(tp(A,B,C), strabc, td(W,S,N));
            labelapd(M, "M_", strabc, td(E,W,S));
            label("$O$",O1,2SSW);
        }
    \end{wrapfigure}

    \nbf{Construction: }Intersection of perpendicular bisectors of sides $AB, AC\textnormal{ and }BC$

    \nbf{Position: }

    Lies within $\triangle ABC$ if $\triangle ABC$ is a acute triangle

    Lies on mid-point of hypotenuse $BC$ if $\triangle ABC$ is a right-angled triangle

    Lies outisde $\triangle ABC$ if $\triangle ABC$ is obtuse triangle
\end{mysubsection}


\begin{example*}[exp:]{Properties}
    \qitem{%
        Equal distance to \textbf{vertices} ABC and form isosceles triangles with centre $O$.
        }{%
        <++>
        }{%
        <++>
    }

    \qitem{%
        \nbf{Circum-radius: }$\dfrac{a}{\sin A}=\dfrac{b}{\sin B}=\dfrac{c}{\sin C} = 2R$.
        }{%
        \noindent\begin{minipage}{.5\linewidth}
            \begin{figure}[H]
                \asycode{
                    import MOgeom;
                    size(1.5inch);
                    pair A=origin, B=(6,0), C=(0,5), O1=circumcenter(A,B,C);
                    path W1=ccc(A,B,C);
                    draw(W1);
                    dot(O1);
                    draw(cycv(A,B,C));
                    labelpt(tp(A,B,C), strabc, td(W,S,N));
                    label("$O$",O1,2SSW);
                }
            \end{figure}
        \end{minipage}
        \begin{minipage}{.45\linewidth}
            \begin{figure}[H]
                \asycode{
                    import MOgeom;
                    size(1.5inch);
                    pair A=origin, B=(7,2.5), C=(0,5), O1=circumcenter(A,B,C);
                    path W1=ccc(A,B,C);
                    draw(W1);
                    dot(O1);
                    draw(cycv(A,B,C));
                    labelpt(tp(A,B,C), strabc, td(W,S,N));
                    label("$O$",O1,2SSW);
                }
            \end{figure}
        \end{minipage}
        }{%
        <++>
    }

    \qitem{%
        Area of triangle $ABC=\dfrac{abc}{4R}$.
        }{%
        \begin{alignat*}{1}
            [ABC]&= \frac{1}{2}ab \sin C\\
                 &= \frac{1}{2} ab\frac{c}{2R}\\
                 &= \frac{abc}{4R}
        \end{alignat*}
        }{%
        <++>
    }
\end{example*}

\mynewpage


\begin{mysubsection}{Orthocentre (H)}
    \begin{wrapfigure}{r}{2in}
        \asycode{
            import MOgeom;
            size(2inch);
            pair A=origin, B=(6,0), C=(2,5), H=oct(A,B,C);
            pair[] Hu=foots(A,B,C);
            draw(cycv(A,B,C));
            drawalt(A,B,C);
            labelpt(tp(A,B,C), strabc, td(W,E,N));
            labelapd(Hu, "H_", strabc, td(E,W,S));
            label("$H$",H,2SSE);
        }
    \end{wrapfigure}

    \nbf{Construction: }Intersection of altitudes

    \nbf{Position: }

    Lies within $\triangle ABC$ if $\triangle ABC$ is a acute triangle

    Lies on pint $A$ if $\triangle ABC$ is a right-angled triangle (with $\angle A = 90^\circ$)

    Lies outisde $\triangle ABC$ if $\triangle ABC$ is obtuse triangle
\end{mysubsection}

\begin{example*}[exp:]{Properties}
    \qitem{%
        List the similar triangles.
        }{%
        Triangles $AH_BH_C, H_ABH_C, H_AH_BC$ are similar to $ABC$.
        }{%
        <++>
    }

    \qitem{%
        Let the midpoint of $BC, CA, AB$ be $X,Y,Z$ respectively.
        \begin{alignat*}{1}
            \frac{AH}{OX}=\frac{BH}{OY}=\frac{CH}{OZ}=2.
        \end{alignat*}
        }{%
        \begin{figure}[H]
            \asycode{
                import MOgeom;
                size(2inch);
                pair A=origin, B=(6,0), C=(2,5), H=oct(A,B,C), G=ctd(A,B,C);
                pair O=cct(A,B,C);
                pair[] Hu=foots(A,B,C);
                pair[] M=mps(A,B,C);
                draw(cycv(A,B,C));
                labelpt(tp(A,B,C), strabc, td(W,E,N));
                labelapd(Hu, "H_", strabc, td(E,W,S),true);
                labelapd(M, "M_", strabc, td(E,W,S),true);
                label("$H$",H,S);
                label("$G$",G,S);
                label("$O$",O,S);
                dot(H);
                dot(G);
                dot(O);
            }
        \end{figure}
        Circumcentre of $ABC$ same as orthocentre of $M_AM_BM_C$.
        }{%
        <++>
    }

    \qitem{%
        From the section on centroids, conclude that $H, G, O$ are collinear. This line is called the Euler Line of triangle $ABC$.
        }{%
        <++>
        }{%
        <++>
    }
\end{example*}
\mynewpage

