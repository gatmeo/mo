\mysection{Angles in Triangle Centres}

\begin{mysubsection}{Incentre (I)}
    \begin{wrapfigure}{r}{2in}
        \asycode{
            import MOgeom;
            size(2inch);
            pair A=origin, B=(6,0), C=(2,5), I=incenter(A,B,C);
            path ic=incircle(A,B,C);
            pair[] Iu=itps(A,B,C);
            draw(cycv(A,B,C));
            SP(TP(A,B,C),I);
            SP(Iu,I);
            draw(ic);
            labelpt(TP(A,B,C), strabc, td(W,S,N));
            labelapd(Iu, "I_", strabc, td(E,W,S));
            label("$I$",I,SSW);
        }
    \end{wrapfigure}

    \nbf{Construction: }Intersection of angle bisectors

    \nbf{Position: }Always lies within $\triangle ABC$
\end{mysubsection}

\begin{shortque*}[]{}
    \qitem{%
        Find $\angle BII_C$.
        }{%
        $90-B/2$ or $C/2+A/2$.
        }{%
        <++>
    }
    
    \qitem{%
        Find $\angle AIB$.
        }{%
        $180-A/2-B/2$ or $90+C/2$.
        }{%
        <++>
    }
\end{shortque*}

\begin{mysubsection}{CircumCentre (O)}
    \begin{wrapfigure}{r}{2in}
        \asycode{
            import MOgeom;
            size(2inch);
            pair A=origin, B=(6,0), C=(2,5), O1=circumcenter(A,B,C);
            path W1=ccc(A,B,C);
            draw(W1);
            pair M[]=mps(A,B,C);
            draw(cycv(A,B,C));
            SP(TP(A,B,C),O1);
            SP(M,O1);
            labelpt(TP(A,B,C), strabc, td(W,S,N));
            labelapd(M, "M_", strabc, td(E,W,S));
            label("$O$",O1,2SSW);
        }
    \end{wrapfigure}

    \nbf{Construction: }Intersection of perpendicular bisectors of sides $AB, AC\textnormal{ and }BC$

    \nbf{Position: }

    Lies within $\triangle ABC$ if $\triangle ABC$ is a acute triangle

    Lies on mid-point of $BC$ if $\triangle ABC$ is a right-angled triangle

    Lies outisde $\triangle ABC$ if $\triangle ABC$ is obtuse triangle
\end{mysubsection}

\begin{shortque*}[]{}
    \qitem{%
        Find $\angle AOB$.
        }{%
        $2C$.
        }{%
        <++>
    }
    
    \qitem{%
        Find $\angle ABO$.
        }{%
        $90-C$.
        }{%
        <++>
    }

    \qitem{%
        Find $AB$ in terms of $R$.
        }{%
        $2R\sin C$.
        }{%
        <++>
    }
\end{shortque*}

\mynewpage

\begin{mysubsection}{Orthocentre (H)}
    \begin{wrapfigure}{r}{2in}
        \asycode{
            import MOgeom;
            size(2inch);
            pair A=origin, B=(6,0), C=(2,5), H=oct(A,B,C);
            pair[] Hu=foots(A,B,C);
            draw(cycv(A,B,C));
            drawalt(A,B,C);
            labelpt(TP(A,B,C), strabc, td(W,E,N));
            labelapd(Hu, "H_", strabc, td(E,W,S));
            label("$H$",H,2SSE);
        }
    \end{wrapfigure}

    \nbf{Construction: }Intersection of altitudes

    \nbf{Position: }

    Lies within $\triangle ABC$ if $\triangle ABC$ is a acute triangle

    Lies on $A$ if $\triangle ABC$ is a right-angled triangle (with $\angle A = 90^\circ$)

    Lies outisde $\triangle ABC$ if $\triangle ABC$ is obtuse triangle
\end{mysubsection}

\begin{shortque*}[]{}
    \qitem{%
        Find $\angle HAB$.
        }{%
        $90-B$.
        }{%
        <++>
    }

    \qitem{%
        Find $\angle BHC$.
        }{%
        $180-A$.
        }{%
        <++>
    }
    
    \qitem{%
        Find $\angle AHH_C$.
        }{%
        $B$.
        }{%
        <++>
    }

    \qitem{%
        Find $AH$ in terms of $R$.
        }{%
        $\angle HAH_B=90-C$,

        $AH=\dfrac{AH_B}{\cos(90-C)}=\frac{AB\cos A}{\sin C}=\dfrac{c}{\sin C}\cos A=2R\cos A$
        }{%
        <++>
    }

    \qitem{%
        Find $H_AH$ in terms of $R$.
        }{%
        $\angle HBH_A=90-C$,

        $HH_A=BH_A\tan(90-C)=AB\cos B\dfrac{\cos C}{\sin C}=\dfrac{c}{\sin C} \cos B \cos C=2R\cos B \cos C$
        }{%
        <++>
    }

\end{shortque*}
