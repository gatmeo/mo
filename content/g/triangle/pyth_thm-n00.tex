\mysection{Pythagoras' Theorem}

\begin{mysubsection}{}
    \begin{theorem}[thm:]{Pythagoras' Theorem}
        For any right-angled triangle, if the lengths of the legs are $a$ and $b$ while that of the hypotenuse is $c$, then $a^2+b^2=c^2$.
    \end{theorem}

    \begin{theorem}[thm:]{Converse of Pythagoras' Theorem}
        For any triangle with the lengths $a, b \textnormal{ and }c$, if $a^2+b^2=c^2$, then this triangle is right-angled.
    \end{theorem}

    \begin{theorem}[thm:]{Extention of Pythagoras' Theorem}
        For a triangle with side lengths $a, b \textnormal{ and }c,$ where $a\leq b\leq c$, if

        \begin{itemize}
            \item $a^2+b^2=c^2,$ then it is a \textbf{right-angled }triangle.
            \item $a^2+b^2>c^2,  $then it is a \textbf{acuted-angled }triangle.
            \item $a^2+b^2<c^2, $then it is a \textbf{obtuse-angled }triangle.
        \end{itemize}

    \end{theorem}

    \nbf{Examples of Pyth. triples: }(3, 4, 5), (5, 12, 13), (8, 15, 17), (7, 24, 25), (20, 21, 29).
\end{mysubsection}
