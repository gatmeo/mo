\mysection{Tangents to a Circle}

\begin{mysubsection}{}
    \begin{definition}[def:]{Tangents}
        \rightasy[1.5in]{
            import MOgeom;
            pair Q=(2,-.5), T=tg(Q,D(CR(o,1))), P=(-2,-1);
            pair A=MD("A",IP(P--Q,CR(o,1))), B=MD("B",OP(P--Q,CR(o,1)));
            D(MD("P",P)--MD("Q",Q)--MD("P'",WP(L(Q,MD("T",T)),1)));
        }
        Consider a circle and a line, there is 3 possible cases when we draw them on the same place:
        \begin{enumerate}
            \item The line intersects the circle at two distinct points $A$ and $B$.
            \item The line intersects the circle at only one point $T$. This line is called the tangent to the circle at $T$, and $T$ is called the point of contact.
            \item The line does not intersect the circle.
        \end{enumerate}
    \end{definition}

    \rightasy[1.5in]{
        import MOgeom;
        pair P=(-1.1,-1), T=(0,-1), Q=(1.1,-1);
        D(MD("O",o,N)--MD("T",T));
        D(CR(o,1));
        D(MD("P",P)--MD("Q",Q));
        MR(o,T,Q);
    }
    \begin{theorem}[thm:]{tangent $\perp $ radius}
        If $PQ$ is the tangent to the circle at $T$, then $PQ\perp OT$.
    \end{theorem}

    \righttext[1.5in]{}
    \begin{theorem}[thm:]{converse of tangent $\perp $ radius}
        Let $OT$ be a radius of the circle and $PTQ$ be a straight line. If $PQ\perp OT$, then $PQ$ is the tangent to the circle at $T$.
    \end{theorem}

    \righttext[1.5in]{}
    \begin{theorem}[thm:]{$\perp $ to tangent at its point of contact passes through centre}
        The perpendicular to a tangent $PQ$ at its point of contact $T$ passes through the centre $O$ of the circle.
    \end{theorem}
\end{mysubsection}

\begin{shortque}[]{}
    \qitem{%
        \rightasy[1.5in]{
            import MOgeom;
            pair A=dir(-160), B=dir(-40), C=dir(60), T=dir(125), P=WP(L(B,A,.6),1);
            D(CR(o,1));
            D(D("T",T)--D("A",A)--D("O",o,E)--T--D("C",C)--D("B",B)--D("A",A)--D("P",P)--T);
        }
        In the figure, $PT$ is the tangent to the circle at $T$. $TABC$ is a cyclic quadrilateral. If $\angle TPA=42^{\circ}, \angle TCB=100^{\circ}$ and $PAB$ is a straight line, find $\angle TOA$.
        }{%
        $\angle TAP=\angle TCB=100$, In $\triangle APT$, $\angle PTA=38^{\circ}$.\\Since $\angle PTO=90^{\circ}$, $\angle ATO=52^{\circ}$, $\angle TOA=180-52\times 2=76^{\circ}$.
        }{%
    }
\end{shortque}

\newpage
\begin{mysubsection}{Tangents from an External Point}
    For any point outside a circle, we can always draw two tangents to the circle.

    \begin{theorem}[thm:]{tangent properties}
        \rightasy[1.5in]{
            import MOgeom;
            path c=CR(o,1);
            pair T=(2.3,0), P=tg(T,D(c)), Q=tg(T,c,2);
            pen q=dashed+blue;
            D(WP(L(T,Q,0,.3),1)--T--WP(L(T,P,0,.3),1));
            SP(TP(D("P",P),D("Q",Q),D("T",T)),D("O",o,W),q);
            MR(o,Q,T);
            MR(o,P,T);
        }
        If two tangents, $TP$ and $TQ$, are drawn to a circle from an external point $T$ and touch the circle at $P$ and $Q$ respectively, then
        \begin{enumerate}
            \item $TP=TQ$,
            \item $\angle POT=\angle QOT$,
            \item $\angle PTO=\angle QTO$.
        \end{enumerate}
    \end{theorem}
\end{mysubsection}

\begin{shortque}[]{}
    \qitem{%
        \rightasy[1.5in]{
            import MOgeom;
            pair A=(0,2), B=dir(150), C=dir(30), P,D,Q;
            Q=WP(A--C,0.7); D=tg(Q,D(CR(o,1)),1); P=IP(A--B,L(Q,D,6));
            D(D("C",C)--D("B",B)--D("A",A)--D("Q",Q)--D("D",D,S)--D("P",P));
            D(B--D--C);
            MA("56^{\circ}",A,Q,P,5,f=8);
            MAR("64^{\circ}",A,P,Q,5,f=8);
            D(L(A,B,0,0.4));
            D(L(A,C,0,0.4));
        }
        In the figure, $AB$ and $AC$ are tangents to the circle at $B$ and $C$ respectively. $P$ and $Q$ are points on $AB$ and $AC$ respectively such that $PQ$ touches the circle at $D$. It is given that $\angle APQ=64^{\circ}$ and $\angle AQP=56^{\circ}$. Find $\angle CBD, \angle DCB$.
        }{%
        $\angle BAC=60^{\circ}$, $AB=AC$, $\triangle ABC$is an equilateral triangle.\\$DQ=CQ$, $\angle QDC=\angle QCD=28^{\circ}$. $\angle DCB=60^{\circ}-28^{\circ}=32^{\circ}$.\\Similarly we have $\angle ABD=32^{\circ}$, $\angle DBC=60-32=28^{\circ}$.
        }{%
    }
\end{shortque}
\newpage

\begin{mysubsection}{Angles in the Alternate segment}
    \begin{definition}[def:]{Tangent-chord angles and Angle in alternate segment}
        \rightasy[1.5in]{
            import MOgeom;
            pair P=(-1.1,-1), T=(0,-1), Q=(1.1,-1), A=dir(60),B=dir(190);
            D(D("A",A)--D("B",B)--D("T",T)--cycle);
            D(CR(o,1));
            D(MD("P",P)--MD("Q",Q));
            MA(Q,T,A,10);
            MA(T,B,A,10);
        }

        The angles between tangent and chord passing trhough the point of contact are called tangent-chord angles.\\The sagment opposed to one of the tangent-chord angles is called alternate segment, and the angle wihtin the segment is call angle in the alternate segment.
    \end{definition}
        \\\\\nbf{Proof:}
        $\angle AB'T+\angle ATB'=90, \angle ATB'+\angle ATQ=90$, $\angle AB'T=\angle ATQ$
        \vspace{1em}

        \rightasy[1.5in]{
            import MOgeom;
            pair Q=(-1.1,-1), T=(0,-1), P=(1.1,-1), A=dir(60),B=dir(-40);
            D(D("A",A)--D("B",B)--D("T",T)--cycle);
            D(CR(o,1));
            D(MD("Q",P)--MD("P",Q));
            MA(A,T,Q,10);
            MAR(T,B,A,10);
        }
        \begin{theorem}[thm:]{$\angle $ in alt. segment}
            A tangent-chord angle of a circle is equal to an angle in the alternate segment.
        \end{theorem}

    \begin{theorem}[thm:]{converse of $ag$ in alt. segment}
        \righttext[1.5in]{}
        Suppose $A,B$ and $T$ are points on the circle. If $TP$ is a straight line such that $\angle ATP=\angle ABT$, then $TP$ is the tangent to the circle at $T$.
    \end{theorem}
\end{mysubsection}

\begin{shortque}[]{}
    \qitem{%
        \rightasy[1.5in]{
            import MOgeom;
            pair P=(-1.5,-1), B=(0,-1), A=tg(P,D(CR(o,1)),2), C=dir(0), D=dir(30), F=WP(L(P,A,.6),1);
            D(D("F",F)--D("A",A)--D("P",P)--D("B",B)--D("D",D)--A--D("C",C)--D--B--A);
        }
        In the figure, $PF$ is the tangent to the circle at $A$. $B$ is a point on the circle such that $AD\parallel PB$. $\angle FAD=56^{\circ}, \angle DAC=24^{\circ}$ and $\angle BDC=38^{\circ}$. Is $PB$ the tanget to the circle at $B$?
        }{%
        We have $\angle BAC=\angle BDC=38^{\circ}$, $\angle ABD=\angle FAD=56^{\circ}$.\\In $\triangle ABD$, we have $\angle ADB=190-56-(38+24)=62^{\circ}$.\\Since $\angle PBA=\angle BAD=38+24=62^{\circ}=\angle ADB$,\\$PB$ is the tangent at $B$.
        }{%
    }
\end{shortque}
