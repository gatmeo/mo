\mysection{Angles in a Circle}

\begin{mysubsection}{}
    \begin{definition}[def:]{Angle at centre and angle at circumference}
        If the vertex of the angle subtended by an arc lies at the centre, then this angle is called the angle at the centre.

        If the vertex of the angle subtended by an arc lies at the circumference, then this angle is called the angle at the circumference.
    \end{definition}

    \begin{theorem}[thm:]{$\angle $ at centre twice $\angle $ and circumference}
        The angle at the centre of a circle subtended by an arc is twice the angle at the circumference subtended by the same arc.
        \begin{figure}[H]
            \centering
            \asycode{
                import MOgeom;
                size(1.5inch);
                pair A=dir(-135), B=dir(-45), O=origin,P=(0,1);
                path c=DCR(O,1);
                D(D("A",A)--D("O",o)--D("B",B)--D("P",P)--A);
                MA("p",A,P,B,5);
                MA("q",A,o,B,5);
            }
            \quad
            \asycode{
                import MOgeom;
                size(1.5inch);
                pair A=dir(135), B=dir(45), O=origin,P=(0,1);
                path c=DCR(O,1);
                D(D("A",A)--D("O",o)--D("B",B)--D("P",P)--A);
                MA("p",A,P,B,5);
                MA("q",A,o,B,5);
            }
            \quad
            \asycode{
                import MOgeom;
                size(1.5inch);
                pair A=dir(-135), B=dir(-45), O=origin,P=dir(5);
                path c=DCR(O,1);
                D(D("A",A)--D("O",o)--D("B",B)--D("P",P)--A);
                MA("p",A,P,B,5);
                MA("q",A,o,B,5);
            }
        \end{figure}
        %\vocab{Proof:} $\angle AOQ=2x, \angle BOQ=2y$ $q=2y-2x$, $p=y-x$.
    \end{theorem}

    \rightasy[1.5in]{
        import MOgeom;
        pair A=dir(180), B=dir(0), O=origin;
        pair P=dir(40);
        path c=DCR(O,1);
        D(D("A",A)--D("O",o)--D("B",B)--D("P",P)--A);
    }
    \begin{theorem}[thm:]{$\angle $ in semi-circle}
        The angle in a semi-circle is a right angle.

        i.e. If $AB$ is a diameter, then $\angle APB=90^{\circ}$.
    \end{theorem}
    \vspace{2em}

    \begin{theorem}[thm:]{Converse of $\angle $ in semi-circle}
        If the angle at the circumference subtended by a chord is a right angle, then the chord is a diameter of a circle.

        i.e. If $\angle APB=90^{\circ} $, then  $AB$ is a diameter.
    \end{theorem}
\end{mysubsection}

\begin{shortque}[]{}
    \qitem{%
        \rightasy[1.5in]{
            import MOgeom;
            pair D=dir(0), C=dir(180), B=dir(60), A=(2*dx(B,C)-1,0), O=origin;
            path c=DCR(O,1);
            D(D("D",D)--D("B",B)--D("C",C)--D("A",A)--B);
        }
        In the figure, $CODA$ is a straight line, $BC=BA$ and $\angle CDB=70^{\circ}$. Find $\angle CAB$ and $\angle DBA$.
        }{%
        $\angle CBD=90^{\circ}$, $\angle BCD=20^{\circ}$, $\angle BAC=20^{\circ}$, $\angle DBA=70-20=50^{\circ}$.
        }{%
    }
\end{shortque}

\mynewpage

\begin{mysubsection}{}
    \rightasy[1.5in]{
        import MOgeom;
        pair A=dir(-135), B=dir(-45), Q=dir(100),P=dir(5);
        path c=D(circle(o,1));
        D(D("A",A)--D("Q",Q)--D("B",B)--D("P",P)--A);
        MA("x",A,P,B,5);
        MA("y",A,Q,B,5);
    }
    \begin{definition}[def:]{Angles in the Same Segment}
        If the angles at the circumference subtended by the same arc $AB$.

        i.e. They all lie in the major segment or the same minor segment.
    \end{definition}

    \begin{theorem}[thm:]{$\angle $s in the same segment}
        The angles in the same segment of a circle are equal.
    \end{theorem}
\end{mysubsection}

\begin{shortque}[]{}
    \qitem{%
        \rightasy[1.5in]{
            import MOgeom;
            pair A=dir(-140), B=dir(100), C=dir(40), D=dir(-40);
            path c=D(CR(o,1));
            D(D("A",A)--D("B",B)--D("D",D)--D("C",C)--D("O",o)--A--D);
        }
        $AC$ is a diameter of the circle, $\angle BAD=75^{\circ}$ and $\angle ABD=40^{\circ}$. Find $\angle BAC$.
        }{%
        $\angle ACD=\angle ABD=40^{\circ}$, $\angle ADC=90^{\circ}$, $\angle CAD=50$.\\$\angle BAC=\angle BAD-\angle CAD=25^{\circ}$. 
        }{%
        <++>
    }

    \qitem{%
        \rightasy[1.5in]{
            import MOgeom;
            pair A=dir(180), B=dir(0), R=dir(120), P=(2,0);
            path c=D(CR(o,1));
            pair Q=intersectionpoint(P--R,c);
            D(D("R",R)--D("A",A)--D("O",o)--D("B",B)--D("P",P)--R--B--A--D("Q",Q));
        }
        Chord $RQ$ and diameter $AB$ of the circle are produced to meet at $P$. If $\angle RPA=30^{\circ}$ and $\angle QAR=50^{\circ}$, find $\angle QRB$ and $\angle RQA$.
        }{%
        $\angle QRB=\angle QAB$, $\angle ARB=90^{\circ}$. In $\triangle APR$, let $\angle QAB=x$,\\
        we can solve $x=5$. In $\triangle QAP$, $\angle RQA=\angle QAP+\angle QPA=35$.
        }{%
    }
\end{shortque}
\newpage

