\qitem{%
    Many states use a sequence of three letters followed by a sequence of three digits as their standard license-plate pattern. Given that each three-letter three-digit arrangement is equally likely, the probability that such a license plate will contain at least one palindrome (a three-letter arrangement or a three-digit arrangement that reads the same left-to-right as it does right-to-left) is $\dfrac{m}{n}$, where $m$ and $n$ are relatively prime positive integers. Find
    }{%
    Consider the three-digit arrangement, $\overline{aba}$. There are $10$ choices for $a$ and $10$ choices for $b$ (since it is possible for $a=b$), and so the probability of picking the palindrome is $\frac{10 \times 10}{10^3} = \frac 1{10}$. Similarly, there is a $\frac 1{26}$ probability of picking the three-letter palindrome.

    By the Principle of Inclusion-Exclusion, the total probability is
    $\frac{1}{26}+\frac{1}{10}-\frac{1}{260}=\frac{35}{260}=\frac{7}{52}\quad\Longrightarrow\quad7+52=\boxed{059}$
    }{%
    https://artofproblemsolving.com/wiki/index.php?title=2002_AIME_I_Problems/Problem_1#Problem
}

