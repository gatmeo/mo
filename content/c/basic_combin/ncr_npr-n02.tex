\mysection{Combinatorics}

\begin{mysubsection}{}
    \begin{definition}[def:]{Injective, Surjective}
        A map $f:A\rightarrow B$ is \textbf{injective} if for all $a_1,a_2\in A$, $f(a_1)=f(a_2)$ implies $a_1=a_2$.

        A map $f:A\rightarrow B$ is \textbf{surjective} if for all $b\in B$, there exist an $a\in A$ such that $f(a)=b$.
    \end{definition}
    \nbf{Facts}. Assume $A$ and $B$ to be finite. If $f:A\rightarrow B$ is surjective, and the preimage of each $b$ has the same cardinality (i.e. $|f^{-1}(b)|=n$ for all $b$), then $|A|=n|B|$.\\
\end{mysubsection}

\begin{mysubsection}{}
    \begin{definition}[def:]{Permutation}
        Count the number of ways to arrange $r$ out of $n$ objects where \textbf{order does matter.}

        The number of ways can be calculated with
        \begin{equation*}
            P^n_r=\dfrac{n!}{(n-r)!}=n\times (n-1)\times (n-2)\times \cdots\times (n-r+1).
        \end{equation*}
    \end{definition}
    \myframebreak

    \begin{definition}[def:]{Combination}
        Count the number of ways to select $r$  out of $n$ objects where \textbf{order does not matter}.

        The number of ways can be calculated by
        \begin{equation*}
            C^n_r=\dfrac{n!}{r!(n-r)!}=\dfrac{n\times (n-1)\times  \cdots\times (n-r+1)}{1\times 2\times \cdots\times r}.
        \end{equation*}
    \end{definition}

    Properties: $C^n_r = C^n_{n-r}$ (why?)

    Compute $C^{2019}_{2017}$
\end{mysubsection}

\begin{mysubsection}{Relationship between \texorpdfstring{$C^n_r$ and $P^n_r$}{ncr and npr}}
    \nbf{Numerical relationship: }$C^n_r = \dfrac{P^n_r}{r!}$

    \nbf{Combinatoric Point of View:} 
    Consider the set of all possible permutations arrange $r$ out of $n$ elements. There will be exactly $r!$ permutation representing the same set of combination.
    For example, when choose 3 elements from a set of 5 elements $(a,b,c,d,e)$, there is $3!=6$ permutation representing the same combination of choosing 3 elements out of 5.  
    And for every combination, it matches exactly $3!$ permutation.  
    Therefore, we have $P^5_3 = 3!C^5_3$.

    \nbf{Functional Point of View:}

    We can construct a map $f:P^n_r\rightarrow C^n_r$, with each element in $C^n_r$ has $r!$ preimage.
\end{mysubsection}
