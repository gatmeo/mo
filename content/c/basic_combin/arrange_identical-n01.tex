\mysection{Arranging identical elements}

\begin{example}[exp:]{}
    \eitem{%
        Find the number of ways arranging $AAABCDE$.
        }{
        \fleft{(a) We define a map $f:Perm(A_1A_2A_3BCDE)\rightarrow Perm(AAABCDE)$}

        We can first gives different names to the object $A$, namely $A_1,A_2\textnormal{ and }A_3$.

        Then there will be $7!$ ways to arrange $A_1A_2A_3BCDE$.

        As all $A_1,A_2\textnormal{ and }A_3$ are representing the same object, arrangement $A_1A_2A_3BCDE$ and \\$A_1A_3A_2BCDE$ are representing the same arrangement. Indeed, there are $3!$ different arrangement of  $A_1A_2A_3BCDE$ representing the same $AAABCDE$ arrangement.

        Therefore, there are $\dfrac{7!}{3!}$ arrangements.

        \mycontinueframe
        \fleft{(b) (directly using $C^n_r$)}

        The questions are equivalent to asking the number of ways to put the 7 elements into 7 boxes.

        So, we can first choose 3 boxes out of 7 to put $A$ in it.

        Then, there are 4 remaining empty boxes. We can further have $C^4_1$ ways to put $B$ in it.

        And there will be 3 remaining empty boxes, ... and so on.

        Therefore, there are $C^7_3C^4_1C^3_1C^2_1C^1_1$ arrangements.
    }
\end{example}

\begin{shortque}[]{2}
    \qitem{%
        Find the number of ways to arrange 9 out of 9 elements from the set with 2A, 3B, and 4C.
        }{%
        $A_1A_2B_1B_2B_3C_1...$
        \begin{equation*}
            \dfrac{9!}{2!\cdot 3!\cdot 4!}
        \end{equation*}
        \begin{equation*}
            C^9_2 C^7_3C^4_4
        \end{equation*}
        }{%
        <++>
    }

    \qitem{%
        Find the number of ways to arrange 8 out of 9 elements from the set with 2A, 3B, and 4C.
        }{%
        It's the same

        \begin{equation*}
            \dfrac{9!}{2!\cdot 3!\cdot 4!}
        \end{equation*}
        arrange 3 out 4 students : arrange 4 out of 4

        =1:1

        ignore 1 student: 4
        
        arrange remaining: 3!

        }{%
        <++>
    }
\end{shortque}
