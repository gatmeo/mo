\mysection{Generating Functions}

\begin{definition}[def:]{Generating Functions}
    The idea behind generating functions is to create a power series (a polynomial with infinite degree) whose coefficients, $c_0,c_1,c_2,\dots $ give the terms of a sequence which is of interest, i.e. the generating function is $c_0+c_1x+c_2x^2+\cdots $ and the sequence is $c_0,c_1,c_2,\dots $. For our consideration in combinatorics, the term $x^i$ mostly represents "achieving a quantity $i$-times", and the coefficient $c_i$ denotes the number of ways of achieving that.
\end{definition}

\begin{shortque}[Introductory Question]{}
    \label{sec:generating_func_intro}
    \eitem{%
        How many ways can we get $k$ heads when flipping $n$ different coins?
        }{%
        Obviously we know the answer is $\binom{n}{k}$. But to translate this question in the generating function, we can consider the polynomial $(1+x)^n$. For a factor $(1+x)$, we can say that $x$ represents one gets a head for the coin, and $1$ represents one gets a tail. So the coefficient of $x^k$ denotes the number of ways of getting $k$ heads out of $n$ coins.
    }

    \eitem{%
        Suppose we have 6 one-dollar coin, 1 five-dollar coin, and 2 ten-dollar coin. For what prices can we give exact change? and in how many different ways?
        }{%
        Say for 16 dollars, we can always split cases of considering first that how many ten-dollar coin can we use, then brute force for five-dollar, and see if there's enough one-dollar to pay for it. Again now with the generating function concept, we have the polynomial $(1+x+x^2+\cdots +x^6)$ corresponding to using $0$ one-dollar coin, using $1,2,\dots ,6$. Similarly $(1+x^5)$ and $(1+x^{10})$ for $5$-dollar and 10-dollar coins. Hence we can just multiply the three polynomial and get the generating function for our original problem.
    }
\end{shortque}

But waht if we want to ask not only for a exact case, but a series of cases.

\begin{shortque}[Using Roots of Unity]{}
    \eitem{%
        How many ways can we get an even number of heads when flipping $n$ different coins? 
        }{%
        From the previous subsection, we know $(1+x)^n$ is the generating function. Now, we need to find the coefficient of the even degree terms: $c_0+c_2+c_4+\cdots $. We can see that
        \begin{alignat*}{1}
            (1+1)^n&= 1 + \binom{n}{1}1 + \binom{n}{2}1^2+\cdots +1^n\\
            (1+(-1))^n&= 1 + \binom{n}{1}(-1) + \binom{n}{2}(-1)^2+\cdots +(-1)^n
        \end{alignat*}
        Since $1^n+(-1)^n=0$ for $n$ odd, we see that 
        \begin{alignat*}{1}
            (1+1)^n+(1+(-1))^n&= 
            \begin{cases}
                2(1 + \binom{n}{2}1^2+\cdots +\binom{n}{n-1}1^{n-1})&\textnormal{ if }n\textnormal{ is odd. }\\
                2(1 + \binom{n}{2}1^2+\cdots +1^n)&\textnormal{ if }n\textnormal{ is even. }
            \end{cases}
        \end{alignat*}
        Hence $c_0+c_2+c_4+\cdots =\frac{1}{2}[(1+1)^n+(1+(-1))^n]=2^{n-1}$.
    }

    \qitem{%
        How many ways can we get number of heads with a multiple of 3?
        }{%
        let $\zeta=e^{2\pi i}{3}$.
        $$f(1)=\binom{n}{n}+\binom{n}{n-1}+\cdots+\binom{n}{0}$$$$f(\zeta)=\binom{n}{n}(\zeta)^n+\binom{n}{n-1}(\zeta)^{n-1}+\cdots+\binom{n}{0}(\zeta)^0$$$$f(\zeta^2)=\binom{n}{n}(\zeta)^{2n}+\binom{n}{n-1}(\zeta)^{2n-2}+\cdots+\binom{n}{0}(\zeta)^0$$and when adding them together, we get $$f(1)+f(\zeta)+f(\zeta^2)$$$$=\binom{n}{n}(1^n+\zeta^n+\zeta^{2n})+\binom{n}{n-1}(1^{n-1}+\zeta^{n-1}+\zeta^{2n-2})+\cdots+\binom{n}{0}(1^0+\zeta^0+\zeta^0)$$$$=3(\binom{n}{3m}+\binom{n}{3m-3}+\cdots+\binom{n}{0})$$So our desired expression is $\frac{1}{3}(f(1)+f(\zeta)+f(\zeta^2))$.
        }{%
        https://artofproblemsolving.com/community/c4t29821f4h1846859_making_a_book_and_need_input
    }

    \qitem{%
        Find $\binom{n}{2}+\binom{n}{5}+\binom{n}{8}+\cdots$.
        }{%
        Note that if the coefficient of $x^2$ in $(\frac{1}{3})(f(1)+f(\zeta)+f(\zeta^2))$ is $\binom{n}{2}(1+\zeta^2+\zeta^4).$ The $1$ represents $f(1)$, the $\zeta^2$ represents the $f(\zeta)$, and the $\zeta^4$ represents $f(\zeta^2)$. We want to get all of these to equal $1$. By multiplying the $\zeta^2$ by $\zeta$, and the $\zeta^4$ by $\zeta^2$, we get the desired $1$'s. Therefore, to get the value of the expression, we find the value of $(\frac{1}{3})(f(1)+\zeta f(z)+\zeta^2 f(\zeta^2))$.
        }{%
        https://artofproblemsolving.com/community/c4t29821f4h1846859_making_a_book_and_need_input
    }
\end{shortque}

\begin{mysubsection}{Infinite Generating Functions}
    Infinite generating functions is not as scary as it seems. For some generating functions, it is easier to express it as the infinite form: for the generating functioin with $c_i=0$ for all $i$, we have
    \[
        1+x+x^2+x^3+\cdots =\frac{1}{1-x}.
    \]
    Such generating function represents the case where each outcome will happen equally, i.e. for \cref{sec:generating_func_intro} question 2, if we have infinite one-dollar, five-dollar, and ten-dollar coins, then the cooresponding generating functions are $(1+x+x^2+\cdots )$, $(1+x^5+x^{10}+\cdots )$, and $(1+x^{10}+x^{20})$. Then the generating function for this new question is simply 
    \[
        f(x)=\frac{1}{1-x}\cdot \frac{1}{1-x^5}\cdot \frac{1}{1-x^{10}}.
    \]
    Say, we want to find the coefficient of $x^{50}$ of $f(x)$, a way to find it is by the observation we made before that $x^n-1=(x-1)(1+x+x^2+\cdots +x^{n-1})$. Then we have
    \begin{alignat*}{1}
        \frac{1}{\left(1-x\right)\left(1-x^5\right)\left(1-x^{10}\right)}
        &= \frac{\left(1+x+x^2+x^3+x^4\right)}{\left(1-x^5\right)^2\left(1-x^{10}\right)}\\
        &= \frac{\left(1+x+x^2+x^3+x^4\right)\left(1+x^5\right)^2}{\left(1-x^{10}\right)^3}\\
        &= \frac{\left(1+x+x^2+x^3+x^4\right)\left(1+{2x}^5+x^{10}\right)}{\left(1-x^{10}\right)^3}\\
        &= \frac{\left(1+x+x^2+x^3+x^4\right)\left(1+{2x}^5+x^{10}\right)}{\left(1-x^{10}\right)^3}\\
        &= \left(1+x+x^2+x^3+x^4\right)\left(1+{2x}^5+x^{10}\right)\sum_{k=0}^{\infty}{\binom{k+2}{2}x^{10k}}.
    \end{alignat*}
    So the coefficient of $x^{50}$ is 
    $\binom{5+2}{2}+\binom{4+2}{2}=\binom{7}{2}+\binom{6}{2}=21+15=36$.

    \newpage
    Indeed, this particular form $1/(1-x)$ has some nice property:

    \begin{shortque*}[]{}
        \qitem{%
            Show that
            \[
                \left(\frac{1}{1-x}\right)'=\frac{1}{(1-x)^2},\textnormal{ and }
                \left(\frac{1}{1-x}\right)^{(k)}=\frac{k!}{(1-x)^{k+1}}.
            \]
            }{%
            <++>
            }{%
            <++>
        }

        With this fact we can do the following question:

        \eitem{%
            Find the number of ways of distributing $15$ apples to $5$ students. 
            }{%
            Sure, we can do it with the sticks and balls method. But now with generating functions, it is not hard to see our required generating function is
            \[
                f(x)=(1+x+x^2+\cdots )^5=\left(\frac{1}{1-x}\right)^5=\frac{1}{4!}\frac{d^4}{dx^4}\frac{1}{1-x}=\frac{1}{24}\cdot \frac{d^4}{dx^4}(1+x+x^2+\cdots ).
            \]
            Hence the coefficient of $x^{15}$ is $19\cdot 18\cdot 17\cdot 16/4!=\binom{19}{4}$.
        }

        \qitem{%
            Give an alternate form of $\left(1/(1-x)\right)^n$.
            }{%
            $\frac{1}{(1-x)^n} = \binom{n-1}{n-1} + \binom{n}{n-1}x + \binom{n+1}{n-1}x^2 \cdots$.
            }{%
            <++>
        }
    \end{shortque*}
\end{mysubsection}
