\documentclass[a4paper,10pt]{article}
\usepackage[common_math_textnormal,imo,todonotes,math_base_note,math_simple]{gatmeo}
\usepackage{blkarray}
%\usepackage[math_simple,imo]{omegat}

\renewcommand{\courseTitle}{\FirstBigRestSmallCaps{IMO Phase III Level 2}}
\renewcommand{\courseTopic}{\FirstBigRestSmallCaps{Matroids}}
\DTMsavedate{mydate}{2023-03-04}
\renewcommand{\vocab}[1]{\textbf{#1}}

%\toggletrue{ownans}

\newcommand{\ee}{\mathbf{e}}
\newcommand{\II}{\mathcal{I}}
\newcommand{\IIm}{\mathcal{I}^{\max}}
\newcommand{\CC}{\mathcal{C}}
\newcommand{\MM}{\mathcal{M}}
\newcommand{\PP}{\mathcal{P}}
\renewcommand{\AA}{\mathcal{A}}
\newcommand{\BB}{\mathcal{B}}
\newcommand{\hh}{\mathbf{h}}
\newcommand{\vv}{\mathbf{v}}
\newcommand{\uu}{\mathbf{u}}
\newcommand{\xx}{\mathbf{x}}
\newcommand{\ga}{\mathfrak{a}}
\newcommand{\ff}{\mathbf{f}}
\renewcommand{\tt}{\mathbf{t}}

\usepackage[titles]{tocloft}
\setlength{\cftbeforechapskip}{0pt}

\begin{document}
\maketitle
\tableofcontents

\section{Introduction}

The main idea of \textbf{matroid} is to \emph{capture the abstract/combinatorial essence of independence}. Matroid originated from linear algebra and graph theory. It is a concept first introduced by Hassler Whitney in 1935, also independently discovered by Takeo Nakasaw. But in the past decade, there have been quite some breakthroughs regarding matroids. Indeed, one recent Fields medalist June Huh also has works based on matroids.

In today's lecture, we will first define independence in two important contexts: linear algebra and graph theory. Then, we will define matroids from three different viewpoints: independence sets, circuits, and bases. After that, we will look into a concept that matroids handle exceptionally well: duality. And we will conclude the lecture with some graph theories from the matroid point of view.

\section{Independency}

\nbf{Linear Algebra}

The most common concept of independence is in linear algebra.

\begin{definition}[def:lin_dep]{Linearly Dependency}
  A sequence of vectors $\vect{v}_1,\vect{v}_2,\dots ,\vect{v}_k$ from a vector space $V=\mathbb{R}^n$ is said to be \vocab{linearly dependent}, if there exist scalars $a_1,a_2,\dots ,a_k$, not all zero, such that
  \[
    a_1\vect{v}_1+a_2\vect{v}_2+\cdots +a_k\vect{v}_k=\vect{0}.
  \]
  A sequence of vectors $\flds{\vect{v}}{,}{k}$ is said to be linearly independent otherwise, i.e. $\sum_{i=1}^{k}a_i\vect{v}_k=0\iff a_i=0\ \forall\ i$.
  \begin{remark}
    When $\flds{\vect{v}}{,}{k}$ is linearly dependent, i.e. at least one of the scalars $a_i$ is nonzero, say $a_1\neq 0$, then the above equation can be rewritten as
    \[
      \vect{v}_1=-\frac{a_2}{a_1}-\cdots -\frac{a_k}{a_1}\vect{v}_k.
    \]
    Thus, a set of vectors is linearly dependent iff one of them is zero or a linear combination of the others.
  \end{remark}
\end{definition}

A good way to think of linear dependency is to use the concept of linear span.

\begin{definition}[def:]{Linear Span}
  The span of a finite set of vectors $E=\left\{\flds{\vect{v}}{,}{k}\right\}$ is defined as the set of all linear combinations of elements of $E$, i.e.
  \[
    \Span(E)=\left\{\sum_{i=1}^{k}a_i\vect{v}_i\mid \vect{v}_i\in E,a_i\in \mathbb{R}\right\}.
  \]
  The set $\Span(E)$ is a vector subspace of $\mathbb{R}^n$, i.e. it is closed under addition and multiplication by scalars.
\end{definition}

\begin{definition}[def:]{Basis}
  A basis $B$ of a vector space $V$ is a linearly independent subset of $V$ that spans $V$. 
  \begin{remark}
    A basis can also be treated as a maximal linearly independent set, as any additional vectors will be a linear combination of the maximal linearly independent set.
  \end{remark}
\end{definition}

\begin{definition}[def:]{Rank}
  The $\rank(V)$ of a vector space $V$ is the cardinality $|B|$ of a basis $B$ of $V$.
  \begin{remark}
    Rank can be thought of as the "dimension" of the vector space. It is a good exercise to prove that all bases have the same cardinality.
  \end{remark}
  \begin{remark}
    By the definition of rank, we can see that $|E|=\rank(\Span(E))$ iff $E$ is linearly independent.
  \end{remark}
\end{definition}

\begin{example}[exp:matroid_lin_indep]{Linearly Independency}
  \Cref{fig:exp_matroids} will be our main example for introducing matroids. We will start with our linear algebra case \Cref{fig:exp_linear_algebra}.
  Since $f$ is the zero vector, by the remark of \cref{def:lin_dep}, $f$ will never be part of a linearly independent set. And since $e=2d$, $\left\{d,e\right\}$ is linearly dependent. Similarly, we have $b+c/2-d=0$, hence $\left\{b,c,d\right\}$ is linearly dependent.

  Now, with the linear span point of view, since $\Span(\{d,e\})$ is one "dimension", they are dependent. And since $\Span(\{b,c,d\})$ is the two "dimension" $xy$-plane, they are dependent as well. Note that the vector $a$ is the only vector pointed outside of the $xy$-plane, we see that any maximal independent set contains $a$ (as adding vector $a$ to the set will always increase the "dimension" of the linear span).
\end{example}

\begin{figure}[H]
  \centering
  \begin{subfigure}{0.33\linewidth}
    \centering
    \begin{asy}
      import MOgeom;
      size(0,1.2inch);
      pointfontsize=7;
      pair B=(1,0), C=(-1,-1), D=(.5,-.5), E=(1,-1), A=(0,1);
      defaultarrow=EndArrow;
      D(o--D("a(0,0,1)",3*A));
      D(o--D("b(1,0,0)",3*B));
      D(o--D("c(0,2,0)",3*C));
      D(o--D("d(1,1,0)",3*D,(1,0)));
      D(o--D("e(2,2,0)",3*E));
      D("f(0,0,0)",o,W);
    \end{asy}
    \caption{Linear Algebra}
    \label{fig:exp_linear_algebra}
  \end{subfigure}
  \begin{subfigure}{0.32\textwidth}
    \centering
    \begin{asy}
      import MOgeom;
      size(0,1inch);
      pair C[] = CircleOfUnity(3), D=(2,0);
      D("1",C[0],S);
      D("2",C[1]);
      D("3",C[2]);
      D("4",D,S);
      D(C[0]--C[1]);
      D(C[0]--C[-1]);
      D(C[1]..(C[1].x-0.3,0)..C[-1]);
      D(C[1]..(C[1].x+0.3,0)..C[-1]);
      label("$d$",(C[1].x+.3,0),E);
      label("$e$",(C[1].x-.3,0),W);
      label("$b$",(C[0]+C[1])/2,N);
      label("$c$",(C[0]+C[-1])/2,S);
      label("$a$",(1.5,0),N);
      D(C[0]--D);
      real dis=.8;
      D(D..(D.x+dis/2,-dis/4)..(D.x+dis,0)..(D.x+dis/2,+dis/4)..D);
      label("$f$",(D.x+dis,0),E);
    \end{asy}
    \caption{Graph}
    \label{fig:exp_graph}
  \end{subfigure}
  \begin{subfigure}{0.32\textwidth}
    \[
      \begin{blockarray}{ccccccc}
        &a & b & c & d & e&f\\
        \begin{block}{c[cccccc]}
          1 & 1 & -1 & 1 & 0 & 0 & 0 \\
          2 & 0 & 1 & 0 & -1 & 1 & 0 \\
          3 & 0 & 0 & -1 & 1 & -1 & 0 \\
          4 & -1 & 0 & 0 & 0 & 0 & 0 \\
        \end{block}
      \end{blockarray}
    \]
    \vspace{-2em}
    \caption{Representation}
    \label{fig:exp_representation}
  \end{subfigure}
  \caption{Example: Matroids}
  \label{fig:exp_matroids}
\end{figure}

In this note, we use "maximal" and "minimal" with respect to inclusion. It is not hard to have the following observation about linear dependency:

\begin{lemma}[lem:ind_subset]{}
  If a set of vectors $E$ of the vector space $\mathbb{R}^n$ is linearly independent, then any subset of $E$ is linearly independent. If $E$ is linearly dependent, then $E\cup F$ is linearly dependent for any set of vectors $F$ of $\mathbb{R}^n$.
\end{lemma}

Hence to record the collection of all linearly independent sets, it is enough to record the maximal linearly independent sets. Similarly, to record the collection of all linearly dependent sets, it is enough to record the minimal dependent sets. For our example \Cref{fig:exp_linear_algebra}, we have the following collections.

\begin{enumerate}
  \item Linear independent set: $\left\{a,b,c,d,e,ab,ac,ad,ae,bc,bd,be,cd,ce,abd,abe,abc,acd,ace\right\}$.
  \item Maximal independent set: $\{abd,abe,abc,acd,ace\}$.
  \item Minimal dependent set: $\left\{f,de,bcd,bce\right\}$
\end{enumerate}

\nbf{Graph}

For linear algebra, a combinatorial property of a set of vectors is their linear independence. For graphs, the "corresponding" combinatorial property of a set of edges is if they contain a cycle. Hence the "base set" in the graph case will be the set of edges (w.r.t. the set of vectors in linear algebra).

For the pure combinatorial purpose, we assume our graphs to be undirected. And for a graph $G$, we denote $E(G)$ and $V(G)$ to be the set of edges and vertices of $G$ respectively.

\begin{definition}[def:]{Cycle}
  A cycle is a non-empty trail in which only the first and last vertices are equal.
\end{definition}

\begin{definition}[def:]{Independency -- Graph}
  A set of edges is called independent if it does not contain a cycle.
  \begin{remark}
    In the case where the graph $G$ is connected, a maximal independent would hence be a spanning tree.
  \end{remark}
\end{definition}

With this notion of independency, for the graph \Cref{fig:exp_graph}, we see that all the observations in \cref{exp:matroid_lin_indep} can be applied also. Indeed, one can check that \Cref{fig:exp_linear_algebra} and \Cref{fig:exp_graph} shares the exact collection of independent sets, maximal independent sets, and minimal dependent sets.

\nbf{Combinatorial Essence of Independence}

If we are only interested in the concept of dependency, a set of vectors or a graph obviously contain too much information. In the linear algebra case, it is obvious that the length of the vector will not affect the dependency. Moreover, the directions of the vectors do not matter also to some extent. For example, in our example \Cref{fig:exp_linear_algebra}, tilting the vector $a$ from $(0,0,1)$ to $(1,1,1)$ will not affect the dependency. 

In the graph case, how the edges are connected exactly will not affect the dependency as well. For example, which vertex is the loop $f$ placed on will not affect the dependency. It is not hard to see that rotating some other edges will keep the dependency as well.

Indeed, here we have two different mathematical objects, but they share the same dependency. It makes us wonder if we can extract the dependency information from them.

\section{Definition of Matroids}

There are several ways (perspectives) to define a matroid, namely independent sets, circuits, bases, flats, rank functions, and so on. They treat different points of view on the same concept of independence. We will introduce the first three perspectives today.

\nbf{Independent Sets}

\begin{definition}[def:matroid_indep]{Matroid (Independence)}
  A \vocab{matroid} $M$ is an ordered pair $(E,\II)$ consisting of a finite set $E$ and a collection $\II\subseteq 2^E$ of subsets of $E$ having the following properties:
  \begin{enumerate}
    \item [I1.] $\emptyset \in \II$.
    \item [I2.] If $I_1\in \II$ and $I_2\subseteq I_1$ then $I_1\in \II$, i.e., $\II$ is closed under taking subsets.
      %$\ceil{\II}=\II$, i.e., $\II$ is closed under taking subsets.
    \item [I3.] (Independent set exchange property) If $I_1$ and $I_2$ are in $\II$ and $|I_1|>|I_2|$, then there is an element $e$ of $I_1-I_2$ such that $I_2\cup e\in \II$.
  \end{enumerate}
  The members of $\II$ are the \vocab{independent sets} of $M$, and $E$ is the group set of $M$. A subset of $E$ that is not in $\II$ is called \textbf{dependent}. We often write $\II(M)$ for $\II$ and $E(M)$ for $E$.
\end{definition}

%We are going to show that both linear algebra and graphs form matroids.
The name "matroid" was coined by Whitney (1935) because a class of fundamental examples of such objects arises from matrices. Indeed, we can rephrase our linear algebra example as follows.

\begin{theorem}[thm:matrix_matroid]{}
  Let $E$ be the set of column labels of an $m\times n$ matrix $A$, and let $\II$ be the set of subsets $X$ of $E$ for which the multiset of columns labeled by $X$ is linearly independent. Then $(E,\II)$ is a matroid.
  \begin{proof}
    $I1$ and $I2$ follows from \cref{lem:ind_subset}. For $I3$, consider $W=\Span(I_1\cup I_2)$, we have $\rank(W)\geq |I_2|$. If $I_1\cup e$ is dependent for all $e\in I_2\setminus I_1$, then $\Span(I_1)=W$. But $\rank(\Span(I_1))=|I_1|<|I_2|\leq \rank(W)$, contradiction.
  \end{proof}
\end{theorem}

\nbf{Circuits}

\begin{definition}[def:]{Circuit}
  A minimal dependent set in a matroid $M$ will be called a \vocab{circuit}. We will denote the collection of circuits of $M$ by $\CC$ or $\CC(M)$.
\end{definition}

From \cref{def:matroid_indep}, to define a matroid (to specify $\II$), we can instead specify the complement $2^E\setminus \II$, or the collection of dependent sets. In view of \cref{lem:ind_subset}, the collection of all dependent sets can be specified by only the collection of minimal dependent sets. Hence, once $\II(M)$ has been specified, $\CC(M)$ can be determined. Similarly, $\II(M)$ can be determined from $\CC(M)$: the members of $\II(M)$ are those subsets of $E(M)$ that contain no member of $\CC(M)$.
Thus a matroid is uniquely determined by its collection $\CC$ of circuits.

By the "minimal" property of circuits, it is obvious that $\CC$ satisfies:
\begin{enumerate}
  \item [C1.] $\emptyset \not\in \CC$.
  \item [C2.] If $C_1, C_2\in \CC$ and $C_1\subseteq C_2$, then $C_1=C_2$.
\end{enumerate}

Furthermore:
\begin{lemma}[lem:]{Circuit Elimination Axiom}
  The set $\CC$ of circuits of a matroid satisfies:
  \begin{enumerate}
    \item [C3.] If $C_1,C_2\in \CC$ with $C_1\neq C_2$ and $\ee\in C_1\cap C_2$, then exist $C_3\in \CC$ such that $C_3\subseteq (C_1\cup C_2)-\ee$.
  \end{enumerate}
  \begin{proof}
    Assume $C_1\cup C_2\setminus \ee$ is in $\II$. By $C2$, the set $C_2-C_1$ is non-empty, so we can choose an element $f$ from this set. As $C_2$ is a minimal dependent set, $C_2-f\in \II$. Now choose a subset $I$ of $C_1\cup C_2$ which is maximal with the properties that it contains $C_2-f$ and is independent. Evidently $f\notin I$. Moreover, as $C_1$ is a circuit, some element $g$ of $C_1$ is not in $I$. As $f\in C_2-C_1$, the elements $f$ and $g$ are distinct. Hence 
    \[
      |I|\leq |(C_1\cup C_2)-\left\{f,g\right\}|=|C_1\cup C_2|-2<|(C_1\cup C_2)-e|.
    \]
    Now Apply $I3$, taking $I_1$ and $I_2$ to be $I$ and $(C_1\cup C_2)-e$, respectively. The resulting independent set contradicts the maximality of $I$.
  \end{proof}
\end{lemma}

And turns out this $C3$ is enough for us to characterize the collections of circuits:

\begin{theorem}[thm:]{}
  Let $E$ be a set and $\CC$ be a collection of subsets of $E$ satisfying $C1$--$C3$. Let $\II$ be the collection of subsets of $E$ that contain no member of $\CC$. Then $(E,\II)$ is a matroid having $\CC$ as its collection of circuits.
  \begin{proof}
    I1: By $C1$, $\emptyset\notin \CC$, so $\emptyset \in \II$.
    I2: If $I$ contains no member of $\CC$ and $I'\subseteq I$, then $I'$ contains no member of $\CC$. With some tedious work of chasing the definition, we can prove I3, which we will omit here.
  \end{proof}
\end{theorem}

\begin{proposition}[pps:]{}
  Suppose that $I$ is an independent set in a matroid $M$ and $e$ is an element of $M$ such that $I\cup e$ is dependent. Then $M$ has a unique circuit contained in $I\cup e$, and this circuit contains $e$.
  \begin{proof}
    Clearly, $I\cup e$ contains a circuit, and all such circuits must contain $e$. Let $C$ and $C'$ be distinct such circuits, then $(C\cup C')-e$ contains a circuit. But $(C\cup C')-e\subseteq I$, contradiction.
  \end{proof}
\end{proposition}

\begin{proposition}[pps:]{Cycle Matroid}
  Let $E$ be the set of edges of a graph $G$ and $\CC$ be the set of edge sets of cycles of $G$. Then $\CC$ is the set of circuits of a matroid on $E$.
  \begin{remark}
    We call the matroid derived above from the graph $G$ the \vocab{cycle matroid} of $G$. 
  \end{remark}
  \begin{proof}
    Clearly $\CC$ satisfies $C1$ and $C2$. For $C3$, let $C_1$ and $C_2$ be the edge sets of two distinct cycles of $G$ that have $e$ as a common edge. Let $u$ and $v$ be the endpoints of $e$. We now construct a cycle of $G$ whose edge set is contained in $(C_1\cup C_2)-e$. For each $i$ in $\left\{1,2\right\}$, let $P_i$ be the path from $u$ to $v$ in $G$ whose edge set is $C_i-e$. Beginning at $u$, traverse $P_1$ towards $v$ letting $w$ be the first vertex at which the next edge of $P_1$ is not in $P_2$. Continue traversing $P_1$ from $w$ towards $v$ until the first time a vertex $x$ is reached that is distinct from $w$ but is also in $P_2$. Since $P_1$ and $P_2$ both end at $v$, such a vertex must exist. Now adjoin the section of $P_1$ from $w$ to $x$ to the section of $P_2$ from $x$ to $w$. The result is a cycle, the edge set of which is contained in $(C_1\cup C_2)-e$. 
  \end{proof}
\end{proposition}

In the sense of matroid, the connectedness of a graph does not affect its matroidal structure to some extent:

\begin{proposition}[pps:]{}
  Let $M$ be a graphic matroid. Then $M\simeq M(G)$ for some connected graph $G$.
  \begin{proof}
    As $M$ is graphic, $M\simeq M(H)$ for some graph $H$. If $H$ is connected, the result is proved. If not, suppose $\flds{H}{,}{n}$ are the connected components of $H$. For each $i$ in $\left\{1,\dots ,n\right\}$, choose a vertex $v_i\in H_i$. Form a new graph $G$ by identifying $\flds{v}{,}{n}$ as a single vertex. Clearly $E(H)=E(G)$ and $G$ is connected. The proof follows by observing that if $X\subseteq E(H)$, then $X$ is the set of edges of a cycle in $H$ iff $X$ is the set of edges of a cycle in $G$. 
  \end{proof}
\end{proposition}

\newpage
\nbf{Bases}

\begin{definition}[def:]{Basis or Base}
  We call a maximal independent in a matroid $M$ a \vocab{bases} or a \vocab{base}.
\end{definition}

\begin{lemma}[lem:]{}
  If $B_1$ and $B_2$ are bases of a matroid $M$, then $|B_1|=|B_2|$.
\end{lemma}

If $M$ is a matroid and $\BB$ is its collection of bases, then by $I1$, we have $\BB$ satisfies:
\begin{enumerate}
  \item [B1.] $\BB$ is non-empty.
\end{enumerate}
And corresponding to $I3$ and $C3$, we have the following property of $\BB$:

\begin{lemma}[lem:]{Basis Exchange Property}
  The set $\BB$ of bases of a matroid satisfies:
  \begin{enumerate}
    \item [B2.] If $B_1$ and $B_2$ are members of $\BB$ and $x\in B_1-B_2$, then there is an element $y\in B_2-B_1$ s.t. $(B_1-x)\cup y\in \BB$.
  \end{enumerate}
\end{lemma}

And again, we can show that $B1$ and $B2$ characterize the bases of a matroid:
\begin{theorem}[thm:]{}
  Let $E$ be a set and $\BB\subseteq 2^E$ satisfying $B1$ and $B2$. Let $\II$ be the collection of subsets of $E$ that are contained in some member of $\BB$. Then $(E,\II)$ is a matroid having $\BB$ as its collection of bases.
\end{theorem}

\begin{corollary}[crl:fund_cir_of_basis]{Fundamental Circuit of $e$ w.r.t. $B$}
  Let $B$ be a basis of a matroid $M$. If $e\in E(M)-B$, then $B\cup e$ contains a unique circuit, $C(e,B)$. Moreover, $e\in C(e,B)$.
\end{corollary}

\section{Duality}

The concept of duality appears in basically any mathematics object: graph theory, vector space, and so on. Certainly, matroids also have duality, which serves as one of their most attractive features.

\begin{theorem}[thm:]{Dual Matroid}
  Let $M$ be a matroid and $\BB^*(M)$ be $\left\{E(M)-B:B\in \BB(M)\right\}$. Then $\BB^*(M)$ is the set of bases of a matroid on $E(M)$. 
  \begin{remark}
    This matroid with ground set $E(M)$ and bases $\BB^*(M)$ is called the \vocab{dual} of $M$ and is denoted by $M^*$. Hence $\BB(M^*)=\BB^*(M)$.
  \end{remark}
  \begin{lemma}[lem:]{}
    The set $\BB$ of bases of a matroid $M$ has the following property:
    \begin{enumerate}
      \item [B2$^*$.] If $B_1,B_2\in \BB$ and $x\in B_2-B_1$, then there is an element $y$ of $B_1-B_2$ s.t. $(B_1-y)\cup x\in \BB$.
    \end{enumerate}
    \begin{proof}
      By \cref{crl:fund_cir_of_basis}, $B_1\cup x$ contain a unique circuit $C(x,B_1)$. As $C(x,B_1)$ is dependent and $B_2$, is independent, $C(x,B_1)-B_2$ is non-empty. Let $y$ be an element of this set. Evidently, $y\in B_1-B_2$. Moreover, $(B_1-y)\cup x$ is independent since it does not contain $C(x,B_1)$. As $|(B_1-y)\cup x|=|B_1|$, it follows that $(B_1-y)\cup x$ is a basis.
    \end{proof}
  \end{lemma}
  \begin{proof}
    As $\BB(M)$ is non-empty, so is $\BB^*(M)$. Hence $\BB^*(M)$ satisfies $B1$. Now suppose $B_1^*$ and $B_2^*$ are in $\BB^*(M)$ and $x\in B_1^*-B_2^*$. Let $E:=E(M)$, and let $B_i=E-B_i^*$ for each $i=1,2$. Then $B_i\in \BB(M)$ and $B_1^*-B_2^*=B_1^*\cap (E-B_2^*)=(E-B_1)\cap B_2=B_2-B_1$. By $B2^*$, as $x\in B_2-B_1$, there is an element $y$ of $B_1-B_2$ s.t. $(B_1-y)\cup x\in \BB(M)$. Clearly $y\in B_2^*-B_1^*$ and $E-((B_1-y)\cup x)\in \BB^*(M)$. But $E-((B_1-y)\cup x)=((E-B_1)-x)\cup y=(B_1^*-x)\cup y$. Thus $\BB^*(M)$ satisfies $B2$, and $\BB^*(M)$ is indeed the set of bases of a matroid on $E$.
  \end{proof}
\end{theorem}

The bases of $M^*$ are called cobases of $M$. A similar convention applies to other distinguished subsets of $E(M^*)$. For example, the circuits and independent sets of $M^*$ are called cocircuits and coindependent sets. 

\begin{proposition}[pps:intersection_circuit_cocircuit]{}
  In a matroid $M$, let $C$ be a circuit and $C^*$ be a cocircuit. Then $|C\cap C^*|\neq 1$.
  \begin{proof}
    We first obvserve the following equivalence for $X\subseteq E(M)$:
    \begin{enumerate}
      \item $X$ is a non-spanning set of $M$ but $X\cup y$ is spanning for all $y\notin X$ (or equivalently $X$ is of rank $r(M)-1$).
      \item $E-X$ is dependent in $M^*$ but $(E-X)-y$ is independent in $M^*$ for all $y$ in $E-X$ (or equivalently $E-X$ is a cocircuit of $M$).
    \end{enumerate}
    Assume contrary $C\cap C^*=\{x\}$. Since $C$ is a circuit, $C-x$ is independent. We can then extend it to maximal independent set $B$ of $M-C^*$, which by (1) is of rank $r(M)-1$, and $B+y$ for any $y\in C^*$ is a basis of $M$. In particular, $B+x$ is a basis. But $C\subseteq B+x$, contradiction.
  \end{proof}
\end{proposition}

For the particular type of matroids we are introducing, cycle matroid, we give a specific name to its dual:

\begin{definition}[def:]{Bond/Cocycle Matroid}
  For a graph $G$, we denote the dual of the cycle matroid of $G$ by $M^*(G)$. We call this matroid \vocab{bond matroid} or \vocab{cocycle matroid}.
\end{definition}

We will see later that the duality of a graph matroid has a deep connection with the duality of graphs. The graphs we will focus on are a specific type:

\begin{definition}[def:]{Planar Embedding}
  A \vocab{planar embedding} of a graph $G$ is a mapping from every vertice to a point on a plane, and from every edge to a plane curve on that plane, such that the extreme points of each curve are the points mapped from its end nodes, and all curves are disjoint except on their extreme points. 
\end{definition}

\begin{definition}[def:]{Plane Graph}
  A \vocab{plane graph} is a graph $G$ together with a planar embedding of $G$. We call a graph \textbf{planar} if it has a planar embedding.
\end{definition}

\begin{example}[exp:]{}
  Not every graph is planar, and some graph may seem non-planar but indeed is.
  \begin{figure}[H]
    \centering
    \begin{subfigure}{0.45\textwidth}
      \centering
      \begin{subfigure}{0.45\textwidth}
        \centering
        \begin{asy}
          import MOgeom;
          size(0,.9inch);
          pair A=(-1,-1), B=(-1,1), C=(1,1), D=(1,-1); 
          D(D(A)--D(B)--D(C)--D(D)--cycle);
          D(A--C);
          D(B--D);
        \end{asy}
      \end{subfigure}
      \textrightarrow
      \begin{subfigure}{0.45\textwidth}
        \centering
        \begin{asy}
          import MOgeom;
          size(0,.9inch);
          pair A=(-1,-1), B=(-1,1), C=(1,1), D=(1,-1); 
          D(D(A)--D(B)--D(C)--D(D)--cycle);
          D(A..(-1.2,1.2)..C);
          D(B--D);
        \end{asy}
      \end{subfigure}
    \caption{Planar Graphs: $K_4$}
    \end{subfigure}
    \begin{subfigure}{0.45\textwidth}
      \begin{subfigure}{0.45\textwidth}
        \centering
        \begin{asy}
          import MOgeom;
          size(0,.9inch);
          pair C[] = CircleOfUnity(5)*dir(18);
          DCP(C,ar=None);
          D(C[0]--C[2]);
          D(C[1]--C[3]);
          D(C[2]--C[4]);
          D(C[3]--C[0]);
          D(C[4]--C[1]);
        \end{asy}
      \end{subfigure}
      \textrightarrow
      \begin{subfigure}{0.45\textwidth}
        \centering
        \begin{asy}
          import MOgeom;
          size(0,.9inch);
          pair C[] = CircleOfUnity(5)*dir(18);
          DCP(C,ar=None);
          D(C[0]..(0,1.2)..C[2]);
          D(C[1]--C[3]);
          D(C[2]..(0,-1.2)..C[4]);
          D(C[3]--C[0]);
          D(C[4]--C[1]);
        \end{asy}
      \end{subfigure}
      \caption{Non-Planar Graphs: $K_5$}
    \end{subfigure}
  \end{figure}
\end{example}

\begin{definition}[def:]{Dual Graph}
  For a graph $G$ with at least one edge, the construction of $G^*$, the \vocab{dual} of $G$, is formally described as follows: Choose a single point $v_F$ in each face of $F$ of $G$. These points are to be the vertices of $G^*$. Suppose that the set of edges common to the boundaries of two faces $F$ and $F'$ is $\left\{\flds{e}{,}{k}\right\}$. Then we join $v_F$ and $v_{F'}$ by $k$ edges $\flds{e'}{,}{k}$, where $e_i'$ crosses $e_i$ but no other edge of $G$, and each $e_i'$ is a simple curve. The only points common to two distinct edges of $G^*$ can be their endpoints. If the edge $e$ of $G$ lies on the boundary of a single face $F$, we add a loop $e'$ at $v_F$ crossing $e$ but no other edge of $G$ or $G^*$.
  \begin{remark}
    It is not difficult to show that $G^*$ is connected. And if $G$ is connected, $(G^*)^*=G$.
  \end{remark}
\end{definition}

\begin{example}[exp:]{Dual Graphs}
  Here we provide some examples of dual graphs. Note that a dual graph depends on a planer embedding. In \cref{fig:diff_embedding}, we can see that $G_1$ and $G_2$ represent the same graph, but $G_1^*$ and $G_2*$ are not isomorphic. Yet, we can prove that $G_1^*$ and $G_2^*$ have isomorphic cycle matroids. 
  \begin{figure}[H]
  \centering
  \begin{subfigure}{0.35\textwidth}
    \centering
    \begin{asy}
      import MOgeom;
      size(0,.9inch);
      pair C[] = CircleOfUnity(3), D=(2.5,0);
      pair M[];
      D(C[0]);
      D(C[1]);
      D(C[2]);
      D(D);
      D(C[0]--C[1]);
      D(C[0]--C[-1]);
      D(C[1]..(C[1].x-0.3,0)..C[-1]);
      D(C[1]..(C[1].x+0.3,0)..C[-1]);
      D(C[0]--D);
      real dis=1;
      D(D..(D.x+dis/2,-dis/4)..(D.x+dis,0)..(D.x+dis/2,+dis/4)..D);
      M[0]=(C[1].x,0);
      M[1]=o+0.3;
      M[2]=((C[0].x+D.x)/2,1);
      M[3]=D+dis/2;
      pointpen=green;
      pathpen=grey+dashed;
      for (int i:sequence(4)){D(M[i]);}
      D(M[0]--M[1]--M[2]--M[3]);
      D(M[0]..(-1.5,2)..(0.2,1.5)..M[2]);
      D(M[1]..(0.4,-0.1)..(5,0)..(3.5,1.3)..M[2]);
      D(M[2]..(3,.8)..(2*M[3]-M[2])*.8..((C[0].x+D.x)/2,0)..M[2]);
    \end{asy}
    \caption{Example of Dual Graphs}
  \end{subfigure}
  \begin{subfigure}{0.6\textwidth}
    \centering
    $G_1$
    \begin{asy}
      import MOgeom;
      size(0,.9inch);
      pair A=(-1,-1), B=(-1,1), C=(1,1), D=(1,-1), E=(3,0), F=(-3,0); 
      D(D(A)--D(B)--D(C)--D(D)--cycle);
      D(A--D(F)--B);
      D(C--D(E)--D);
      pointpen=green;
      pathpen=grey+dashed;
      pair O=D((0,0)), N=D((-2,0)), P=D((2,0)), T=D((0,2));
      D(T--O--P);
      D(O--N);
      D(T..(1.3,1.3)..P);
      D(T..(3,0.8)..(3.5,0)..(3,-0.5)..P);
      D(T..(-1.3,1.3)..N);
      D(T..(-3,0.8)..(-3.5,0)..(-3,-0.5)..N);
      D(T..(3.7,0.8)..(3.7,-0.8)..(1,-1.5)..O);
    \end{asy}
    \qquad \quad 
    $G_1^*$
    \begin{asy}
      import MOgeom;
      size(0,.9inch);
      pointpen=green;
      pathpen=grey;
      pair C[] = CircleOfUnity(4);
      D(D(C[1])--D(C[3])--D(C[2]));
      D(C[3]--D(C[0]));
      D(C[1]..(0.9*dir(135))..C[2]);
      D(C[1]..(0.6*dir(135))..C[2]);
      D(C[1]..(0.9*dir(45))..C[0]);
      D(C[1]..(0.5*dir(45))..C[0]);
      D(C[1]..(1.5,0)..C[3]);
    \end{asy}
    \vspace{0.4em}

    $G_2$
    \begin{asy}
      import MOgeom;
      size(0,.9inch);
      pair A=(-1,-1), B=(-1,1), C=(3,1), D=(3,-1), E=(1.2,0), F=(-3,0); 
      D(D(A)--D(B)--D(C)--D(D)--cycle);
      D(A--D(F)--B);
      D(C--D(E)--D);
      pointpen=green;
      pathpen=grey+dashed;
      pair O=D((0,0)), N=D((-2,0)), P=D((2.3,0)), T=D((0,2));
      D(T--O--N);
      D(O..(1.2,-0.5)..P);
      D(O..(1.2,+0.5)..P);
      D(T..(3,1.5)..(3.5,0.5)..P);
      D(T..(B+(-0.3,0.3))..N);
      D(T..(-3,0.8)..(-3.5,0)..(-3,-0.5)..N);
      D(T..(3.7,0.8)..(3.7,-0.8)..(1,-1.5)..O);
    \end{asy}
    \qquad \quad 
    $G_2^*$
    \begin{asy}
      import MOgeom;
      size(0,.9inch);
      pointpen=green;
      pathpen=grey;
      pair C[] = CircleOfUnity(4);
      D(D(C[1])--D(C[3])--D(C[2]));
      D(C[1]--D(C[0]));
      D(C[1]..(0.9*dir(135))..C[2]);
      D(C[1]..(0.6*dir(135))..C[2]);
      D(C[3]..(0.9*dir(-45))..C[0]);
      D(C[3]..(0.5*dir(-45))..C[0]);
      D(C[1]..(1.5,0)..C[3]);
    \end{asy}
    \caption{Different embeddings giving different geometric duals}
    \label{fig:diff_embedding}
  \end{subfigure}
    \caption{Example}
\end{figure}
\end{example}

\begin{definition}[def:]{Edge Cut}
  For a set $X$ of edges in a graph $G$, we shall denote by $G\setminus X$ the subgraph of $G$ obtained by deleting all the edges in $X$. If $G\setminus X$ has more connected components than $G$, we call $X$ an \vocab{edge cut} of $G$.
\end{definition}

\begin{definition}[def:]{Bond/Cocycle}
  We call a minimum edge cut a \vocab{bond} or \vocab{cocycle} of $G$.
\end{definition}

It turns out that the circuits of the dual of a cycle matroid (bond matroid) of $G$ are exactly the bonds of $G$. Or more generally,

\begin{lemma}[lem:]{}
  If $G^*$ is a geometric dual of a connected planer graph $G$, then
  \[
    M(G^*)\simeq M^*(G).
  \]
  \vspace{-2em}
  \begin{proof}
    Let $G^*$ be the geometric dual of a planar embedding $G_0$ of $G$. The construction of $G^*$ from $G_0$ determines a bijection $\alpha$ from $E(G_0)$ to $E(G^*)$. We shall show that, under the map $\alpha$, $M(G_0)\simeq M^*(G^*)$. Since $M(G_0)=M(G)$, the required result will be the following.

    Let $C$ be a circuit in $M(G_0)$. We want to show that $\alpha(C)$ is a bond in $G^*$. Now $C$ forms a Jordan curve (a closed curve that does not self-intersect) in the plane, and each edge in $\alpha(C)$ has one endpoint inside and the other endpoint outside this closed curve. Thus $\alpha(C)$ is an edge cut in $G^*$. The fact that $\alpha(C)$ is a minimal edge cut will follow from what we shall show next, namely, that if $X$ is a minimal edge cut in $G^*$, then $\alpha^{-1}(X)$ contains a cycle of $G_0$. Evidently, on deleting the edges of $X$ from $G^*$, we obtain a graph having two components, $G_1^*$ and $G_2^*$. Moreover, every edge in $X$ has one endpoint in $G_1^*$ and the other is $G_2^*$. If $|X|=1$, then it follows from the construction of $G^*$ that the single edge in $\alpha^{-1}(X)$ is a loop of $G_0$. Now suppose that $|X|>1$ and let $F$ be a face of $G^*$ s.t. some edge, say $x$, of $X$, is in the set $F'$ of boundary edges of $F$. Then, as $x$ is not a cut-edge of $G^*$, it is not difficult to check that $x$ is not a cut-edge of $G^*[F']$. It is now an easy exercise in graph theory to show that $F'$ contains a circuit $C_x$ of $M(G^*)$ that contains $x$. 

    Since $X$ is a cocircuit of $M(G^*)$ and $x\in X\cap C_x$, it follows, by \cref{pps:intersection_circuit_cocircuit}, that $|X\cap C_x|\geq 2$. Hence if a face of $G^*$ meets an edge of $X$, it meets more than one such edge. Since $G_0$ is connected, $(G^*)^*=G_0$. Thus the faces of $G^*$ correspond to vertices of $G_0$. Therefore, every vertex of $G_0$ meeting an edge of $\alpha^{-1}(X)$ meets at least two such edges. Now a graph in which every vertex has a degree of at least two contains a cycle. Thus the induced graph $G_0[\alpha^{-1}(X)]$ contains a cycle of $G_0$. 
  \end{proof}
\end{lemma}

\begin{corollary}[crl:]{}
  TFAE for a subset $X$ of the set of edges of a graph $G$:
  \begin{enumerate}
    \item $X$ is a circuit of $M^*(G)$.
    \item $X$ is a cocircuit of $M(G)$.
    \item $X$ is a bond of $G$.
  \end{enumerate}
\end{corollary}

\begin{theorem}[thm:]{}
  Let $M$ be a matroid.
  \begin{enumerate}
    \item A set $C^*$ is a cocircuit of $M$ iff $C^*$ is a minimal set having a non-empty intersection with every basis of $M$.
    \item A set $B$ is a basis of $M$ iff $B$ is a minimal set having a non-empty intersection with every cocircuit of $M$.
  \end{enumerate}
\end{theorem}

This blocking property suggests the following two-person game. Given a matroid $M$ with ground set $E$, two players B and C alternately tag elements of $E$. The goal for B is to tag all the elements of some basis of $M$, while the goal for C is to prevent this. Equivalently, by the last result, C’s goal is to tag all the elements of some cocircuit of $M$. We shall specify when B can win against all possible strategies of C. If B has a winning strategy playing second, then it will certainly have a winning strategy playing first. This game has some interesting results.

\begin{theorem}[thm:]{}
  The following statements are equivalent to a matroid $M$ with ground set $E$.
  \begin{enumerate}
    \item Player C plays first and player B can win against all possible strategies of C.
    \item The matroid $M$ has 2 disjoint bases.
    \item For all subsets $X$ of $E$, $|X|\geq 2(r(M)-r(M\setminus X))$.
  \end{enumerate}
  \begin{remark}
    Edmonds also specified when player C has a winning strategy but this is more complicated and we omit it.
  \end{remark}
\end{theorem}

\section{The Greedy Algorithm}

\begin{definition}[def:]{Weight Function}
  Let $G$ be a connected graph and let $w$ be a function from $E(G)$ into $\mathbb{R}$. We call $w$ a weight function on $G$ and, for all $X\subseteq E(G)$, we define the \vocab{weight} $w(X)$ of $X$ to be $\sum_{x\in X}^{}w(x)$.
\end{definition}

For a given $G$ and $w$, we are interested in finding a spanning tree of $G$ of minimum weight. The solution to this question is by the greedy algorithm. This optimization problem can be extended to a matroid-like setting:

Let $\II\subseteq 2^E$ and suppose $\II$ satisfies $I1$ and $I2$. Let $w$ be a function from $E$ into $\mathbb{R}$. Similarly, define $w(X):=\sum_{x\in X}^{}w(x)$ for $X\subseteq E$. The optimzation problem for the pair $(\II,w)$ is as follows:

\begin{mdframed}
  \textbf{Optimazation Problem}. 
  \textit{Find a maximal member $B$ of $\II$ of maximum weight.}
\end{mdframed}

\begin{lemma}[lem:]{}
  If $(E,\II)$ is a matroid $M$, then $B_G$ constructed by the greedy algorithm define below is a solution to the optimization problem:
  \begin{enumerate}
    \item Set $X_0=\emptyset $ and $j=0$.
    \item If $E-X_j$ contains an element $e$ s.t. $X_j\cup e\in \II$, choose such an element $e_{j+1}$ of maximum weight, let $X_{j+1}=X_j\cup e_{j+1}$, and go to (3.); otherwise, let $B_G=X_j$ and go to (4.).
    \item Add $1$ to $j$ and go to (2.).
    \item Stop.
  \end{enumerate}
\end{lemma}

Indeed, greedy algorithms characterize matroids.

\begin{theorem}[thm:]{}
  Let $\II\subseteq 2^E$. Then $(E,\II)$ is a matroid iff $\II$ satisfies $I1$, $I2$, and
  \begin{enumerate}
    \item [G.] For all weight funcs $w:E\rightarrow \mathbb{R}$, the greedy algorithm produces a maximal member of $\II$ of max weight.
  \end{enumerate}
\end{theorem}

\section{Classification of Matroids}

Matroids not only capture the essence of independence from linear algebra and graph but also allow us to define new types of "independence" in the following sense: We call a matroid \vocab{representable} over field $\mathbb{F}$ if it is isomorphic to the matroid arise from a matrix $A$ with entries in $\mathbb{F}$ in the sense of \cref{thm:matrix_matroid}. And we call a matroid that is isomorphic to the cycle matroid of some graph \vocab{graphic}. Indeed, all graphic matroids are representable.

\begin{proposition}[pps:]{}
  If $G$ is a graph, then the cycle matroid of $G$ is representable over every field.
  \begin{proof}
    We can arbitrarily assign a direction to each edge of $G$. Then for each edge $e$ with head at vertex $i$ and tail at vertex $j$, we assign to each $e$ a column vector $\vect{v}_e$ with entry equals $1$ at the $i$-th entry; $-1$ at the $j$-th entry; and $0$ otherwise. Then we can check that such a set of column vectors form the same matroid.
  \end{proof}
  \begin{remark}
    An example would be our main example \cref{fig:exp_matroids}.
  \end{remark}
\end{proposition}

But almost all matroids are non-representable. Such a result is only proven very recently in 2010 by Peter Nelson. Some of the non-representable matroids can be drawn with a graphical representation.

\begin{definition}[def:]{Graphical Representation (rank 3 or below)}
  For a matroid $M$, if we can draw $E$ as the set of points with a set of lines (need not be straight) such that for $X\in \II$ we have $|X|\leq 3$ and $X$ does not contain $3$ collinear points.
\end{definition}

We have the following famous example of non-representable matroids:
\begin{figure}[H]
  \centering
  \begin{subfigure}{0.45\textwidth}
    \centering
    \begin{asy}
      import MOgeom;
      size(0,1inch);
      pair C[] = CircleOfUnity(3)*dir(-30);
      pair D[]={(0,C[0].y)};
      DCP(C,li=false);
      D(CR(o,-C[0].y));
      D[1]=D[0]*dir(120);
      D[2]=D[0]*dir(240);
      D.cyclic=true;
      for (int i:sequence(3)){D(D(string(i+1),C[i])--D(string(4+i),D[i+2]));}
      D("7",o);
    \end{asy}
    \caption{Fano Matroid (Only 2-representable)}
  \end{subfigure}
  \begin{subfigure}{0.45\textwidth}
    \centering
    \begin{asy}
      import MOgeom;
      size(0,1inch);
      pair A[] = {(-1,1),(.9,1),(2,1)};
      pair B[] = {(-1.1,-1),(.8,-1),(1.9,-1)};
      D(B[2]--A[0]--A[2]--B[0]--B[2]--A[1]--B[0]);
      D(A[0]--B[1]--A[2]);
      for (int i:sequence(3)){
        D(string(i+1),A[i]);
        D(string(i+4),B[i]);
      }
      D("7",IP(A[0]--B[1],A[1]--B[0]));
      D("8",IP(A[0]--B[2],A[2]--B[0]));
      D("9",IP(A[1]--B[2],A[2]--B[1]));
    \end{asy}
    \caption{Non-Pappus Matroid (Not representable)}
  \end{subfigure}
\end{figure}

\section*{Conclusion}

Hope that this note provides a new perspective to understand graphs, and hopefully raises your interest in matroids. For those who are interested, you may look into the survey "What is Matroids?" by Oxley, or his textbook "Matroid Theory".

%\section{Linearity of Expectation}

\begin{definition}[def:]{Random Variable}
    A \vocab{random variable} is a variable whose value is unknown or a function that assigns values to each of an experiment's outcomes.
\end{definition}

\begin{definition}[def:]{Expected Value}
    The \vocab{expected value} is the "weighted average" of a random variable $X$:
    \[\mathbb{E}[X]:=\sum_{x}^{}\mathbb{P}(X=x)\cdot x.\]
\end{definition}

\begin{example}[exp:]{}
    There are $n$ people, each of who has a name tag. We shuffle the name tags and randomly give each person one of the name tags. Let $S$ be the number of people who receive their own name tag. Prove that the expected value of $S$ is $1$. (i.e. $S$ independent of $n$)
\end{example}

\begin{proof}
    We can list all the $n!$ permutations and count the number of cases where the $i$-th person get his own name tag. We can easily see that there are $(n-1)!$ cases where the $i$-th person get his own tag. Hence we have
    $\mathbb{E}[S]=\frac{1}{n!}\left((n-1)!*n\right)=1$.
\end{proof}

\begin{theorem}[thm:]{Linearity of Expectation}
    Given any random variables $X_1,X_2,\dots ,X_n$, we always have
    \[\EE[X_1+X_2+\cdots +X_n]=\EE[X_1]+\EE[X_2]+\cdots +\EE[X_n].\]
\end{theorem}

\begin{remark}
    It is true even if $X_i$s are dependent of each other.
\end{remark}

\begin{proof}
    \begin{alignat*}{1}
        \EE[X+Y]&= \sum_{x}^{}\sum_{y}^{}[(x+y)\cdot P(X=x,Y=y)]\\
                %&= \sum_{x}^{}\sum_{y}^{}[x\cdot P(X=x,Y=y)]+\sum_{x}^{}\sum_{y}^{}[y\cdot P(X=x,Y=y)]\\
                &= \sum_{x}^{}x\sum_{y}^{}P(X=x,Y=y)+\sum_{y}^{}y\sum_{x}^{}P(X=x,Y=y)\\
                &= \sum_{x}^{}x\cdot P(X=x)+\sum_{y}^{}y\cdot P(y=y)\\
                &= \EE[X]+\EE[Y].
    \end{alignat*}
\end{proof}

\begin{example}[exp:]{}
    At a nursery, $2006$ babies sit in a circle. Suddenly, each baby randomly pokes either the baby to its left or to its right. What is the expected value of the number of unpoked babies?
\end{example}

\begin{proof}
    Number the babies $1,2,\dots ,2006$. Define
    \[X_i:=
        \begin{cases}
            1,&\quad \textnormal{if baby $i$ is unpoked}\\
            0,&\quad \textnormal{otherwise}.
        \end{cases}
    \]
    We seek $\EE[X_1+X_2+\cdots +X_{2006}]$. Note that any particular baby has probably $(\frac{1}{2})^2=\frac{1}{4}$ of being unpoked. Hence $\EE[X_i]=\frac{1}{4}$ for each $i$ and hence
    \[\EE[X_1+\cdots +X_2+\cdots +X_{2006}]=\sum_{i=1}^{2006}\EE[X_i]=2006\cdot \frac{1}{4}=\frac{1003}{2}.\]
\end{proof}

\begin{newenv}[rnd:]{Use Expected Value to Show Existence}{Usage}
    Suppose we know the average score of a test is 12.1, then there exists a student who got at least 13 points, and a contestant who got at most 12 points.    
\end{newenv}

\begin{remark}
    It is similar in spirit to the pigeonhole principle.
\end{remark}

\begin{example}[exp:]{}
    Prove that any subgraph of $K_{n,n}$ with at least $n^2-n+1$ edges has a perfect matching.
\end{example}

\begin{proof}
    We randomly pair off one set of points with the other (whether there is an actual edge or not), and define the score of such a pairing be the number of pairs which are actually connected by ane edge. Number the pairs by $1,2,\dots ,n$. Define
    \[X_i:=
        \begin{cases}
            1&\quad \textnormal{if the $i$th pair is connected by an edge}\\
            0&\quad \textnormal{otherwise}.
        \end{cases}
    \]
    Hence the score of a config $X=X_1+\cdots +X_n$. For any pair of points, they are connected with probability at least $\frac{n^2-n+1}{n^2}$. Hence $\EE[X_i]=\frac{n^2-n+1}{n^2}$, and hence
    \begin{alignat*}{1}
        \EE[X]&= \EE[X_1]+\cdots +\EE[X_n] = n\cdot \EE[X_1]\\
              &= \frac{n^2-n+1}{n}\\
              &= n-1+\frac{1}{n}.
    \end{alignat*}
    Since $X$ takes only integer values, there exists a config with $X=n$.
\end{proof}

\begin{theorem}[thm:]{Jenson's Inequality}
    If $f$ is a convex function and $t\in [0,1]$, we have
    \[f(tx_1+(1-t)x_2)\leq tf(x_1)+(1-t)f(x_2).\]
    In context of probability theory, if $X$ is a random variable and $\varphi $ is a convex function, then
    \[\varphi (\EE[X])\leq \EE[\varphi (X)].\]
\end{theorem}

%
%\newpage
%\begin{question}[]{}
%  \pitem[IMC 2002 day2 q2]{%
    200 students participated in a math contest. They had 6 problems to solve. Each problem was correctly solved by at least 120 participants. Prove that there must be 2 participants such that every problem was solved by at least one of these two students.
    }{%
    Pick two contestants randomly. Let $X_i$ be the indicator that both contestants miss problem $i$, so each $\EE[X_i]<\left(\frac{80}{200}\right)^2=4/25$, and their expected number of both-missed problems is $24/25<1$.
    %Pick two contestants randomly, the probability that those two can't so a problem is at most $\frac{\binom{80}{2}}{\binom{200}{2}}$. Hence, if $X$ is the number of problems that $2$ contestants can't solve then $\mathbb{E}[X]= 6 \cdot \frac{\binom{80}{2}}{\binom{200}{2}}<1.$ Hence, there exists a way to pick two contestants so that $\mathbb{E}[X]=0$, or each problem can be solved by at least one of them.
    }{%
    https://artofproblemsolving.com/community/c7h46086p291620
}

%  \pitem[Russia 1996]{%
    In the Duma there are 1600 delegates, who have formed 16000 committees of 80 persons each. Prove that one can find two committees having no fewer than four common members.
    }{%
    Consider a random couple of committees, and let $X$ be the number of delegates that join both of them. Consider $a_{d}$ the number of committees that include delegate ${d}$. As the average person is on $16000\cdot 80/1600=800$, by Jensen's inequality the expected value of $X$ is
    \[\mathbb E [X]=\sum_{d} \frac{\binom{a_{d}}{2}}{\binom{16000}{2}}\geq \frac{1600\binom{\frac{\sum a_{d}}{1600}}{2}}{\binom{16000}{2}}= \frac{1600\binom{\frac{16000 \cdot 80}{1600}}{2}}{\binom{16000}{2}} = \frac{1600\binom{800}{2}}{\binom{16000}{2}}>3,\]
    and the conclusion follows.
    }{%
    https://artofproblemsolving.com/community/c6h530232p3025510
}

%  %\pitem[NIMO 4.3]{%
    One day, a bishop and a knight were on squares in the same row of an infinite chessboard, when a huge meteor storm occurred, placing a meteor in each square on the chessboard independently and randomly with probability p. Neither the bishop nor the knight were hit, but their movement may have been obstructed by the meteors. For what value of p is the expected number of valid squares that the bishop can move to (in one move) equal to the expected number of squares that the knight can move to (in one move)?
    }{%
    $8(1-p) = 4 ((1 -p) + (1 -p)2 + (1 -p)3 + \cdots ).$
    }{%
    https://artofproblemsolving.com/community/c139h512825
}

%  %\pitem[NIMO 7.3]{%
    Richard has a four infinitely large piles of coins: a pile of pennies (worth 1 cent each), a pile of nickels (5 cents), a pile of dimes (10 cents), and a pile of quarters (25 cents). He chooses one pile at random and takes one coin from that pile. Richard then repeats this process until the sum of the values of the coins he has taken is an integer number of dollars. (One dollar is 100 cents.) What is the expected value of this final sum of money, in cents?
    }{%
    We let $E_n$ denote the expected value of the number of additional cents needed to get an integer, if Richard currently has $n$ cents modulo $100$, and he has not just started. Thus, by definition $E_0 = 0$, for example.

    More generally, for any $0 < k < 100$, we then have \[ E_k = \frac{(E_{k+1}+1) + (E_{k+5}+5) + (E_{k+10}+10) + (E_{k+25}+25)}{4} \]since if Richard picks up a penny, he gained one cent and expects to gain $E_{k+1}$ more, and similarly for the other three types of coins. Simplifying, \[ E_k = \frac{E_{k+1} + E_{k+5} + E_{k+10} + E_{k+25} + 41}{4}. \]
    Also, the actual answer to the problem is given by the value of \begin{align*} E_{\text{initial}} &= \frac{(E_1+1) + (E_5+5) + (E_{10}+10) + (E_{25}+25)}{4} \\ &= \frac{E_1 + E_5 + E_{10} + E_{25} + 41}{4} \end{align*}by the same reasoning.

    Adding these $99$ equations involving $E_k$ gives \[ \sum_{k \ge 1} E_k = \sum_{k \ge 1} E_k - \frac{E_1 + E_5 + E_{10} + E_{25}}{4} + \frac{41}{4} \cdot 99. \]Thus $E_1 + E_5 + E_{10} + E_{25} = 41 \cdot 99$. Consequently, $E_{\text{initial}} = 41 \cdot 25 = 1025$.
    }{%
    https://artofproblemsolving.com/community/c139h536001
}

%  %\pitem[NIMO 5.6]{%
    Tom has a scientific calculator. Unfortunately, all keys are broken except for one row: 1, 2, 3, + and -.
    Tom presses a sequence of $5$ random keystrokes; at each stroke, each key is equally likely to be pressed. The calculator then evaluates the entire expression, yielding a result of $E$. Find the expected value of $E$.
    (Note: Negative numbers are permitted, so 13-22 gives $E = -9$. Any excess operators are parsed as signs, so -2-+3 gives $E=-5$ and -+-31 gives $E = 31$. Trailing operators are discarded, so 2++-+ gives $E=2$. A string consisting only of operators, such as -++-+, gives $E=0$.)
    }{%
    Realize that the moment a sign is pressed, the expected value of the remaining numbers that are pressed after the sign will be equal to 0. This is most evidently seen in a few examples. For instance, the

    NNNSN configuration (where N represents "Number" and S represents "Sign") will have the Ns cancel because there will be an equal amount with S as + and S as -. So, we are only considering up to when the first Sign is pressed.

    We have the expected value is 2 of any one digit, since (1+2+3)/3 = 2.

    Now, we can say that the expected value of a 5-digit number is 22222, a 4 digit number is 2222, a 3 digit is 222, a 2 digit number is 22 and one number will be 2. All we have to do is find the probability that the number will be a 5 digit number, or a 4 digit number, and so on.

    So what is the chance we have a 5 digit number? This is just (3/5)^5.

    Similarily, the chance of a 4-digit number is $(3/5)^(4) * (2/5)$ as there must be a sign as the 5th digit.

    The 3-digit number chance is $(3/5)^(3)(2/5)(5/5)$ because the last digit can be anything, it is irrelevant.

    The last two are $(3/5)^(2)*(2/5)*(5/5)^(2)$ and $(3/5)(2/5)(5/5)^3$.

    Then we just multiply this all together. We get:

    $\frac{(22222)(3)^{5}+(3)^{4}(2)(2222)+(3)^{3}(2)(5)(222)+(3)^{2}(2)(5)^{2}(22)+(3)(2)(5)^{3}(2)}{(5)^{5}}$

    which turns out to be $\boxed{1866}$.

    The reason why we multiply each of the chances by the 22222 and 2222's and so on is because those are the average values of the n-digit numbers, and so by expected value formula we do (chance)*(value), where the value is just the average n-digit number value.
    }{%
    https://artofproblemsolving.com/community/c139h517886
}

%  \pitem[BAMO 2004]{%
    Consider $n$ real numbers, not all zero, with sum zero.
    Prove that one can label the numbers as $a_1,a_2,\dots ,a_n$ such that
    $a_1a_2+a_2a_3+\cdots +a_na_1<0$.
    }{%
    Show that a random permutations has expected value at most 0. By summing all the equations with $n!$ permutations. We can get $\sum-a_i^2$. Since they are not all $0$ we have expected value $<0$, hence the sum.
    \[\EE[a_ia_{i+1}]=\EE[a_i\EE[a_{i+1}\mid a_i]=\EE\left[a_i\left(\frac{-a_i}{n-1}\right)\right]=-\EE\left[\frac{a_i^2}{n-1}\right]<0.\]
    }{%
    https://web.evanchen.cc/handouts/ProbabilisticMethod/ProbabilisticMethod.pdf
    https://www.bamo.org/attachments/bamo2004examsol.pdf
}

%  %\pitem[Kürschák 2003, problem 3]{%
    Prove that the following inequality holds with the exception of finitely many positive integers $n$:

    \[\sum_{i=1}^n\sum_{j=1}^n gcd(i,j)>4n^2.\]
    }{%
    Let $g_p(x,y)$ for prime $p$ and $x,y\in \mathbb Z^+$ be $p$ if $p\mid \gcd(x,y)$ and $0$ otherwise. Since $\gcd(x,y)\geq \sum_{\text{prime }p\mid \gcd(x,y)}p$, it suffices to show that $\mathbb E\left[\sum_{\text{prime }p}g_p(x,y)\right]>4$. In fact, by LoE we have $\mathbb E\left[\sum_{\text{prime }p}g_p(x,y)\right]=\sum_{\text{prime }p}\mathbb E\left[g_p(x,y)\right]=\sum_{\text{prime }p}\frac{1}{p}>4\ \Box$.

    Or

    Here's a pretty quick but sloppy approach:

    \begin{align*} \sum_{i,j=1}^n \text{gcd}(i,j)&\ge \sum_{1\le i\le j\le n} \text{gcd}(i,j)=\sum_j \sum_{1\le i\le j}\text{gcd}(i,j) \\ &=\sum_{j=1}^n \sum_{d|j}d\cdot \varphi\left(\frac{j}{d}\right) \\ &=\sum_{ab\le n}a\cdot \varphi(b) =\sum_{b=1}^n \varphi(b) \sum_{1\le a\le \frac{n}{b}}a \\ &\ge \sum_{1\le b\le n/2} \varphi(b)\cdot \frac{(n/b)^2}{4}, \end{align*}from which it is clear that we are done if we can prove that the series $\sum_{b=1}^\infty \frac{\varphi(b)}{b^2}$ diverges, which is obvious, because the sum of the reciprocals of the primes is $\infty$ (but the sum of the reciprocals squared is finite), and the series contains the summands $\frac{\varphi(p)}{p^2}=\frac1p-\frac1{p^2}$.
    }{%
    https://artofproblemsolving.com/community/c6h596988p7647237
}

%  %\pitem[]{%
    16 students took part in a mathematical competition where every problem was a multiple choice question with four choices. After the contest, it is found that any two students had at most one answer in common. Prove that there are at most 5 problems in the contest.
    }{%
    At most $\binom{16}{2} = 120$ shared answers, but at least $\binom{4}{2} + \binom{4}{2} + \binom{4}{2} + \binom{4}{2} = 24$ shared answers per problem, so at most $\frac{120}{24} = 5$ problems.
    }{%
    https://artofproblemsolving.com/community/c6h1479862p8635802
}

%  %\pitem[]{%
    Show that one can construct a round-robin tournament outcome with more than 1000 people such that for any set of 1000 people, some contestant outside the set beats all of them.
    }{%
    Choose tournament outcome randomly. The result then follows from the expected number of 1000-subset failures $$ < (1-2^{1000})^n n^{1000} < 1$$when $n$ is large.
    }{%
    https://artofproblemsolving.com/community/c6h1870412
}

%  \pitem[IMOSL 1999 C4]{%
    Let $A$ be any set of $n$ residues mod $n^2$. Show that there is a set $B$ of $n$ residues mod $n^2$ that at least half of the residues mod $n^2$ can be written as $a + b$ with $a$ in $A$ and $b$ in $B$.
    }{%
    Select elements of $B=\left\{b_1,\dots ,b_n\right\}$ uniformly randomly (we'll even allow repetitions for simplicity). For each $r$ (mod $N^2$), and each $i,j\in 1,\dots ,n$, by considering the number of choices of $A$, we have
    \[\mathbb{P}(r\notin A+b_i)=\mathbb{P}(b_i\notin A-r)=\frac{N^2-N}{N^2}=1-\frac{1}{N}.\]
    Hence $\mathbb{P}(r\notin A+B)<\left(1-\frac{1}{N}\right)^N<\frac{1}{e}<\frac{1}{2}$. Thus $\EE[|A+B|]>(1-1/2)N^2$.
    %Let R be the set of all n2 residues and let S be the set of all subsets of R with n elements. Let P be the set of all pairs (r, X), where r belongs to R and X belongs to S, such that r cannot be written as a + x for any a in A, x in X.  For any given r, there are (n2-n)Cn subsets X in S such that (r, X) is in P. But (n2-n)Cn ≤ ½ (n2Cn), so at most half the subsets X have (r, X) in P (for any given r). Hence for some B in S, at most half the elements r of R have (r, B) in P. So for this B at least half the elements of R can be written as a + b with a in A and b in B.  (Note that (n2Cn/(n2-n)Cn = (n2(n2-1)(n2-2) ... (n2-n+1)/( (n2-n)(n2-n-1) ... (n2-2n+1) ) ≥ (n2/(n2-n) = (1 + 1/(n-1) )n > e > 2. ) 
    }{%
    https://prase.cz/kalva/short/soln/sh99c4.html
}

%  \pitem[Online Math Open q29]{%
    Kevin has $2^n-1$ cookies, each labeled with a unique nonempty subset of $\left\{1,2,\dots ,n\right\}$. Each day, he chooses one cookie uniformly at random out of the cookies not yet eaten. Then, he eats that cookie, and all remaining cookies that are labeled with a subset of that cookie. Compute the expected value of the number of days that Kevin eats a cookie before all cookies are gone.
    }{%
    The number of days equals the number of times a cookie is chosen (rather than merely eliminated). Let $C$ be the set of cookies chosen by the process and $S=\left\{1,\dots ,n\right\}$, so
    \[\EE[\#\textnormal{days}]=\EE[|C|]=\sum_{A\subseteq S}^{}\EE[ind(A\in C)]=\sum_{A\subseteq S}^{}\mathbb{P}(A\in C).\]
    And since cookie $A$ is uneaten iff all its supersets are also uneaten, we have
    \[\mathbb(A\in C)=\frac{1}{\#\left\{\textnormal{supersets of }A\right\}}=\frac{1}{2^{n-|A|}}.\]
    Hence 
    \[\sum_{A\subseteq S}^{}\mathbb{P}(A\in C)=\sum_{A\subseteq S}^{}2^{|A|-n}=\sum_{1\leq k\leq n}^{}\frac{{n\choose k}}{2^{n-k}}=\frac{3^n-1}{2^n}.\]
    }{%
    services.artofproblemsolving.com/download.php?id=YXR0YWNobWVudHMvMS9kL2E1NjhiZDZlNWMzMDJiODZiMTViMzMwMjI5NDMwOTY5NmY2MDhm&rn=T01PRmFsbDEzU29sbnMucGRm
}

%  \pitem[EGMO 2019 P5]{%
    Let $n\ge 2$ be an integer, and let $a_1, a_2, \cdots , a_n$ be positive integers. Show that there exist positive integers $b_1, b_2, \cdots, b_n$ satisfying the following three conditions:

    $\text{(A)} \ a_i\le b_i$ for $i=1, 2, \cdots , n;$

    $\text{(B)} \ $ the remainders of $b_1, b_2, \cdots, b_n$ on division by $n$ are pairwise different; and

    $\text{(C)} \ $ $b_1+b_2+\cdots b_n \le n\left(\frac{n-1}{2}+\left\lfloor \frac{a_1+a_2+\cdots a_n}{n}\right \rfloor \right)$
    }{%
    Note that if $a_i\ge n$, we can replace $a_i$ with $a_i-n$ and also $b_i$ with $b_i-n$ without affecting the validity of the problem (and if it holds for $a_i>0$, there is no reason for it to fail at $a_i=0$); therefore, we may assume that each $a_i$ is between 0 and $n-1$.

    Define a new sequence $c_i=b_i\pmod{n}$. For some $a_i$, we randomly select the value of the corresponding $c_i$; notice that there is a $\frac{a_i}{n}$ chance that the value of $c_i$ is less than $a_i$, and therefore it will be necessary that $b_i$ is equal to $c_i+n$. We have $\sum b_i\le \sum a_i+\sum c_i$ by LoE; notice that $\sum c_i=\frac{n(n-1)}{2}$, and by mod $n$, we can indeed replace $\sum a_i$ with $n\lfloor\frac{\sum a_i}{n}\rfloor$, and therefore obtaining the bound in the original problem statement.
    }{%
    https://artofproblemsolving.com/community/c6h1819303p20292058
}

%  \pitem[IMC 2017 q4]{%
    There are $n$ people in a city, and each of them has exactly $1000$ friends (friendship is always symmetric). Prove that it is possible to select a group $S$ of people such that at least $\frac{n}{2017}$ persons in $S$ have exactly two friends in $S$.
    }{%
    Let $d=1000$ and let $0<p<1$. Choose the set $S$ randomly such that each people is selected with probability $p$, independently from the others.

    The probability that a certain person is selected for $S$ and knows exactly two members of $S$ is $q={d\choose 2}p^3(1-p)^{d-2}$. Choose $p=3/(d+1)$ (this is the value of $p$ for which $q$ is maximal); then
    \begin{alignat*}{1}
        q&= {d\choose 2}\left(\frac{3}{d+1}\right)^3\left(\frac{d-2}{d+1}\right)^{d-2}\\
         &= \frac{27d(d-1)}{2(d+1)^3}\left(1+\frac{3}{d-2}\right)^{-(d-2)}\\
         &>\frac{27d(d-1)}{2(d+1)^3}\cdot e^{-3}\\
         &> \frac{1}{2017}.
    \end{alignat*}
    Hence $E(\left|S\right|)=nq>\frac{n}{2017}$, so there is a choice for $S$ when $\left|S\right|>\frac{n}{2017}$.
    }{%
    https://www.imc-math.org.uk/?year=2017&section=problems&item=prob4s
}

%  \pitem[Iran TST 2008 P6]{%
    Suppose $799$ teams participate in a round-robin tournament. Prove that one can find two disjoint groups $A$ and $B$ of seven teams each such that all teams in $A$ defeated all teams in $B$.
    }{%
    We are going to prove we can partition the group of 14 into two groups of 7.

    Let $D_k$ be the set of teams which defeat the $k$-th team (here $1\leq k\leq 799$), and $d_k=|D_k|$. Select $A=\left\{a_1,\dots ,a_7\right\}$ randomly, so $\mathbb{P}(A\subseteq D_k)={d_k\choose 7}/{799\choose 7}$. Let $N$ be the number of teams domintated by $A$, we have $\EE[N]=\sum_{k}^{}{d_k\choose 7}/{799\choose 7}$. Since ${x\choose 7}$ is convex, and the average value of $d_k$ is $798/2=398$, by Jensen's we have
    \[\EE[N]\geq 799\cdot {398\choose 7}/{799\choose 7}>799\cdot \left(\frac{1}{2}\right)^7>6.\]
    Hence sometimes $N\geq 7$.
    }{%
    https://artofproblemsolving.com/community/c6h206628p1136962
}

%  \pitem[Caro-Wei Theorem]{%
    Consider a graph $G$ with vertex set $V$ . Prove that one can find an independent set of $V$ (pairwise non-adjacent) with size at least
    \[\sum_{v\in V}^{}\frac{1}{\deg v + 1}.\]
    \begin{remark}
        By applying Jensen's inequality, our independent set has size at least $\frac{n}{d+1}$, where $d$ is the average degree. This result is called Turan's Theorem.
    \end{remark}
    }{%
    Randomly order the vertices $\left\{v_1,\dots ,v_n\right\}$ of $G$, and take $W$ to be the subset of those $v_i$ which occur before all their neighbors. Then
    \[\EE[|W|]=\sum_{i}^{}ind(v_i\in W)=\sum_{i}^{}\frac{1}{deg(v)+1}.\]
    }{%
    https://cs.stackexchange.com/questions/100650/alternate-proof-of-the-caro-wei-theorem-for-lower-bounding-the-independence-numb
}

%\end{question}


\end{document}
