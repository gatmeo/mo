\documentclass[a4paper]{article}
\usepackage[math_simple,imo]{gatmeo}

\renewcommand{\courseTitle}{\FirstBigRestSmallCaps{IMO Phase II Level 2}}
\renewcommand{\courseTopic}{\FirstBigRestSmallCaps{Bijective Proof}}
\DTMsavedate{mydate}{2022-11-26}

%\toggletrue{ownans}
\togglefalse{officialans}
\rfoot{}

\begin{document}
\maketitle
\thispagestyle{empty}
%\tableofcontents
\begin{mdframed}
  \textbf{Bijective Proof}: The main idea is similar to Double Counting: we are interested in finding a different (easier) way to count the required quantity.
\end{mdframed}
\vspace{.5em}

%https://artofproblemsolving.com/community/c6t45317f6h2590271_double_counting_olympiad_combinatorics
%https://artofproblemsolving.com/community/c6t45317f6h2743372_counting_in_2_ways
\begin{example_question*}[]{}
  \begin{mdframed}
    \pitem[]{%
    Let $n$ be a positive integer. Determine the number of lattice paths from $(0, 0)$ to $(n, n)$ using only unit up and right steps, such that the path stays in the region $x\geq y$.
    }{%
    It is easy to see that the total number of lattice paths from $(0, 0)$ to $(n, n)$ without the $x\geq y$ restriction is $\binom{2n}{n}$. Let us count the number of paths that goes into the $x<y$ region. Call these paths bad paths.  
    We are going to prove there is a bijection between bad paths and paths from $(-1,1)$ to $(n,n)$, hence the total number of good paths equal
    \[
        \binom{2n}{n}-\binom{2n}{n+1}=\frac{1}{n+1}\binom{2n}{n}.
    \]
    Suppose that $P$ is a bad path. Since $P$ goes into the region $x < y$, it must hit the line $y = x + 1$ at some point. Let $X$ be the first point on the path $P$ that lies on the line $y = x + 1$. Now, reflect the portion of path $P$ up to $X$ about the line $y = x + 1$, keeping the latter portion of $P$ the same. This gives us a new path $P'$ starting from $(-1,1)$. 

    For the inverse construction, for any path from $(-1,1)$ to $(n,n)$, let $X$ be the first point on the path lies on the line $y = x + 1$, and let $Q'$ be constructed from $Q$ by reflecting the first portion of $Q$ up to $X$ through the line $y = x + 1$ and keeping the rest the same. Then the inverse of the bijection given above sends $Q$ to $Q'$.

    The remaining part is to prove these two constructions are of an inverse relation. We will leave this as a simple exercise.
    }{%
    https://yufeizhao.com/olympiad/bijections.pdf
}

    \vspace{-1em}
  \end{mdframed}

  \begin{mdframed}
    \pitem[]{%
    For $n\geq 1$, prove that
    \[
        \sum_{k=0}^{n}(-1)^k\binom{n}{k}=0.
    \]
    }{%
    Take the elements of $[n]$. Take the element $1$ (we know that $n\geq 1$). For every subset $S$ of $[n]$ with $k$ elements, where $k$ is even, we can get a unique subset with an odd number of elements as follows: if $1\in S$ remove it, otherwise, add it. And obviously, every subset with an odd number of element arise from this way. Hence there is a bijection between even subsets and odd subsets.
    }{%
    1.10
}

    \vspace{-1em}
  \end{mdframed}
\end{example_question*}

\newpage

\begin{question*}[]{}
  \pitem[]{%
    For $n\geq 0$, prove that
    \[
        \sum_{k=0}^{n}\binom{x}{k}\binom{y}{n-k}=\binom{x+y}{n}.
    \]
    }{%
    Let us choose $n$ elements from the set $\left\{1,2,\dots ,x,x+1,\dots ,x+y\right\}$. Some of these $n$ elements have to be among $1,2,\dots ,x$. Let the number of those be $k$. The remaining $n-k$ elements have to be among $x+1,\dots ,x+y$. There exists $\binom{x}{k}\binom{y}{n-k}$ such configurations and $k$ could be between $0$ and $n$. This is the so-called Vandermonde identity.
    }{%
    1.11
}

  \pitem[]{%
    A composition of $n$ is a sequence $\alpha=(\alpha_1,\alpha_2,\dots ,\alpha_k)$ of positive integers such that $\sum\alpha_i=n$. Proof that 
    \begin{enumerate}
        \item the number of compositions of $n$ is $2^{n-1}$,
        \item the total number of parts of all compositions of $n$ is equal to $(n+1)2^{n-2}$, and
        \item for $n\geq 2$, the number of compositions of $n$ with an even number of even parts is equal to $2^{n-2}$.
    \end{enumerate}
    }{%
    \begin{enumerate}
        \item The compositions $\alpha=(\alpha_1,\alpha_2,\dots ,\alpha_k)$ is in bijection to subsets of $[n-1]$: $\alpha\mapsto \left\{\alpha_1,\alpha_1+\alpha_2,\dots ,\sum_{i=1}^{k-1}\alpha_i\right\}$.
        \item Use the "dots and bars" model. The bar putted after dot $i$ participates in $2^{n-2}$ partitions in total (choose 'put'/ 'don’t put a bar' after the remaining $n-2$ dots among $1,\dots ,n-1$). Thus we have $(n-1)\cdot 2^{n-2}$ parts ending before the $n$-th dot, in total. That way, the last part in every composition is missed. After we add it, i.e. after addition of $2^{n-1}$ parts, we get $(n-1)2^{n-2}+2^{n-1}=(n+1)2^{n-2}$.
        \item Assume you have determined to put or not a bar for each of the positions after the first $n-2$ dots. You can do this in $2^{n-2}$ ways. You get a composition with last part of length $\geq 2$. Take the number of even parts in the received composition. If this number is even, keep the composition as it is. Otherwise, put a bar on the last possible position, i.e. after dot $n-1$. In fact, this way you divide the last part of length $l\geq 2$ to two parts of lengths $l-1$ and $1$. As a result, the parity of the last part is changed and the received
composition will have even number of even parts.We actually get a bijection between the compositions with odd number of even parts and those with even number of even parts.
    \end{enumerate}
    }{%
    1.2,3,4
}

  \pitem[]{%
    The Fibonacci numbers $F_n$ are defined by $F_1=F_2=1$ and $F_{n+1}=F_n+F_{n_1}$ for $n\geq 2$. Prove that
    \begin{enumerate}
        \item the number $f(n)$ of compositions of $n$ with parts $1$ and $2$ is $F_{n+1}$,
        \item the numberof compositions of $n$ with all parts $>1$ is $F_{n-1}$, and
        \item the number of compositions of $n$ with odd parts is $F_n$.
    \end{enumerate}
    }{%
    \begin{enumerate}
        \item Clearly, $F(1)=1=F_2$ and $f(2)=2=F_3$. Consider the $f(n)$ compositions of $n\geq 3$ with parts $1$ and $2$. The number of them having last part $1$ is $f(n-1)$ and those having last part $2$ is $f(n-2)$ since after removing this last part, we obtain another compositions with parts $1$ and $2$ for $n-1$ and $n-2$ respectively. Also, having any of the $f(n-2)$ (or the $f(n-1)$) compositions to 1s and 2s of $n-2$ (or $n-1$), we get such composition for $n$ by adding $2$ (or $1$) at the end.
        \item Denote the number of compositions of $n$ with all parts $>1$ with $g(n)$. Trivially, $g(2)=g(3)=1$. Consider all the $g(n)$ compositions for some $n>3$. If the last part is $2$ for some of them and we remove this part, we get a composition of $n-2$ having all parts $> 1$. Now, consider the case when the last part is $\geq 3$. If we remove $1$ from this last part, we obtain partition with all parts $> 1$ of $n-1$. On the other hand, we can add one part of size $2$ to the end of each composition among those $g(n-2)$ or 1 to the last part of each such composition of $n-1$ to obtain one of the compositions in $g(n)$. Since the only options are to have last part $2$ or last part $\geq 3$ and these two subsets of compositions does not intersect, we can conclude that $g(n)=g(n-1)+g(n-2)$.
        \item Denote the number of described compositions of $n$ with $h(n)$. We have $h(1) = 1 = F(1)$ and $h(2) = 1 = F(2)$. Use similar observations to those for the previous problem.  Given a composition with odd parts, we could have either last part 1 or otherwise we can remove $2$ from this last part.
    \end{enumerate}
    }{%
    1.30 - 1.32
}

  \pitem[]{%
    Fix positive integers $n$ and $k$. Find the number of $k$-tuples $(S_1,S_2,\dots ,S_k)$ of subsets of $S_i$ of $\left\{1,\dots ,n\right\}$ for each of the following conditions separately:
    \begin{enumerate}
        \item $S_1\subseteq S_2\subseteq \cdots \subseteq S_k$,
        \item The $S_i$'s are pairwise disjoin,
        \item $S_1\cap S_2\cap \cdots \cap S_k=\emptyset $, and
        \item $S_1\subseteq S_2\supseteq S_3\subseteq S_4\supseteq S_5\subseteq \cdots $.
    \end{enumerate}
    }{%
    \begin{enumerate}
        \item For every number $a$ in $\left\{1,\dots ,n\right\}$, we just need to choose the least index $i_a$ of a set s.t. $a\in S_{i_{a}}$ ($i_a=0$ if neither of the subsets contains $a$) in order to determine the sets $S_1\subseteq S_2\subseteq \cdots \subseteq S_k$. We see that for every $a$ we have $k+1$ possibilities for the choice of this index, namely $0,1,\dots ,k$. Thus the answer is $(k+1)^n$.
        \item The answer is $(k+1)^n$ again. Use first part and the fact that for every set of subsets with the inclusion $S_1\subseteq S_2\subseteq \cdots \subseteq S_k$ we have a corresponding set of pairwise disjoint subsets $S_1,S_2\setminus S_1,\dots ,S_k\setminus \cup _{i=1}^{k-1}S_i$. Another way to get the answer is to observe that for every number $a$ in $\left\{1,\dots ,n\right\}$ could belongs only to a set with index $i(S_i),i=1,\dots ,k$ or to neither of those sets, i.e. $k+1$ cases for $n$ different numbers.
        \item Here, as in the previous parts of the problem, in order to determine such configuration, we have to choose one $0-1$ vector of length $k$ for every number $i\in \left\{1,\dots ,n\right\}$. The additional condition is that none of these vectors could be consisted of 1s entirely. Therefore, the answer is $(2^k-1)^n$.
        \item For every number $a\in \left\{1,\dots ,n\right\}$ and every $i\in \left\{1,\dots ,k\right\}$ write $\varepsilon_i^a=1$ if $a\in S_i$ and $0$ otherwise. You'll get that for every fixed $a\in \left\{1,\dots ,n\right\}$, the sequence $\left\{\varepsilon_i^a\right\}_{i=1}^{k}$ suffices the condition $\varepsilon_1^a\leq \varepsilon_2^a\geq \varepsilon_3^a\leq \varepsilon_4^a\geq \varepsilon_5^a\leq \cdots $. Using the part below, the answer is $F^n_k+2$.
    \end{enumerate}
    }{%
    1.5, 1.41
}

  \pitem[]{%
    Find the number of subsets of $S$ of $[n]$ that do not contain two consecutive integers.
    }{%
    Let the number of those subsets be denoted with $g(n)$. Check that $g(1)=2^1$ and $g(2)=2^2-1=3$. Consider those $S\subseteq [n]$ containing $n$ and those not containing $n$. For the former case, we know that $n-1\notin S$ and $S\setminus n$ does not contain two consecutive integers. The set $S\setminus n$ could be any such subset of $[n-2]$. Thus we get $g(n-2)$ subsets of $[n]$ containing $n$ and withouth two consecutive integers in them. For the case $n\notin S$, we know that $S\subseteq [n-1]$ without having two consecutive integers in it and that each such $S$ is a subsets of $[n]$. Therefore, we obtain $g(n)=g(n-1)+g(n-2)$ for $n\geq 3$ and given the initial conditions we get that $g(n)=F_{n+2}$.
    }{%
    1.33
}

  \pitem[]{%
    Find the number of binary sequences $(\varepsilon_1,\dots ,\varepsilon_n)$ satisfying
    \[
        \varepsilon_1\leq \varepsilon_2\geq \varepsilon_3\leq \varepsilon_4\geq \varepsilon_5\leq \cdots .
    \]
    }{%
    Denote this number by $g(n)$ and call the sequences sufficing this property "good". If $\varepsilon_1=0$, then $(\varepsilon_2,\dots ,\varepsilon_n)$ is also "good". In addition, for any "good" $(\varepsilon_2,\dots ,\varepsilon_n)$ of length $n-1$, we can add one 0 in front to make a "good" sequence starting with a $0$ of length $n$. Thus there are $g(n-1)$ such sequences. On the other hand, if $\varepsilon_1=1$, then $\varepsilon_2$ have to be $1$, too and the rest is a "good" sequence of length $n-2$ (and this sequence could be any of those $g(n-2)$). Thus we have the relation $g(n)=g(n-1)+g(n-2)$ again and it remains to check that $g(1)=2$ and $g(2)=3$ to obtain that the answer is $F_{n+2}$.
    }{%
    1.34
}

  \pitem[]{%
    For $n\geq 0$, prove that
    \[
        \sum_{k=0}^{n}\binom{x+k}{k}=\binom{x+n+1}{n}.
    \]
    }{%
    Consider a choice of $n$ elements out of $x+n+1$ elements. Among the last $n+1$ there must be at least one unchosen element. Denote the unchosen element with the biggest number with $x+k+1$. We know that the $n-k=(x+n+1)-(x+k+1)$ elements having bigger numbers are selected, thus in order to determine the choice of the $n$ elements completely, we have to select the rest $n-(n-k)=k$ elements from the remaining $x+k$ elements. This reasoning shows that the given equality is a partition of all possible choices of $n$ elements out of $x+n+1$ elements by the number of the largest unchosen element.
    }{%
    1.12
}

  %\pitem[]{%
    How many $m\times n$ matrices of 0's and 1's are there, such that every row and column contains an odd number of 1's?
    }{%
    <++>
    }{%
    1.19
}

  \pitem[]{%
    Let $p$ be a prime. Then $\binom{2p}{p}-2$ is divisible by $p^2$.
    }{%
    Take a set of $2p$ elements $A$. Consider the set of all $p$-subsets of $A$. Represent one such subset by a $2\times p$ table of $0$s and $1$s, containing exactly $p$ 1s. Remove the trivial choices, i.e., those $2$ choices having all 1s in the first row or in the second row.  Thus, all elements of the remaining set of choices have both 0s and 1s in each of the two rows. Therefore, if we shift the first row by $i$ positions, $i=0,1,\dots ,p-1$, we would always obtain a different $0-1$ vector. The same could be said for the second row. Now, we see that each choice of $p$-subset belongs to a class of $p\times p$ choices obtained by shifting the first and the second row of this choice to $i$ and $j$ positions respectively ($i,j=0,1,\dots ,p-1$). Every such class is closed thus $\binom{2p}{p}-2$  could be divided by $p^2$ (we removed the two trivial choices).
    \[
        \textnormal{Example:} \qquad 
        \begin{pmatrix}
            1&0&0&1&\cdots &0\\1&0&1&0&\cdots &1
        \end{pmatrix}
    \]
    }{%
    1.22
}

  \pitem[]{%
    Show that
    \[
        \sum_{}^{}(2^{a_1-1}-1)\cdots (2^{a_k-1}-1)=F_{2n-2},
    \]
    where the sum is over all compositions $a_1+a_2+\cdots +a_k=n$.
    }{%
    Given a composition $(a_1,a_2,\dots ,a_k)$ of $n$, replace each part $a_i$ with a composition $a_i$ of $2a_i$ into parts $1$ and $2$, such that $a_i$ begins with $1$, ends in 2, and for all $j$ the $2j$-th 1 in $a$ is followed by a 1, unless this $2j$-th $1$ is the last $1$ in $a$. For instance, the part $a_i=4$ can be
    replaced by any of the seven compositions $1111112, 111122, 111212, 11222, 121112, 12122, 12212$.
    Now, let’s check the following two facts that suffices to get a proof:

    \begin{enumerate}
        \item every composition of $2n$ into parts $1$ and $2$, beginning with $1$ and ending with $2$, occurs exactly once by applying this procedure to all compositions of $n$. Indeed, assume we have such composition. One will be able to divide it into segments representing the different $2\alpha_i$-s forming a corresponding composition of $n$. The latter is true, because it is enough to look for those $2$ followed by $1$ and having even number of $1$s in front of them. All of these 2s and only they are ends of a segments representing a part. For example, if we are given the $1-2$ composition $111111211121212122$, we can see that it corresponds to $1111112 | 111212 | 12122$, which is one of the possible representations for the partition of 12 - (4,4,4).
        \item the number of compositions that can replace $a_i$ is $2^{a_i-1}-1$. Well, here is a way that we can create all such compositions. Write 1 at the beginning, 2 at the end and choose $a_i-1$ times to write "11" or "2" for the middle portion(you can do this in $2^{a_i}-1$ ways). But, the total sum became $2n+1$. It remains to remove the last $1$. However, this could be done if you have at least one $1$. To gurantee this, just do not consider the choice of $a_i-1$ 2s for the middle portion. Now, the number of ways you can perform the described steps is $2^{a_i-1}-1$.
    \end{enumerate}
    Now the number of compositions of $2n$ starting with $1$ and ending with $2$ is $F_{2n-3+1}=F_{2n}$.
    }{%
    1.36
}

\end{question*}

\end{document}
