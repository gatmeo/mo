\documentclass[a4paper]{article}
\usepackage[math_simple,imo,asy]{gatmeo}

\renewcommand{\courseTitle}{\FirstBigRestSmallCaps{2020 IMO Intensive Training}}
\renewcommand{\courseTopic}{\FirstBigRestSmallCaps{Problem Set: Combinatorics: Coloring}}
\DTMsavedate{mydate}{2021-06-12}

\toggletrue{ownans}
\togglefalse{officialans}
\rfoot{}

\begin{document}
\maketitle
\thispagestyle{empty}

\begin{question*}{}
    \pitem[2017 Belarus Team Selection Test 6.2]{%
        Any cell of a $5\times 5$ table is colored black or white.  Find the greatest possible value of the ways to place a $T$-tetramino on the table so that it covers exactly two white and two black cells (for various colorings of the table).
        }{%
        The answer is $37.$

        To see that $37$ is attainable, consider the following table, where $W$ means white and $B$ means black.

        \[ \begin{bmatrix} W & W & B & B & W \\ B & B & W & W & B \\ W & W & B & B & B \\ B & B & W & W & B \\ W & W & B & B & W \end{bmatrix} \]
        Now, we will show that $37$ is optimal. For a $T-$tetromino, we will define its $\mathbf{core}$ as the cell which is in its center (the one neighboring the other three). Notice that any point is the core of at most $3$ $T-$tetrominoes, with equality iff it is surrounded by $3$ cells of its opposite color and $1$ of the same color. Furthermore, any point on the boundary is the core of at most $1$ $T-$tetromino and the corners cannot be the cores of any $T-$tetrominoes. Therefore, this gives an upper bound of $3 * 3 * 3 + 12 * 1 = 27 + 12 = 39.$

        Assuming that $39$ was possible, we can WLOG assume that the center is white and then uniquely determine the other cells (up to rotation) one by one, and easily find that $39$ is not possible.

        Hence, we've shown that the answer is at most $38.$ Now, notice that in the case for $38,$ there is only one of the squares which is "bad," in the sense that it's the core of one too few $T-$tetrominoes. Then, caseworking on whether this bad square is on the edge or in the middle $3 \times 3$, and then splitting into subcases on where it is will finish (it's pretty straightforward, since using the non-bad squares turns out to always be enough to uniquely determine the grid).

        Note that an interior cell is bad iff it has $2$ neighbors of each color.
        }{%
        https://artofproblemsolving.com/community/c6h1813398p12089401
    }

    \pitem[BMO 2020 q3]{%
        A $2019\times2019$ grid is made up of $2019^2$ unit cells. Each cell is colored either black or white. A coloring is called balanced if, within every square subgrid made up of $k^2$ cells for $1\leq k\leq2019$, the number of black cells differs from the number of white cells by at most one. How many different balanced colorings are there?

        (Two colorings are different if there is at least one cell that is black in exactly one of them.)
        }{%
        I believe the answer is $\boxed{2^{1011} - 6}.$

        First of all, there are $2$ ways to color the $2019 \times 2019$ board checkerboard style, and these two colorings obviously work.

        We'll now show that there are $2^{1011} - 8$ other colorings which work.

        Call a row or column tasty in a given coloring if the squares in that row or column are colored alternately black and white.

        Claim. In any balanced coloring, either every row is tasty of every column is tasty.

        Proof. If the coloring is checkerboard style, the claim is obvious. Else there must exist two adjacent squares which are the same color. WLOG they are in the same row, the other case is similar. Then, it's easy to see that the columns containing these two squares are both tasty and are identical. From here, it's not hard to see that all columns must be tasty (if not, consider the "closest" non-tasty column to the two columns).

        $\blacksquare$

        Note that since we're looking at colorings which aren't checkerboard style, we only have two cases.

        Case 1. All rows are tasty.

        We claim that there are $2^{1010}-4$ ways to choose the leftmost column which generate a balanced $2019 \times 2019$ grid (which isn't colored checkerboard style) after enforcing the condition that all rows are tasty. This would solve the problem because the case where all columns are tasty is similar.

        Note that the condition of being balanced is equivalent to any $1 \times t$ subgrid of the column having its numbers of black and white squares differ by at most $1$, for any odd $t.$

        Since the coloring isn't checkerboard style, there are two adjacent squares in the column which are the same color. Now, partition the column into dominoes and a single $1 \times 1$ square at one end of the column so that these two adjacent squares are not in the same domino. We claim that the two squares of any domino are oppositely colored. Indeed, this is not hard to verify with induction by consider $k \times k$'s with $k$ odd.

        After this observation, the finish is not far. Indeed, there are two ways to tile into dominoes (pick where the $1 \times 1$ is) and two ways to select each piece, for a total of $2 \cdot 2^{1009} = 2^{1010}$ ways to color the column. However, observe that the two colorings where the column is alternately colored are each counted twice in this calculation, so our true total is $2^{1010} - 2 \cdot 2 = 2^{1010} - 4.$

        Case 2. All columns are tasty.

        This case is analogous to the above.

        Considering all cases, we arrive at our answer of $2 \cdot (2^{1010}-4) + 2 = \boxed{2^{1011} - 6}.$
        }{%
        https://artofproblemsolving.com/community/c6h2009907p14093485
    }

    \pitem[USA TST for EGMO 2020]{%
        Vulcan and Neptune play a turn-based game on an infinite grid of unit squares. Before the game starts, Neptune chooses a finite number of cells to be flooded. Vulcan is building a levee, which is a subset of unit edges of the grid (called walls) forming a connected, non-self-intersecting path or loop*.

        The game then begins with Vulcan moving first. On each of Vulcan’s turns, he may add up to three new walls to the levee (maintaining the conditions for the levee). On each of Neptune’s turns, every cell which is adjacent to an already flooded cell and with no wall between them becomes flooded as well. Prove that Vulcan can always, in a finite number of turns, build the levee into a closed loop such that all flooded cells are contained in the interior of the loop, regardless of which cells Neptune initially floods. 

        *More formally, there must exist lattice points $\mbox{\footnotesize \(A_0, A_1, \dotsc, A_k\)}$, pairwise distinct except possibly $\mbox{\footnotesize \(A_0 = A_k\)}$, such that the set of walls is exactly $\mbox{\footnotesize \(\{A_0A_1, A_1A_2, \dotsc , A_{k-1}A_k\}\)}$. Once a wall is built it cannot be destroyed; in particular, if the levee is a closed loop (i.e. $\mbox{\footnotesize \(A_0 = A_k\)}$) then Vulcan cannot add more walls. Since each wall has length $\mbox{\footnotesize \(1\)}$, the length of the levee is $\mbox{\footnotesize \(k\)}$.

        Reference: https://artofproblemsolving.com/community/c6h1970140\_greek\_gods\_flood\_the\_world
        }{%
        We can draw a square containing all initially flooded cells; let the side length be $s$. We will prove the problem when the all cells inside this square are initially flooded, since removing initially flooded cells does not harm Vulcan. In this four-step construction, Vulcan will build 3 new walls each turn.

        The following diagrams include the square of side length $s$ outlined in green, the flooded cell boundary outlined in blue, the newly built walls in red, and the previously built walls in black.

        \rasy[2.2in]{
        1. In Vulcan's first $2s$ turns, he will build the top right corner of the levee such that the top and right walls are $2s$ away from the top and right sides of the square. This is possible because only immediately after $2s$ turns does the flood boundary reach the levee.
            }{
            import MOgeom;
            pair A, B, C; A = (-15, 0); B = (0, 0); C = (0, -15); draw(A--B--C, red+linewidth(2)); pair X, Y; X = (-10, 0); Y = (0, -10); draw(A--X, blue); draw((-10, 0)--(-10, -1)--(-9, -1)--(-9, -2)--(-8, -2)--(-8, -3), blue); draw((0, -15)--(0, -10)--(-1, -10)--(-1, -9)--(-2, -9)--(-2, -8)--(-3, -8), blue); draw((0, -15)--(-1, -15)--(-1, -16)--(-2, -16)--(-2, -17)--(-3, -17), blue); draw((-15, 0)--(-15, -1)--(-16, -1)--(-16, -2)--(-17, -2)--(-17, -3), blue); draw((-25, -10)--(-25, -15)--(-24, -15)--(-24, -16)--(-23, -16)--(-23, -17)--(-22, -17), blue); draw((-25, -10)--(-24, -10)--(-24, -9)--(-23, -9)--(-23, -8)--(-22, -8), blue); draw((-15, -25)--(-10, -25)--(-10, -24)--(-9, -24)--(-9, -23)--(-8, -23)--(-8, -22), blue); draw((-15, -25)--(-15, -24)--(-16, -24)--(-16, -23)--(-17, -23)--(-17, -22), blue); draw((-4.5, -6.5)--(-6.5, -4.5), dotted+blue+linewidth(2)); draw((-20.5, -6.5)--(-18.5, -4.5), dotted+blue+linewidth(2)); draw((-20.5, -18.5)--(-18.5, -20.5), dotted+blue+linewidth(2)); draw((-4.5, -18.5)--(-6.5, -20.5), dotted+blue+linewidth(2)); draw((-15, -15)--(-15, -10)--(-10, -10)--(-10, -15)--cycle, heavygreen); label("$s$", (-12.5, 0), N); label("$2s$", (-5, 0), N); 
        }

        \rasy[2.2in]{
            2.  In each of the next $5s$ turns, Vulcan will build one wall downward on the right side of the levee and two walls to finish $7s$ of the top wall and continue downward on the left side. The left side of the flood boundary just reaches the left side of the levee after $7s$ turns in total.
            }{
            import MOgeom;
            draw((-50, -15)--(-50, 0)--(-15, 0), red+linewidth(2)); draw((-15, 0)--(0, 0)--(0, -15), black+linewidth(2)); draw((0, -15)--(0, -40), red+linewidth(2)); draw((-15, -15)--(-15, -10)--(-10, -10)--(-10, -15)--cycle, heavygreen); draw((-50, -15)--(-49, -15)--(-49, -16)--(-48, -16)--(-48, -17)--(-47, -17)--(-47, -18)--(-46, -18), blue); draw((0, -40)--(-1, -40)--(-1, -41)--(-2, -41)--(-2, -42)--(-3, -42)--(-3, -43)--(-4, -43), blue); draw((-15, -50)--(-10, -50), blue); draw((-10, -50)--(-10, -49)--(-9, -49)--(-9, -48)--(-8, -48)--(-8, -47), blue); draw((-15, -50)--(-15, -49)--(-16, -49)--(-16, -48)--(-17, -48)--(-17, -47)--(-18, -47)--(-18, -46), blue); draw((-7, -46)--(-5, -44), blue+dotted); draw((-19, -45)--(-45, -19), blue+dotted); label("$3s$", (-7.5, 0), N); label("$7s$", (-65/2, 0), N); label("$3s$", (-50, -7.5), W); label("$5s$", (0, -55/2), E); 
        }

        \rasy[2.2in]{
            3. In each of the next $5s$ turns, Vulcan will build two walls on the left side of the levee and one wall on the right, so the left and right walls are equal in length.
            }{
            import MOgeom;
            draw((-50, -15)--(-50, 0)--(-15, 0), linewidth(2)); draw((-15, 0)--(0, 0)--(0, -15), linewidth(2)); draw((0, -15)--(0, -40), linewidth(2)); draw((-15, -15)--(-15, -10)--(-10, -10)--(-10, -15)--cycle, heavygreen); draw((0, -40)--(0, -65), red+linewidth(2)); draw((-50, -15)--(-50, -65), red+linewidth(2)); draw((-50, -40)--(-49, -40)--(-49, -41)--(-48, -41)--(-48, -42)--(-47, -42)--(-47, -43)--(-46, -43), blue); draw((0, -65)--(-1, -65)--(-1, -66)--(-2, -66)--(-2, -67)--(-3, -67)--(-3, -68)--(-4, -68), blue); draw((-15, -75)--(-10, -75), blue); draw((-10, -75)--(-10, -74)--(-9, -74)--(-9, -73)--(-8, -73)--(-8, -72), blue); draw((-15, -75)--(-15, -74)--(-16, -74)--(-16, -73)--(-17, -73)--(-17, -72)--(-18, -72)--(-18, -71), blue); draw((-7, -71)--(-5, -69), blue+dotted); draw((-19, -70)--(-45, -44), blue+dotted); label("$10s$", (-25, 0), N); label("$3s$", (-50, -7.5), W); label("$8s$", (0, -20), E); label("$10s$", (-50, -40), W); label("$5s$", (0, -105/2), E); 
        }

        \rasy[2.2in]{
            4.  In the final $14s$ moves, we increase the length downward of one of the left or right walls by 1 and the other by 2. Once the total length of the left or right wall reaches length $29s$, we begin constructing the bottom wall of the levee. Note that after these $14s$ moves (and $26s$ moves in total), the bottom edge of the flood boundary just hits the bottom of levee ($26s$ below the bottom edge of the green square).
            }{
            import MOgeom;
            draw((-50, -15)--(-50, 0)--(-15, 0), linewidth(2)); draw((-15, 0)--(0, 0)--(0, -15), linewidth(2)); draw((0, -15)--(0, -40), linewidth(2)); draw((-15, -15)--(-15, -10)--(-10, -10)--(-10, -15)--cycle, heavygreen); draw((0, -40)--(0, -65), linewidth(2)); draw((-50, -15)--(-50, -65), linewidth(2)); draw((-50, -65)--(-50, -145)--(0, -145)--(0, -65), linewidth(2)+red); draw((-50, -110)--(-49, -110)--(-49, -111)--(-48, -111)--(-48, -112)--(-47, -112)--(-47, -113)--(-46, -113), blue); draw((0, -135)--(-1, -135)--(-1, -136)--(-2, -136)--(-2, -137)--(-3, -137)--(-3, -138), blue); draw((-15, -145)--(-10, -145), blue); draw((-10, -145)--(-10, -144)--(-9, -144)--(-9, -143)--(-8, -143), blue); draw((-15, -145)--(-15, -144)--(-16, -144)--(-16, -143)--(-17, -143)--(-17, -142)--(-18, -142)--(-18, -141), blue); draw((-7, -141)--(-5, -139), blue+dotted); draw((-19, -140)--(-45, -114), blue+dotted); label("$10s$", (-25, 0), N); label("$13s$", (-50, -65/2), W); label("$16s$", (-50, -105), W); 
        }
        }{%
        https://artofproblemsolving.com/community/c6t309f6h1970142_roman_gods_flood_the_world

        https://artofproblemsolving.com/community/c6h1970140_greek_gods_flood_the_world
    }

    \pitem[Croatia TST 2016]{%
        Let $N$ be a positive integer. Consider a $N \times N$ array of square unit cells. Two corner cells that lie on the same longest diagonal are colored black, and the rest of the array is white. A move consists of choosing a row or a column and changing the color of every cell in the chosen row or column.
        What is the minimal number of additional cells that one has to color black such that, after a finite number of moves, a completely black board can be reached?
        }{%
        For $2 \times 2$, it's clear to see that there's no need for additional cells.

        For $N \times N$ with $N \ge 3$, we rephrase the problem to have a better approach. We denote a black cell as it has $-1$ in it, a white cell as it has $1$ in it. A move is denoted as choosing a column or a row and times every number in that row (or column) by $-1$. Hence, if we denote $a_1,a_2, \cdots , a_N$ as the number of times we time column $1,2 \cdots , N$ by $-1$ (or number of times making a move), respectively. Similar to $b_1,b_2, \cdots , b_N$ for column. Hence, after some move to get all black board, cell $(i,j)$ (column $i$, row $j$) has timed to $(-1)^{a_i+b_j}$.

        This means that if a cell $(i,j)$ has black as its initial colour, or has $-1$ as it initial number, then $(-1)\cdot(-1)^{a_i+b_j}=-1$ or $a_i+b_j$ is even. If a cell $(i,j)$ has $1$ as it initial number, then $a_i+b_j$ is odd.

        Now, to the finding minimum part.

        \rasy[2.2in]{
            Case 1. Initially, if column $1$ has only cell $(1,1)$ coloured black (or has $-1$ written on it) then $a_1+b_i$ must have the same parity for all $2 \le i \le N$ (since they all start with white colour and all end with black colour) or $b_i \; ( 2 \le i \le N)$ have the same parity. On the other hand, since cell $(N,N)$ is black cell therefore, cells $(N,i) \; (2 \le i \le N)$ must also have black cells at the beginning in order to satisfy all $b_i \; (2 \le i \le N)$ have same parity.

            Since $b_i \; (2 \le i \le N)$ must have the same parity so consider any column from $2$ to $N-1$, either it has $1$ black cell at the beginning or $N-1$ black cells at the beginning. So, each column from $2$ to $N-1$ need at least $1$ black cell. So the number of additional cells in this case is $2N-4$.
            We can easily check that this minimum works by add come black cells as below figure. We reach a completely black board by changing colour for the first $N-1$ column and then changes the colour of the first row.
            }{
            import MOgeom;
            fill((0,8)--(8,8)--(8,9)--(0,9)--cycle, gray); fill((8,8)--(9,8)--(9,0)--(8,0)--cycle, gray); for (int i=0; i<=9; ++i) { draw((0,i)--(9,i));draw((i,0)--(i,9)); }
        }

        Case 2. If there are at least $2$ black cells in column $1$, let says $x \ge 2$ black cells in column $1$. Similar to case 1, by considering the parity of $b_i$, we obtain each next column either have $x \ge 2$ black cells or $n-x \ge 2$ black cells. Hence, the minimum number of cells required is $2(N-2)+2=2N-2>2N-4$. So $2N-4$ is the better minimum.

        Thus, for $N \ge 3$, the answer is $2N-4$.
        }{%
        https://artofproblemsolving.com/community/c6t45241f6h1234383_flipping_rows_on_a_matrix_in_f2
    }

    \pitem[India TST 2016 Day 3 Problem 3]{%
        Let $n$ be an odd natural number. We consider an $n\times n$ grid which is made up of $n^2$ unit squares and $2n(n+1)$ edges. We colour each of these edges either red or blue. If there are at most $n^2$ red edges, then show that there exists a unit square at least three of whose edges are blue.
        }{%
        I will prove the equivalent result that if at least $n^2 + 2n$ edges are coloured blue in an $n$x$n$ grid, then some cell must have three blue edges. Let $n=2k+1$ since its odd.

        First, define the weight of a cell to be the number of blue edges it has. Suppose FTSOC that $n^2 + 2n$ edges can be coloured blue without any cell having three edges. Then, the sum of weights of all cells is at most $2n^2$ since there are $n^2$ cells and each has at most $2$ blue edges.

        Also, every blue edge contributes to the weights of at most $2$ cells and contributes to one cell only if it is a boundary edge. So, the sum of contributions to weights by border edges is at most $4n$ since there are $4n$ edges on the boundary.

        So, the remaining edges contribute at least $(n^2 + 2n-4n)(2) = 2n^2 - 4n$ to the weights and so the sum of contributions becomes exactly $2n^2$ and so equality holds everywhere in all the inequalities. This means that all the $4n$ edges on the boundary have to be coloured blue.

        Now, consider the inner $(n-2)$x$(n-2)$ grid. Since equality had held, it also means that every cell must have exactly $2$ blue edges. So, this means that all the edges on the boundary of this $(n-2)$x$(n-2)$ grid must be colored blue. Similarly, we can repeat this again for a $(n-4)$x$(n-4)$ grid and so on. But at the end, we will reach a $1$x$1$ grid which needs to have all of its border edges coloured blue. But, this is impossible since that would mean that this cell has $4$ blue edges.

        Therefore, it is not possible to have at most $n^2$ red edges and not have a cell with at least $3$ blue edges
        }{%
        https://artofproblemsolving.com/community/c6h1276390p6696375
    }
\end{question*}
\end{document}
