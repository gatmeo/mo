\pitem[USAMO 1999 Problem 5]{%
    The Y2K Game is played on a $1 \times 2000$ grid as follows. Two players in turn write either an S or an O in an empty square. The first player who produces three consecutive boxes that spell SOS wins. If all boxes are filled without producing SOS then the game is a draw. Prove that the second player has a winning strategy.
    }{%
    For brevity, let the first and second players be named A and B, respectively. Also, we call the three configurations of boxes (S, empty, S), or (S, O, empty), or (empty, O, S) winning configurations (since if any of these configurations are ever achieved, the next player to play can win).

    Define a group of four consecutive boxes to be deadly if the first and last contain the letter S, and the middle two boxes are empty. Notice that if any move is made in the two empty boxes of a deadly group, the next player to play can win: if an S is written, the (S, empty, S) configuration is achieved, and if an O is written, the (S, O, empty) or (empty, O, S) configuration is achieved.

    Lemma: B can always force the existence of a deadly group.

    Proof: If A writes an S on his first turn, then B can write an S three boxes away, producing a deadly group. If A writes an O on his first turn, then B can write an S somewhere at least $100$ tiles away from the O and from each end of the grid (the exact number is not important, only that it is far enough from the O that no winning configurations can be achieved next). Then, on B's second turn, he will either win immediately if it is possible, or he can place a second S three boxes away, opposite the direction of A's second move.

    After B produces a deadly group, he can proceed as follows until the only empty boxes remaining lie in deadly group. After each of A's moves, if it is not possible for B to win immediately, then B chooses an empty box $X$ that is not in a deadly group. If the neighboring boxes are both empty, B writes an O in box $X$. If a neighboring box contains an O, B writes an O in box $X$. If a neighboring box contains an S but neither contain O, B writes an S in box $X.$ These moves will not produce any winning configurations. (Specifically, in the third case, notice that writing an S cannot produce a (S, empty, S) configuration opposite the neighboring S, since that would mean that box $X$ is in a deadly group.)

    Eventually, the only squares left will lie in deadly groups. Since deadly groups have two empty boxes and the game began with an even number (2000) of empty boxes, this means that it will be A's turn when this occurs. Then A is forced to make a move in a deadly group, so B wins.
    }{%
    https://artofproblemsolving.com/community/c6h54505p340040
}

\pitem[IMOSL 2005 C5]{%
    There are $n$ markers, each with one side white and the other side black. In the beginning, these $n$ markers are aligned in a row so that their white sides are all up. In each step, if possible, we choose a marker whose white side is up (but not one of the outermost markers), remove it, and reverse the closest marker to the left of it and also reverse the closest marker to the right of it. Prove that, by a finite sequence of such steps, one can achieve a state with only two markers remaining if and only if $n - 1$ is not divisible by 3 .
    }{%
    Denote by $L$ the leftmost and by $R$ the rightmost marker. To start with, note that the parity of the number of black-side-up markers remains unchanged. Hence, if only two markers remain, these markers must have the same color up.

    We'll show by induction on $n$ that the game can be successfully finished if and only if $n\equiv 0\textnormal{ or }n\equiv 2(\textnormal{mod }3)$, and that the upper sides of L and R will be black in the first case and white in the second case.  The statement is clear for n 2 3. Assume that we finished the game for some n, and denote by k the position of the marker X (counting from the left) that was last removed. Having finished the game, we have also finished the subgames with the k markers from L to X and with the n k 1 markers from X to R (inclusive). Thereby, before X was removed, the upper side of L had been black if k 0 and white if k 2 (mod 3), while the upper side of R had been black if n k 1 0 and white if n k 1 2 (mod 3). Markers L and R were reversed upon the removal of X. Therefore, in the final position L and R are white if and only if k n k 1 0, which yields n 2 (mod 3), and black if and only if k n k 1 2, which yields n 0 (mod 3).

    On the other hand, a game with n markers can be reduced to a game with n 3 markers by removing the second, fourth, and third marker in this order. This finishes the induction.

    Second solution. An invariant can be defined as follows. To each white marker with k black markers to its left we assign the number 1 k . Let S be the sum of the assigned numbers. Then it is easy to verify that the remainder of S modulo 3 remains unchanged throughout the game: For example, when a white marker with two white neighbors and k black markers to its left is removed, S decreases
    by 3 1 t .
    }{%
    <++>
}

\pitem[IMOSL 2015 C4]{%
    Let n be a positive integer. Two players A and B play a game in which they take turns choosing positive integers $k\leq n$. The rules of the game are:
    \begin{enumerate}
        \item A player cannot choose a number that has been chosen by either player on any previous turn.
        \item  A player cannot choose a number consecutive to any of those the player has already chosen on any previous turn.
        \item  The game is a draw if all numbers have been chosen; otherwise the player who cannot choose a number anymore loses the game.
    \end{enumerate}
    The player A takes the first turn. Determine the outcome of the game, assuming that both players use optimal strategies.
    }{%
    }{%
    <++>
}

\pitem[IMOSL 2001 C7]{%
    A pile of $n$ pebbles is placed in a vertical column. This configuration is modified according to the following rules. A pebble can be moved if it is at the top of a column which contains at least two more pebbles than the column immediately to its right. (If there are no pebbles to the right, think of this as a column with 0 pebbles.) At each stage, choose a pebble from among those that can be moved (if there are any) and place it at the top of the column to its right. If no pebbles can be moved, the configuration is called a final configuration. For each $n$, show that, no matter what choices are made at each stage, the final configuration obtained is unique. Describe that configuration in terms of $n$.
    }{%
    <++>
    }{%
    <++>
}


\pitem[IOM 2020 P5]{%
    There is an empty table with $2^{100}$ rows and $100$ columns. Alice and Eva take turns filling the empty cells of the first row of the table, Alice plays first. In each move, Alice chooses an empty cell and puts a cross in it; Eva in each move chooses an empty cell and puts a zero. When no empty cells remain in the first row, the players move on to the second row, and so on (in each new row Alice plays first).
    The game ends when all the rows are filled. Alice wants to make as many different rows in the table as possible, while Eva wants to make as few as possible. How many different rows will be there in the table if both follow their best strategies?
    }{%
    <++>
    }{%
    https://artofproblemsolving.com/community/c6t224166f6h2384039_alice_amp_eva_take_turns_filling_an_empty_table_with_2100_rows_and_100_columns
}

\pitem[IMOSL 2012 C4]{%
    Players A and B play a game with $N\geq 2012$ coins and 2012 boxes arranged
    around a circle. Initially A distributes the coins among the boxes so that there
    is at least 1 coin in each box. Then the two of them make moves in the order
    B, A, B, A, … by the following rules:
    On every move of his B passes 1 coin from every box to an adjacent box.
    On every move of hers A chooses several coins that were not involved in B 's
    previous move and are in different boxes. She passes every chosen coin to
    an adjacent box.
    Player A 's goal is to ensure at least 1 coin in each box after every move of
    hers, regardless of how B plays and how many moves are made. Find the
    least N that enables her to succeed.
    }{%
    <++>
    }{%
    <++>
}

\pitem[IMOSL 2004 C5]{%
    A and B play a game, given an integer $N$, A writes down 1 first, then every player sees the last number written and if it is $n$ then in his turn he writes $n + 1$ or $2n$, but his number cannot be bigger than $N$. The player who writes $N$ wins. For which values of $N$ does B win?
    }{%
    <++>
    }{%
    <++>
}

