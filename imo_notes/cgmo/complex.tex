\documentclass[a4paper]{article}
\usepackage[math_simple,imo]{gatmeo}

\renewcommand{\courseTitle}{\FirstBigRestSmallCaps{CGMO (Complex Number)}}
\renewcommand{\courseTopic}{}
\DTMsavedate{mydate}{2023-07-26}

\toggletrue{ownans}
\togglefalse{officialans}
\rfoot{}

\begin{document}
\maketitle
\thispagestyle{empty}
%\tableofcontents


\begin{question*}[]{}
  \pitem[]{%
    Let $\zeta_n$ be the $n$-th root of unity. Let $m$ be an integer, prove that
    \[
      1+\zeta^m+(\zeta^2)^m+\cdots +(\zeta^{n-1})^m=
      \begin{cases}
        n,&n\mid m\\
        0,&n\not\mid m.
      \end{cases}
    \]
    }{%
    <++>
    }{%
    <++>
  }

    \pitem{%
      We toss a fair coin $n$ times. How many ways can we get number of heads with a multiple of 3? Or equivalently, show that
      \[
        \sum_{k\equiv 0\pmod3}^{}\binom{n}{k}=\frac{1}{3}(2^n+2\cos \left(\frac{n\pi}{3}\right)).
      \]
        }{%
        let $\zeta=e^{2\pi i}{3}$.
        $$f(1)=\binom{n}{n}+\binom{n}{n-1}+\cdots+\binom{n}{0}$$$$f(\zeta)=\binom{n}{n}(\zeta)^n+\binom{n}{n-1}(\zeta)^{n-1}+\cdots+\binom{n}{0}(\zeta)^0$$$$f(\zeta^2)=\binom{n}{n}(\zeta)^{2n}+\binom{n}{n-1}(\zeta)^{2n-2}+\cdots+\binom{n}{0}(\zeta)^0$$and when adding them together, we get $$f(1)+f(\zeta)+f(\zeta^2)$$$$=\binom{n}{n}(1^n+\zeta^n+\zeta^{2n})+\binom{n}{n-1}(1^{n-1}+\zeta^{n-1}+\zeta^{2n-2})+\cdots+\binom{n}{0}(1^0+\zeta^0+\zeta^0)$$$$=3(\binom{n}{3m}+\binom{n}{3m-3}+\cdots+\binom{n}{0})$$So our desired expression is $\frac{1}{3}(f(1)+f(\zeta)+f(\zeta^2))$.
        }{%
        https://artofproblemsolving.com/community/c4t29821f4h1846859_making_a_book_and_need_input
    }

    \pitem{%
        Find $\binom{n}{2}+\binom{n}{5}+\binom{n}{8}+\cdots$.
        }{%
        Note that if the coefficient of $x^2$ in $(\frac{1}{3})(f(1)+f(\zeta)+f(\zeta^2))$ is $\binom{n}{2}(1+\zeta^2+\zeta^4).$ The $1$ represents $f(1)$, the $\zeta^2$ represents the $f(\zeta)$, and the $\zeta^4$ represents $f(\zeta^2)$. We want to get all of these to equal $1$. By multiplying the $\zeta^2$ by $\zeta$, and the $\zeta^4$ by $\zeta^2$, we get the desired $1$'s. Therefore, to get the value of the expression, we find the value of $(\frac{1}{3})(f(1)+\zeta f(z)+\zeta^2 f(\zeta^2))$.
        }{%
        https://artofproblemsolving.com/community/c4t29821f4h1846859_making_a_book_and_need_input
    }

    \pitem[]{%
      Let $O$ be the circumcentre of $\triangle ABC$, $D$ be the midpoint of $AB$, $E$ be the centroid of $\triangle ACD$. If $AB=AC$, prove that $OE\perp CD$.
      }{%
      <++>
      }{%
      <++>
    }

  \pitem[]{%
    Let $a,b\in\mathbb{C}$ and $(\omega_q)_{0\leq q\leq n-1}$ be the $n^{th}$ complex roots of the unit. Show that
    \[
n\big(|a|+|b|\big)\le 2\sum_{q=0}^{n-1}|a+w_qb|.
    \]
    }{%
    Note that
\[ \sum_{q=0}^{n-1}|a+\omega_qb| \geq \left|\sum_{q=0}^{n-1}(a + \omega_q b)\right| = n|a| \]and
\[ \sum_{q=0}^{n-1}|a + \omega_q b| = \sum_{q=0}^{n-1}|a\omega_q^{-1} + b| \geq \left|\sum_{q=0}^{n-1}(a\omega_q^{-1} + b)\right| = n|b|. \]Adding both inequalities yields the desired result.
    }{%
    https://artofproblemsolving.com/community/c6h2975415p26670729
  }

  \pitem[]{%
Let $z_1,z_2,\cdots,z_n \in \mathbb{C}$, with $|z_1|+|z_2|+\cdots+|z_n|=1$. Prove that there must exist some of these n complex numbers, the modulus of whose sum $\geq \frac{1}{6}$.
    }{%
    sharpening:$\frac{1}{4}$
$z_k=a_k+b_ki$
$S_1=\sum_{a_k > 0}z_k \geq \sum_{a_k > 0}|a_k|$
$S_2=\sum_{a_k < 0}z_k \geq \sum_{a_k < 0}|a_k|$
$S_3=\sum_{b_k > 0}z_k \geq \sum_{b_k > 0}|b_k|$
$S_4=\sum_{b_k < 0}z_k \geq \sum_{b_k > 0}|b_k|$
$S_1+S_2+S_3+S_4 \geq \sum_{k=1}^n(|a_k|+|b_k|) \geq \sum_{k=1}^n|z_k|=1$
hence, there's $i$, $S_i \geq \frac{1}{4} \square $
    }{%
    https://artofproblemsolving.com/community/c1229496h2236541p17122700
  }

    \pitem{%
      Find the number of ordered pairs of real numbers $(a,b)$ such that $(a+bi)^{2002} = a-bi$. 
      }{%
      Let $s=\sqrt{a^2+b^2}$ be the magnitude of $a+bi$. Then the magnitude of $(a+bi)^{2002}$ is $s^{2002}$, while the magnitude of $a-bi$ is $s$. We get that $s^{2002}=s$, hence either $s=0$ or $s=1$.

      For $s=0$ we get a single solution $(a,b)=(0,0)$.

      Let's now assume that $s=1$. Multiply both sides by $a+bi$. The left hand side becomes $(a+bi)^{2003}$, the right hand side becomes $(a-bi)(a+bi)=a^2 + b^2 = 1$. Hence the solutions for this case are precisely all the $2003$rd complex roots of unity, and there are $2003$ of those.

      The total number of solutions is therefore $1+2003 = \boxed{2004}$. 
      }{%
      https://artofproblemsolving.com/wiki/index.php/2002_AMC_12A_Problems/Problem_24
    }

    \pitem[]{%
      Factorize $f(x)=x^{12}+x^9+x^6+x^3+1$ into product of two non-constant terms.
      }{%
      <++>
      }{%
      <++>
    }

  \pitem[]{%
    Let $\omega \in \mathbb{C}$, and $\left |  \omega  \right | = 1$. Find the maximum length of $z = \left( \omega + 2 \right) ^3 \left( \omega - 3 \right)^2$.
    }{%
    Consider a $\triangle ABC$ with $BC=5$ and a point $D$ on $BC$ such that $BD=2 \, , \, DC=3$. Let $AB=x \, , \, AC=y$. We are essentially asked given $AD=1$ find the maximum value of $x^3 y^2$. By Stewart's we get:
$$35=3x^2+2y^2$$And then by GM-QM we get:
$$\sqrt[5]{x^3 y^2} \leq \sqrt{\frac{3x^2+2y^2}{5}}=\sqrt{7} \Rightarrow x^3 y^2 \leq 7^{\frac{5}{2}}$$With equality when $x=y=\sqrt{7}$
    }{%
    https://artofproblemsolving.com/community/c6h1691219p10814478
  }

  \pitem[]{%
Let $\zeta$ be a n-th root of unity; n being an odd natural number.  
Evaluate
$$\frac{1}{1+1} + \frac{1}{1+\zeta^1} + ... + \frac{1}{1+\zeta^{n-1}}.$$
    }{%
    The answer is $\frac{n}{2}$

First consider the polynomial $\left(z-1\right)^n-1$. By considering it's roots we see that $\left(z-1\right)^n-1=\prod_{k=0}^{n-1}\left(z-\left(1+w^k\right)\right)$, and by considering the $z$ coefficient, we see that
$$n=\left(\prod_{k=0}^{n-1}\left(1+w^k\right)\right)\left(\frac{1}{1+1}+\frac{1}{1+w}+...\ \frac{1}{1+w^{n-1}}\right)$$
Considering the constant coefficients, we see that $\prod_{k=0}^{n-1}\left(1+w^k\right)=2$, and putting this into our first result, we get the answer $\frac{n}{2}$
    }{%
    https://artofproblemsolving.com/community/c6h1637203p10300418
  }

  \pitem[]{%
    Given a finite set $X$, two positive integers $n,k$, and a map $f:X\to X$. Define $f^{(1)}(x)=f(x),f^{(i+1)}(x)=f^{(i)}(x)$,$i=1,2,3,\ldots$. It is known that for any $x\in X$,$f^{(n)}(x)=x$.
    Define $m_j$ the number of $x\in X$ satisfying $f^{(j)}(x)=x$.
    Prove that:
    \begin{enumerate}
      \item $\frac{1}n \sum_{j=1}^n  m_j\sin {\frac{2kj\pi}{n}}=0$,
      \item $\frac{1}n \sum_{j=1}^n  m_j\cos {\frac{2kj\pi}{n}}$ is a non-negative integer.
    \end{enumerate}
    }{%
    Call the smallest integer $i$ satisfying $f^{(i)}(x)=x$ the length of $x$.
    The length of any $x$ must divide $n$. The number of $x$ with length $d$ is also divided by $d$.
    Define $dt_d$ the number of $x$ with length $d$ ($t_d\in\mathbb{N}$,$d=1,2,\ldots$).Then we have $m_k=\sum_{d|k} dt_d$.
    Pick any $\zeta$ satisfying $\zeta^n=1$,we only need to prove $\displaystyle\frac{1}n\sum_{k=1}^n m_k\zeta^k\in\mathbb{N}$.
    \begin{align} \frac{1}n\sum_{k=1}^n m_k\zeta^k&=\frac{1}n\sum_{k=1}^n (\sum_{d|k} dt_d)\zeta^k\\ &=\frac1n \sum_{d|k,1\le k\le n} dt_d\zeta^k\\ &=\frac1n \sum_{d|n}(\sum_{d|k,1\le k\le n} dt_d\zeta^k)\\ &=\sum_{d|n} t_d(\frac{d}n\sum_{j=1}^{n/d} \zeta^{dj})\in\mathbb{N} \end{align}(3) is true, because for $d\nmid n$,$t_d=0$. (4) is true, because for $d\mid n$,we have $\dfrac{d}n\sum_{j=1}^{n/d} \zeta^{dj}=0$ or $1$.
    This completes the solution
    }{%
    https://artofproblemsolving.com/community/c6h1495527p8798640
  }


\end{question*}



\end{document}

  %\pitem[]{%
  %  Construct a function $f:\mathbb{Q}^{>0}\rightarrow \mathbb{Q}^{>0}$ such that $f(x(f(y)))=\frac{f(x)}{y}$ for all $x,y\in \mathbb{Q}^{>0}$.
  %  }{%
  %  <++>
  %  }{%
  %  <++>
  %}
  
