\documentclass[a4paper]{article}
\usepackage[math_simple,imo]{gatmeo}

\renewcommand{\courseTitle}{\FirstBigRestSmallCaps{CGMO 2}}
\DTMsavedate{mydate}{2023-07-26}

\toggletrue{ownans}
\togglefalse{officialans}
\rfoot{}

\begin{document}
\maketitle
\thispagestyle{empty}
%\tableofcontents
\begin{question*}[c]{}
  \pitem[]{%
Let $(a_n)_{n\geq 1}$ be a sequence of positive real numbers with the property that
$$(a_{n+1})^2 + a_na_{n+2} \leq a_n + a_{n+2}$$for all positive integers $n$. Show that $a_{2022}\leq 1$.
    }{%
    It will be helpful to rewrite the inequality as
$$(1-a_n)(1-a_{n+2}) \leq 1-a_{n+1}^2$$Assume $a_{2022}>1$. We first claim that

Claim: We have either

    $a_{2020} < 1$, $a_{2021} > 1$, $a_{2022} > 1$, $a_{2023} < 1$, or
    $a_{2021} < 1$, $a_{2022} > 1$, $a_{2023} > 1$, $a_{2024} < 1$.

Proof. We have $(1-a_{2021})(1-a_{2023}) < 0$, so we have two cases:

    If $a_{2021} < 1$ and $a_{2023}>1$, we note $(1-a_{2022})(1-a_{2024}) < 0$, implying that $a_{2024}<1$.
    If $a_{2021} > 1$ and $a_{2023}<1$, we note $(1-a_{2020})(1-a_{2022}) < 0$, implying that $a_{2020} < 1$. $\blacksquare$


Then, we prove the following, which, together with the first claim, will obviously finish the problem.

Claim: There does not exists a positive integer $n$ such that
$$a_n < 1, a_{n+1} > 1, a_{n+2}>1, a_{n+3}<1$$Proof. We have
\begin{align*} 1-a_{n+2} < (1-a_n)(1-a_{n+2}) \leq 1-a_{n+1}^2 &\implies a_{n+1}^2 < a_{n+2} \\ 1-a_{n+1} < (1-a_{n+1})(1-a_{n+3}) \leq 1-a_{n+2}^2 &\implies a_{n+2}^2 < a_{n+1}, \end{align*}so we have $a_{n+1}^4 < a_{n+2}^2 < a_{n+1}$, forcing $a_{n+1}<1$, a contradiction. $\blacksquare$
    }{%
    https://artofproblemsolving.com/community/c6t169f6h3107323_weird_sequence
  }

  \pitem{%
    Let $ABC$ be an acute triangle with $AB< AC$. Denote by $P$ and $Q$ points on the segment $BC$ such that $\angle BAP = \angle CAQ < \frac{\angle BAC}{2}$. $B_1$ is a point on segment $AC$. $BB_1$ intersects $AP$ and $AQ$ at $P_1$ and $Q_1$, respectively. The angle bisectors of $\angle BAC$ and $\angle CBB_1$ intersect at $M$. If $PQ_1\perp AC$ and $QP_1\perp AB$, prove that $AQ_1MPB$ is cyclic.
    }{%
    We will now reconstruct the problem:
    $ABC$ is an acute triangle with $AB<AC.$ $P,Q$ lie on $BC$ such that $AP$ and $AQ$ are isogonal. $P_1$ and $Q_1$ lie on $AP$ and $AQ$ respectively such that $P_1Q$ is perpendicular to $AB$ and $Q_1P$ is perpendicular to $AC.$ If $BP_1Q_1$ is collinear, prove that $AQ_1PB$ is cyclic.

    Note that this is equivalent, as $M$ lying on the circle would hold by it being the midpoint of the arc $PQ_1.$

    Now we prove that the condition is also equivalent to $AP$ being perpendicular to $BC.$

    $PP_1Q_1Q$ is cyclic. This is since $\measuredangle P_1Q_1Q=\measuredangle BQ_1A=\measuredangle QPP_1.$ (in fact this is true without the collinear condition, you can try to prove this)

    Now angle chase. $\measuredangle P_1QP=\measuredangle P_1QP=\measuredangle BAP=\measuredangle QPP_1-\measuredangle CBA.$ However, $\measuredangle P_1QP=90^{\circ}-\measuredangle CBA,$ thus we must have $P$ be the foot of altitude.

    Now, we want to prove $\angle AQ_1B=90^{\circ},$ but this is easy because $PQQ_1P_1$ is cyclic.
    }{%
    https://artofproblemsolving.com/community/c6t1535812f6h2949021_bisectors_perpendicularity_and_circles
  }


  \pitem[]{%
    There are $n \geq 2$ coins numbered from $1$ to $n$. These coins are placed around a circle, not necesarily in order.

    In each turn, if we are on the coin numbered $i$, we will jump to the one $i$ places from it, always in a clockwise order, beginning with coin number 1. For an example, see the figure below.

    Find all values of $n$ for which there exists an arrangement of the coins in which every coin will be visited.
    }{%
    Clearly the last coin visited is numbered $n$, after which the moves clearly consist on staying in place indefinitely. The sum of total moves (regardless of their order) until we reach the coin numbered $n$ is $1+2+\cdots+(n-1)=\frac{(n-1)n}{2}$.

If $n$ is odd, this is a multiple of $n$, ie we have made $\frac{n-1}{2}$ full turns to the arrangement. Or the starting coin is numbered both $n$ and $1$, contradiction. Thus no such arrangement exists for odd $n$.

If $n$ is even, we can place the coins such that we visit them in order $1,p-2,3,p-4,5,p-6,\dots$. Using cyclic notation and assuming that coin numbered $1$ is located at position $0$, this means that coin $n-2k$ is located at position $k$ for $k=1,2,\dots,\frac{n}{2}-1$ and coin $2k+1$ is located at position $-k$ for $k=0,1,\dots,\frac{n}{2}-1$. Setting coin numbered $n$ in position $\frac{n}{2}$, clearly all $n$ positions in the coin arrangement are assigned a coin with a distinct number, and hence all $n$ numbered coins are assigned to a different position. By trivial induction we can show that after $2k$ moves we are at position $-k$ on coin numbered $2k+1$, and after $2k+1$ moves we are at position $k+1$ on coin numbered $n-2k-2$, and we are done.
    }{%
    https://artofproblemsolving.com/community/c6t1535812f6h2687946_coins_in_a_circle
  }

  \pitem[]{%
Let $n$ be a positive integer with at least $4$ positive divisors. Let $d(n)$ be the number of positive divisors of $n$. Find all values of $n$ for which there exists a sequence of $d(n) - 1$ positive integers $a_1$, $a_2$, $\dots$, $a_{d(n)-1}$ that forms an arithmetic sequence and satisfies the following condition: for any integers $i$ and $j$ with $1 \leq i < j \leq d(n) - 1$, we have $\gcd(a_i , n) \neq \gcd(a_j , n)$.
    }{%
    The answer is $n = \boxed{8}$, $n = \boxed{12}$, and $n=\boxed{pq}$ for distinct primes $p,q$. I will use $\tau$ instead of $d$ since I am more used to it.

To see that these work, note that $n = 8$ works since we can take the arithmetic sequence $(a_1, a_2, a_3) = (4, 6, 8)$ (note $3 = \tau(8)-1$), and $\gcd(4, 8) = 4$, $\gcd(6, 8) = 2$, $\gcd(8, 8) = 8$ which are all pairwise distinct, and $n = 12$ works since we can take the arithmetic sequence $(a_1, a_2, a_3, a_4, a_5) = (6, 7, 8, 9, 10)$ (note $5 = \tau(12) - 1$), and $\gcd(6, 12) = 6$, $\gcd(7, 12) = 1$, $\gcd(8, 12) = 4$, $\gcd(9, 12) = 3$, $\gcd(10, 12) = 2$, which are all pairwise distinct. To see that $n = pq$ works for distinct primes $p, q$, WLOG suppose that $p < q$, then $q\neq 2$. Then, take $(a_1, a_2, a_3) = (1, ap, 2ap-1)$ for some positive integer $a$ such that $a\equiv \frac{1}{2p}\pmod{q}$ (note that $3 = \tau(pq)-1$), which is well defined since $q > p\geq 2$. This works since $\gcd(1, pq) = 1$, $\gcd(ap, pq) = p$ ($q\nmid ap$ since $ap\equiv \frac{1}{2}\pmod{q}$), and $q\mid 2ap-1$, so $\gcd(2ap-1, pq)\in \{q, pq\}$, so all the $\gcd$s are pairwise distinct.

Now, we will show that this is these are the only such $n$ that work. Note that since all of the $\gcd(a_i, n)$'s are distinct, and there are $\tau(n)-1$ of them, and they all divide $n$, we must have $$\{\gcd(a_i, n) \mid 1\leq i\leq \tau(n)-1\} = \{d \mid d\mid n \} \setminus \{x\},$$for some $x\mid n$.

First, we will rule out some special cases.

Special Case 1: $n$ is a power of a prime.
Then, $n = p^r$ for some $r\geq 3$ (as $\tau(n) \geq 4$). First, assume for the sake of contradiction that $r\geq 4$. If we let $b_i = \gcd(i, n) = \gcd(i, p^r)$, note that we must leave out $1$ in the $b_i$, else all of the $b_i$ besides one of them are divisible by $p$, so it follows that all of the $a_i$ besides one of them are divisible by $p$. However, since $\tau(n)-1 = r \geq 4$, it follows that there must exist two consecutive $a_i$ divisible by $p$. However, that would mean that their common difference is divisible by $p$, so all of the terms would be divisible by $p$, contradiction. Thus, $\{b_i\}_\{1\leq i\leq r\} = \{p, p^2, \cdots, p^r\}$. However, now we can apply the same argument to divisibility by $p^2$ -- $p^2$ dividing some $a_i$ is equivalent to $p^2$ dividing some $b_i$ since $r > 2$, so since all the $b_i$ besides one of them is divisible by $p^2$, all of the $a_i$ besides one of them is divisible by $p^2$, but since there are $r\geq 4$ terms, there exist two consecutive $a_i$ divisible by $p^2$, so their common difference is divisible by $p^2$, so all of the $a_i$ are divisible by $p^2$, contradiction. Thus, we must have that $r = 3$. Now, we will show that $p = 2$. Assume the contrary. Then, note that at least two of the $b_i$ must be divisible by $p$, so at least two of the $a_i$ are divisible by $p$. If they are adjacent, then by the previous argument all of the $a_i$ are divisible by $p$. Else, we must have that $a_1$ and $a_3$ are divisible by $p$, so since $p\neq 2$, $a_2 = \frac{a_1+a_3}{2}$ must also be divisible by $p$. Thus, all of the $a_i$ are divisible by $p$, so all of the $b_i$ are divisible by $p$, implying that $\{b_i\}_\{1\leq i\leq 3\} = \{p, p^2, p^3\}$. Then, two of the $b_i$ are divisible by $p^2$, so two of the $a_i$ are divisible by $p^2$, so just repeat the previous argument to get a contradiction. Thus, we have that $p = 2$ and $r = 3$, so $n = 2^3 = 8$ is the only prime power with at least $4$ divisors that works. $\square$

Special Case 2: $\tau(n) \leq 6$, $n$ is not a prime power.
Then, if $\tau(n) = 4$, we have that $n = pq$ for primes $p\neq q$, which we already showed works. If $\tau(n) = 5$, $n$ must be a prime power. If $\tau(n) = 6$, then the only non-prime such $n$ that can work are $n = pq^2$ for primes $p\neq q$. To see that this implies that $(p, q) = (3, 2)$, assume for the sake of contradiction that $q> 2$. Note that $4$ of the divisors of $n$ are divisible by $q$, so there are at least three $b_i$'s divisible by $q$, so at least three $a_i$'s divisible by $q$. Since $\tau(n)-1 = 5$, it follows that there either exist two adjacent $a_i$'s divisible by $q$ or two $a_i$'s that are two apart (i.e. $a_j$ and $a_{j+2}$) divisible by $q$. If there exist two adjacent $a_i$'s divisible by $q$, then all the $a_i$'s are divisible by $q$ as discussed earlier several times, so all of the $5$ $b_i$'s are divisible by $q$, however, at most $4$ are divisible by $q$ since there are only $4$ divisiors of $n$ that are divisible by $q$, contradiction. Else, if there exists two $a_i$'s that are two apart divisible by $q$, then since $q\neq 2$, their average must be divisible by $q$, so we get two adjacent $a_i$'s divisible by $q$ (one of the original $a_i$'s and their average; i.e. $a_{i+1}$), which is a contradiction from the above case. Thus, we must have $q = 2$. Now, assume for the sake of contradiction that $p > 3$. Then, $3$ of the divisors of $n$ are divisible by $p$, so there are at least $2$ $b_i$'s divisible by $p$. Then, there are at least $2$ $a_i$'s divisible by $p$, say $a_i$ and $a_j$ for $1\leq i < j\leq 5$. If $j = i+1$, then by previous logic we can see that all the $a_i$'s are divisible by $p$, so all of the $5$ $b_i$'s are divisible by $p$, contradiction since there are only $3$ divisors of $n$ divisible by $p$. If $j = i+2$, then $a_{i+1} = \frac{a_i + a_{i+2}}{2}$ must be divisible by $p$ since $p > 3$, so $a_i, a_{i+1}$ divisible by $p$, contradiction as mentioned above. If $j = i+3$, then $a_{i+1} = \frac{2}{3}a_i + \frac{1}{3}a_{i+3}$ must be divisible by $p$ since $p > 3$, so $a_i, a_{i+1}$ divisible by $p$, contradiction. If $j = i+4$, then $a_{i+1} = \frac{3}{4}a_i + \frac{1}{4}a_{i+4}$ must be divisible by $p$ since $p > 3$, so $a_i$, $a_{i+1}$ divisible by $p$, contradiction. Thus, $p \leq 3$, so since $q = 2$, $p = 3$, and so $n=pq^2 = 12$. $\square$

Now, suppose that $n$ was not a prime power and $\tau(n) > 6$. Now, assume for the sake of contradiction was not squarefree, i.e. there existed a prime $p$ dividing $n$ with $\nu_p(n) \geq 2$. Then, note that $\frac{\nu_p(n)}{\nu_p(n)+1}\tau(n)-1 \geq \frac{2}{3}\tau(n)-1$ of the $b_i$'s must be divisible by $p$, so at least $\frac{2}{3}\tau(n)$ of the $a_i$'s must be divisible by $p$. Not all the $a_i$'s can be divisible by $p$, since then all the $b_i$'s would be divisible by $p$, so at least $\tau(n)-1$ divisors of $n$ must be divisible by $p$, but since $n$ is not a prime power, there exists a prime $q\neq p$ dividing $n$, and $1, q$ are distinct divisors not divisible by $p$, contradiction. Thus, there exists a $j$ for which $p\nmid a_j$. Now, note that arithmetic sequences repeat modulo $p$, so if we take the largest $k$ for which $\{a_1, a_2, \cdots, a_k\}$ all have distinct residues mod $p$ (this implies that $a_{k+1} = a_1$ since if $a_{k+1} = a_j$ for $j > 1$, then $a_{k-j+2} = a_1$, contradiction since $k-j+2\leq k$), then there is at most one $0\pmod{p}$ in $\{a_1, a_2, \cdots, a_k\}$ and at least one nonzero element mod $p$ since we know there exists a nonzero element in this arithmetic progression (and everything is contained in $\{a_1, a_2, \cdots, a_k\}$ since we know $a_{k+1} \equiv a_1\pmod{p}$ and so it repeats). If $k = 1$, then there are no elements divisible by $p$ due to the existence of a nonzero element mod $p$, and if $k\geq 2$, then at most $1$ in each block, so at most $\left \lceil \frac{\tau(n)-1}{k}\right \rceil \leq \left \lceil \frac{\tau(n)-1}{2}\right \rceil$ in total. Thus, we have that since at least $\frac{2}{3}\tau(n)-1$ of the $a_i$ must be divisible by $p$, we have that $$\left \lceil \frac{\tau(n)-1}{2}\right \rceil \geq \frac{2}{3}\tau(n) - 1,$$so since the ceiling will add at most $\frac{1}{2}$ to the value, $$\frac{\tau(n)}{2} \geq \frac{2}{3}\tau(n) - 1,$$so $$1 \geq \frac{1}{6}\tau(n),$$so $$\tau(n) \leq 6,$$a contradiction since $\tau(n) > 6$.

Now, suppose that $n$ was squarefree (and $\tau(n) > 6$, and $n$ is not a prime power). Then, $n = p_1p_2\cdots p_k$ for some $k\geq 3$. Then, note that for any prime $p_i$, exactly $2^{k-1}$ such divisors of $n$ are divisible by $p_i$. Therefore, the number of $b_j$ divisible by some $p_i$ is in $\{2^{k-1} - 1, 2^{k-1}\}$, so the number of $a_j$ divisible by some $p_i$ is in $\{2^{k-1}-1, 2^{k-1}\}$. Note that $2^{k-1} < 2^k-1$, so not all the $a_j$'s are divisible by $p_i$. Now, as done in the above case, take the largest $r$ for which $\{a_1, a_2, \cdots, a_r\}$ all produce distinct residues mod $p_i$; then $a_{r+1}\equiv a_1\pmod{p_i}$ and it repeats as proven earlier. Then, assume for the sake of contradiction that $r\neq 2$. If we had $r = 1$, then since there exists a nonzero residue mod $p_i$ in the $a_j$'s, it follows that all residues are nonzero, but since $2^{k-1} - 1 > 0$, we have a contradiction. If we had $r \geq 3$, then at most $1$ in each block of $r$ is $0\pmod{p_i}$, so the number of $a_i$ divisible by $p_i$ is at most $$\left \lceil \frac{2^k-1}{r}\right \rceil \leq \left \lceil \frac{2^k-1}{3}\right \rceil \leq \frac{2^k-1}{3} + 1 = \frac{2^k+2}{3},$$so $$\frac{2^k+2}{3} \geq 2^{k-1} = \frac{2^k}{2},$$implying $$2\cdot 2^k + 4\geq 3\cdot 2^k,$$so $$2^k \leq 4,$$a contradiction. Thus, we must have $r = 2$. Since $2^{k-1} - 1 > 0$, it follows that exactly one of $\{a_1, a_2\}$ are divisible by $p_i$, so we have that in general, $a_j$ is divisible by $p_i$ iff $j$ is a certain parity. Since $k\geq 3$ by PHP there exist two $p_i$ (WLOG suppose they are $p_1, p_2$) with the same such parity which makes $a_j \mid p_i$. Then, $p_1\mid a_i\iff p_2 \mid a_i$, so there are no such $a_i$ divisible by $p_1$ but not $p_2$. Then, there are no such $b_i$ divisible by $p_1$ but not $p_2$. However, there are $2^{k-2} \geq 2$ such divisors of $n$ divisible by $p_1$ but not $p_2$, so at least one $b_i$ divisible by $p_1$ but not $p_2$, a contradiction.

Thus, we've exhausted all cases and so the only $n$ which work are $n = 8, n=12$, and $n=pq$ for distinct primes $p\neq q$, as claimed. $\blacksquare$
    }{%
    https://artofproblemsolving.com/community/c6t177f6h3116579_gcd_and_arithmetic_sequence
  }

\end{question*}
\end{document}
