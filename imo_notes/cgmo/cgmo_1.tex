\documentclass[a4paper]{article}
\usepackage[math_simple,imo]{gatmeo}

\renewcommand{\courseTitle}{\FirstBigRestSmallCaps{CGMO}}
\renewcommand{\courseTopic}{\FirstBigRestSmallCaps{CGMO 1}}
\DTMsavedate{mydate}{2023-07-24}

\toggletrue{ownans}
\togglefalse{officialans}
\rfoot{}

\begin{document}
\maketitle
\thispagestyle{empty}
%\tableofcontents


\begin{question}[]{}
  \pitem[A]{%
    Given a positive integer $n$, determine the maximal constant $C_n$ satisfying the following condition: for any partition of the set $\{1,2,\ldots,2n \}$ into two $n$-element subsets $A$ and $B$, there exist labellings $a_1,a_2,\ldots,a_n$ and $b_1,b_2,\ldots,b_n$ of $A$ and $B$, respectively, such that
$$ (a_1-b_1)^2+(a_2-b_2)^2+\ldots+(a_n-b_n)^2\ge C_n. $$
    }{%
    We're trying to maximize $$C_n=\sum_{i=1}^{n}(a_i-b_i)^2=\sum_{i=1}^{2n}i^2-2\sum_{i=1}^{n}a_ib_i=\frac{2n(2n+1)(4n+1)}{6}-2\sum_{i=1}^{n}a_ib_i$$, so we need to minimize $S:=\sum_{i=1}^{n}a_ib_i$.

Lemma. The minimum of $S$ is achieved when none of the pairs $(a_i,b_i)$ are both greater than $n$.

Proof. Assume not, then there are four numbers $i,j,k,l \in \{1,2,3,\dots,n\}$ such that $(n+i)(n+j)+lk$ appears in $S$, but we can replace these numbers with $(n+i)l+(n+j)k$ and decrease $S$ because $(n+i)(n+j-l)\geq k(n+j-l)$. $\blacksquare$

So, we can let $A=\{1,2,\dots ,n\}$ and $B=\{n+1,n+2,\dots ,2n\}$ and by rearrangement inequality we get that $$S\geq \sum_{i=1}^{n} i(2n+1-i)=\frac{n(n+1)(2n+1)}{3}$$.
Hence, we have $C_n=\frac{2n(2n+1)(4n+1)}{6}-2S\leq \frac{n(2n+1)(2n-1)}{3}$.
    }{%
    https://artofproblemsolving.com/community/c6t169f6h1908505_inequality_with_permutations
  }

  \pitem[]{%
There are 20 people at a party. Each person holds some number of coins. Every minute, each person who has at least 19 coins simultaneously gives one coin to every other person at the party. (So, it is possible that $A$ gives $B$ a coin and $B$ gives $A$ a coin at the same time.) Suppose that this process continues indefinitely. That is, for any positive integer $n$, there exists a person who will give away coins during the $n$th minute. What is the smallest number of coins that could be at the party?
    }{%
    the answer is $190$ which can be archieved by giving $0,1,2,...,19$ coins to each person initially. It suffices to prove that less than $190$ coins are not possible.

    Claim : If $x_i$ is the number of coins that the $i$-th person have, then $\sum_{1\leqslant i < j\leqslant 20}(x_i-x_j)^2$ is always decreasing.

    Proof : WLOG the move is $(x_1,x_2,...,x_{20})\to (x_1-19,x_2+1,x_3+1,...,x_{20}+1)$. Then the resultant change is
    $$\sum_{i=2}^{20}(x_1-x_i-20)^2 - (x_1-x_i)^2 = \sum_{i=2}^{20} 400 - 40(x_1-x_i) = 40(190 - 20x_1 + (x_1+x_2+x_3+...+x_{20}))$$which is negative as $x_1+x_2+...+x_{20} < 190$ and $x_1 \geqslant 19$.

    Since the sum is always non-negative, the process cannot go indefinitely, contradiction.
    }{%
    https://artofproblemsolving.com/community/c6t309f6h545081_coin_operation
  }

  \pitem[]{%
    Let $ABC$ be a triangle with incenter $I$, and $A$-excenter $\Gamma$. Let $A_1,B_1,C_1$ be the points of tangency of $\Gamma$ with $BC,AC$ and $AB$, respectively. Suppose $IA_1, IB_1$ and $IC_1$ intersect $\Gamma$ for the second time at points $A_2,B_2,C_2$, respectively. $M$ is the midpoint of segment $AA_1$. If the intersection of $A_1B_1$ and $A_2B_2$ is $X$, and the intersection of $A_1C_1$ and $A_2C_2$ is $Y$, prove that $MX=MY$.
    }{%
Lemma: $X$ is the midpoint in $A_1B_1$. Similarly, $Y$ is the midpoint in $A_1C_1$.

Proof: Since $I$ is inside $ABC$ and lines $AB,AC$ are tangent to $\Gamma$, $A_1$ is inside segment $IA_2$ and $B_2$ is inside segment $IB_1$. By Menelaus' theorem applied to $A_2,X,B_2$ on the sides of triangle $IA_1B_1$, and using that $IA_1\cdot IA_2=IB_1\cdot IB_2$ is the power of $I$ with respect to $\Gamma$, the Lemma is equivalent to $IB_1\cdot B_1B_2=IA_1\cdot A_1A_2$. Note that the power of $I$ with respect to $\Gamma$ also equals $II_A^2-\rho^2$, where $\rho=I_AA_1$ is the radius of $\Gamma$. Or, denoting by $\alpha$ the angle between $IA_1$ and $BC$, we have $\angle IA_1I_A=90^\circ+\alpha$, and using the Cosine Law we obtain
$$IA_1\cdot A_1A_2=IA_1\cdot IA_2-IA_1^2=II_A^2-I_AA_1^2-IA_1^2=-2I_AA_1\cdot IA_1\cos\angle IA_1I_A=2\rho\cdot IA_1\sin\alpha=2\rho\cdot r.$$Similarly, $IB_1\cdot B_1B_2=2\rho\cdot r=IA_1\cdot A_1A_2$. The first part of the Lemma follows, and the second one is proved analogously, exchanging $B$ and $C$.

Since by the Lemma $X$ is the midpoint in $A_1B_1$, $Y$ is the midpoint in $A_1C_1$ and by definition $M$ is the midpoint of $A_1A$, triangle $XYM$ is similar to triangle $B_1C_1A$. Or since $AB_1C_1$ is isosceles at $A$, $MXY$ is isosceles at $M$. The conclusion follows.
    }{%
    https://artofproblemsolving.com/community/c6t48f6h2687957_incenter_excenter_and_intersections
  }

  \pitem[]{%
Prove that the system \begin{align*} x^6+x^3+x^3y+y & = 147^{157} \\ x^3+x^3y+y^2+y+z^9 & = 157^{147} \end{align*} has no solutions in integers $x$, $y$, and $z$.
    }{%
    We want to mod 19 since then $z^9$ will congruent to $0,1,-1$.

    First, add and subtract the equations to get
    $(x^3+y+1)^2-1+z^9=147^{157}-157^{147}$

    $(x^3-y)(x^3+y)-z^9=147^{157}-157^{147}$

    Now, assume that $x,y,z$ are a solution and work mod 19. $147^{157}\equiv2$ and $157^{147}\equiv11,$ so
    $(x^3+y+1)^2+z^9\equiv14$,
    Since 9th powers are congruent to $\pm1$, either $(x^3+y+1)^2\equiv13$ or $(x^3+y+1)^2\equiv15$. Neither of these values is a square mod 19, so there is no solution and we are done.
    }{%
    https://artofproblemsolving.com/community/c6t169f6h34315_weird_looking_system_of_equations
  }
\end{question}



\end{document}
