\documentclass{article}
\usepackage[tut]{omegat}

\title{Combinatorics: Arranging Identical Elements}
\author{Omega Tong}
\pagestyle{fancy}
\fancyhf{}
\cfoot{\thepage}
\fancyhead[C]{Name:\qquad\qquad \qquad\qquad\qquad\phantom{()}\\
Class:\qquad \qquad \qquad \qquad (\qquad )}
\fancyhead[R]{Score: \qquad/44\\
Time:\quad\ 2 hours}
\fancyhead[L]{S.3,4 Term Test 2\\9/5/2023}
\setlength{\headheight}{25pt}
\date{\today}
\def\myFBRSC#1{{\larger[0]{\MakeTextUppercase{#1}}}\scshape }
\DeclareRobustCommand{\FirstBigRestSmallCaps}[1]{\myFBRSC #1}
%\toggletrue{ownans}

\begin{document}

\begin{center}
    \fbox{\fbox{\parbox{5.5in}{\centering
    Answer the questions in the spaces provided. Box your numerical answers under the question number. For the questions where proof is required, write it out clearly. You are advised to write your sketches of proof for questions that do not require proof for potential partial score if you got it wrong.}}}
\end{center}

\nbf{Questions:}
\begin{shortque*}{3,6,9}

    (2pt)\\ \input{../content/n/modular_arithmetic/cal_inverse_ma-q02.tex}

    (3pt)\\
    \qitem{%
Katie and Allie are playing a game. Katie rolls two fair six-sided dice and Allie flips two fair two-sided coins. Katie’s score is equal to the sum of the numbers on the top of the dice. Allie’s score is the product of the values of two coins, where heads is worth $4$ and tails is worth $2.$ What is the probability Katie’s score is strictly greater than Allie’s?
        }{%
        We proceed by casework. When Allie gets two heads, there is no way for Katie to get a strictly greater score than Allie. When Allie gets first head then second tails, there are $10$ possible ways for Katie to get a strictly greater score than Allie. There are also $10$ possible ways when Allie gets tails first and heads second. When Allie gets both tails, there are $30$ possible ways for Katie to get a strictly greater score than Allie. There are $(6 \cdot 6) \cdot (2 \cdot 2) = 144$ total possible outcomes. Hence, the answer is $\frac{10+10+30}{36 \cdot 4} = \boxed{\frac{25}{72}}.$
        }{%
        https://artofproblemsolving.com/community/c4t177f4h3066888_bmt_2022_fall_discrete_p3
    }

    (3pt)\\
    \qitem{%
Let $BB'$ and $CC'$ be the feet of the altitudes from $B$ and $C$ onto the opposite sides of $\triangle ABC$. Let $H$ be the orthocenter and $D$ be the point of intersection of $B'C'$ and $BC$. If $M$ is the midpoint of $BC$, prove that $DH$ is perpendicular to $AM$.
        }{%
Let $P$ be the intersection of $\overline{DH}$ and $\overline{AM}$. Angle chase
\begin{align*} \angle PAB' & = \angle AMD - \angle ACB \\ & = (90^{\circ} - \angle PDM) - \angle ACB \\ & = 90^{\circ} - (\angle HBC - \angle BHD) - \angle ACB \\ & = ((90^{\circ} - \angle ACB) - \angle HBC) + \angle BHD = \angle BHD \\ & = \angle PHB', \end{align*}so $AHPB'$ is cyclic and $\angle HPA = \angle PB'A = 90^{\circ}$ as desired. $\blacksquare$
        }{%
        https://artofproblemsolving.com/community/c4t48f4h3065092_geometry
    }

    (4pt)\\
    \qitem{%
        Five men and nine women stand equally spaced around a circle in random order. The probability that every man stands diametrically opposite a woman is $\frac{m}{n},$ where $m$ and $n$ are relatively prime positive integers. Find $m+n.$
        }{%
        $\frac{(14\cdot 12\cdot 10\cdot 8\cdot 6)/5!}{\binom{14}{5}}=\frac{48}{143}\implies \boxed{191}$
        }{%
        https://artofproblemsolving.com/community/c5h3011229p27048738
    }

    (4pt)\\
    \qitem{%
The sum of all positive integers $m$ for which $\tfrac{13!}{m}$ is a perfect square can be written as $2^{a}3^{b}5^{c}7^{d}11^{e}13^{f}$, where $a, b, c, d, e,$ and $f$ are positive integers. Find $a+b+c+d+e+f$.
        }{%
        First note $13! = 2^{10} \cdot  3^{5} \cdot 5^2 \cdot 7\cdot 11\cdot 13$.

This implies that $m = 2^{a_1} \cdot 3^{a_2}\cdot 5^{a_3} \cdot 7\cdot 11\cdot 13$ for nonnegative integers $a_1, a_2, a_3$.

We must have $a_1$ is even and $\le 10$, $a_2$ is odd and $\le 5$, $a_3$ is even and $\le 2$.

We get that the sum of all $m$ is \begin{align*} (2^0 + 2^2 + 2^4 + 2^6 + 2^8 + 2^10)(3^1 + 3^3 + 3^5)(5^0 + 5^2)\cdot 7\cdot 11\cdot 13 \\ = 1365\cdot 273 \cdot 26\cdot 7\cdot 11\cdot 13 \\ = 3\cdot 5\cdot 7\cdot 13\cdot 3\cdot 7\cdot 13 \cdot 2\cdot 13\cdot 7\cdot 11\cdot 13 \\ \end{align*}
You can count that there are $\boxed{012}$ (not necessarily distinct) primes here, so we are done.
        }{%
        https://artofproblemsolving.com/community/c5h3011237p27048762
    }

    (4pt)\\ \qitem{%
    Suppose that $N$ can be written in base $6$ as $531340_6$ and in base $8$ as $124154_8$. In base $10$, what is the remainder when $N$ is divided by $210$?
    }{%
    We look at the $0$ at the end of the first number base $6$, which means that $n \equiv 0 \pmod{6}$. Now, we need to find $n \pmod{7}$ and $n \equiv 5 \pmod{7}$. $6 \equiv 1 \pmod{5}$, so
$$n \equiv 5+3+1+3+4+0 \equiv 1 \pmod{5}.$$Also, $6 \equiv -1 \pmod{7}$, so
$$n \equiv -5+3-1+3-4+0 \equiv 3 \pmod{7}.$$Let $n = 6k$. Then $6k \equiv k \equiv 1 \pmod{5}$, so $k = 5a+1$ and $n = 6(5a+1) = 30a + 6$.
$$30a + 6 \equiv 3 \pmod{7} \Rightarrow 2a \equiv 4 \pmod{7} \Rightarrow a \equiv 2 \pmod{7}.$$$a = 7b+2$, so $n = 30(7b+2) + 6 = 210b + 66$, so the remainder is $\boxed{66}$.
    }{%
    https://artofproblemsolving.com/community/c4t30973f4h1482074_chinese_remainder_theorem
}

    (4pt)\\ \qitem{%
    Let $f(n)=n^{99}+2n^{98}+3n^{97}+\cdots +99n+100$. Find $f(2)$ mod 100?
    }{%
    It must be $2 \mod 4$ because all terms but $99 \cdot 2 \equiv 2$ are $0 \mod 4$. Now we evaluate it $\mod 25$.

We use the standard formula $\sum_{k=1}^m kr^k = \frac{r(1-(m+1)r^m+mr^{m+1}}{(1-r)^2}$ for $r \neq 1$ (this result is useful in many other problems as well). A proof's provided below; there's a much shorter proof by differentiating the sum of a geometric series.
Proof of formula
We notice that $f(n)=\sum_{k=1}^{100} kn^{100-k}$
$$ \equiv n^{100} \sum{k=1}^{100} k(n^{-1})^k$$Now we substitute $r=n^{-1}, m=100$ in our formula:
$$f(n)=n^{100} \frac{n^{-1}(1-101(n^{-1})^{100}+100(n^{-1})^{101})}{(1-n^{-1})^2}$$Since $\phi(25)=20, a^{20} \equiv 1 (\mod 25)$ for $a$ coprime to $25$ by Euler's totient theorem. $2^{-1} \equiv 13 (\mod 25)$

$$f(2) \equiv (2^{20})^5 \cdot \frac{13(1-1(13^{20})^5)}{(1-13)^2}$$$$\equiv 1^5 \cdot \frac{13(1-1^5)}{144}$$$$\equiv 0 (\mod 25)$$.
We have $f(2) \equiv 0 (\mod 25), f(2) \equiv 2 (\mod 4)$, which gives $f(2) \equiv 50 (\mod 100)$
    }{%
    https://artofproblemsolving.com/community/c4t30973f4h1885343_a_sum_in_chinese_remainder_theoremfunctions__i_m_unable_to_solve
}

    (4pt)\\ \qitem{%
Let $a, b, c$ be the roots of the equation $x^2(x + 1) = 2021.$ Evaluate $$\frac{a}{b^2} + \frac{b}{c^2} + \frac{c}{a^2}.$$
    }{%
    Expand the equation and obtain $x^3+x^2=2021$
By Vieta's $ab+bc+ac=0$ and $a+b+c=-1$
$a^2=\frac{2021}{a+1}$, $b^2=\frac{2021}{b+1}$, $c^2=\frac{2021}{c+1}$
Now we need to find $\frac{a\cdot (b+1)}{2021} + \frac{b\cdot (c+1)}{2021} + \frac{a\cdot (c+1)}{2021}$
$\frac{a\cdot (b+1)}{2021} + \frac{b\cdot (c+1)}{2021} + \frac{a\cdot (c+1)}{2021}=\frac{ab+bc+ac+a+b+c}{2021}=\frac{-1}{2021}$
    }{%
    https://artofproblemsolving.com/community/c4t173f4h3048199_looks_like_vietas
}

    (4pt)\\ \qitem{%
Given a positive integer $n,$ let $s(n)$ denote the sum of the digits of $n.$ Compute the largest positive integer $n$ such that $n = s(n)^2 + 2s(n) - 2.$
    }{%
    We have given a natural number $N $ satisfying the equation

$(1) \;\; N = S^2 + 2S - 2$,

where $S$ is the sum of the digits of $N$.

Let $k$ be the number of digits of $N$. Hence $N \geq 10^{k-1}$ and $S \leq 9k$, which according to equation (1) implies

$10^{k-1} < (9k + 1)^2$,

yielding $k  \leq 3$.

Assume $k=3$. The fact that $9 \mid N - S$ combined with equation (1) give us

$9 \mid (S - 1)(S + 2)$

i.e.

$(2) \;\; S \equiv 1,7 \pmod{9}$.

According to equation (1)

$10^2 + 3 \leq N + 3 = (S + 1)^2$,

which means $S\geq 10$. The fact that $S \leq 9k \leq 9 \cdot 3 = 27$ result in

$(3) \;\; 10 \leq S \leq 27$.

Combining conditions (2) and (3) we obtain

$S \in \{10,16,19,25\}$

which according to equation (1) result in

$(4) \;\; N + 3 = (S + 1)^2  \in \{11^2, 17^2, 20^2, 26^2\}$,

i.e.

$N \in \{118,286, 397,673\}$,

with

$(5) \;\; S + 1 \in \{11,17,20,17\}$.

respectively. Therefore according to conditions (4) and (5) equation (1) is satisfied when

$N \in \{118, 286, 397\}$.

Conclusion: The largest natural number $N$ which satisfies equation (1) is $N=397$.
    }{%
    https://artofproblemsolving.com/community/c4t177f4h3066892_bmt_2022_fall_discrete_p5
}

    (4pt)\\ \qitem{%
Consider a circle whose circumference has points labeled $1, 2, 3,\dots ,100$. Any chord between two of these points is drawn: the chord is labeled with the product of the numbers its endpoints are labeled with. Find the remainder of the sum of all the numbers that label the chords when divided by 102.
    }{%
    The requested number $\pmod{102}$ is
$$\frac{1(1+2+\cdots +100)+2(1+2+\cdots +100)+\cdots +100(1+2+\cdots 100)-(1^2+2^2+\cdots +100^2)}{2}$$$$=\frac{(1+2+\cdots +100)^2-\frac{100\cdot 101\cdot 201}{6}}{2}$$$$=\frac{(50\cdot 101)^2}{2}-(25\cdot 101\cdot 67)\equiv 25\cdot 50\cdot (-1)^2-(25\cdot 67\cdot (-1))\pmod{102}$$$$\equiv 25(50+67)\equiv 25\cdot 117\equiv 25\cdot 15\equiv\boxed{69}\pmod{102}$$ 

Basically right, but its 18 because you overcount (a,b) and (b,a)
    }{%
    https://artofproblemsolving.com/community/c4t177f4h3067059_number_theory
}


    (4pt)\\
    \qitem{%
Let $P$ be a point on the circumcircle of square $ABCD$ such that $PA \cdot PC = 56$ and $PB \cdot PD = 90.$ What is the area of square $ABCD?$
        }{%
        Let $O$ be the center of the square, and let $X$ and $Y$ be the feet from $P$ to $AC$ and $BD$. Then $OXPY$ is a rectangle, so we have

$$\left(\frac{56}{\sqrt{2}s}\right)^2 + \left(\frac{90}{\sqrt{2}s}\right)^2 = \left(\frac{s}{\sqrt2}\right)^2\implies s^2 = \boxed{106}.$$
        }{%
        https://artofproblemsolving.com/community/c5h3011224p27048721
    }

    (4pt)\\ \qitem{%
Let $x_1$ and $x_2$ be the roots of $x^2 + x -m(m+1)(m^2 +1)=0$, where $m$ is an positive integer. Suppose that both $x_1$ and $x_2$ are integers, prove that $x_1^2 + x_2^2 -x_1x_2$ is a product from 2 prime numbers.
    }{%
    I'm assuming it meant that $x_1$ and $x_2$ were the roots of $x^2 + x -m(m+1)(m^2 +1)=0$

We also know that $x_1^2 + x_2^2 -x_1x_2 = (x_1 + x_2)^2 - x_1 x_2$

By Vieta's, $x_1 + x_2 = -1$ and $x_1 x_2 = -m(m+1)(m^2 + 1)$

Then, $(x_1 + x_2)^2 - x_1 x_2 = (-1)^2 + m(m+1)(m^2+1) = m^4 + m^3 + m^2 + m + 1$

$x_1, x_2= \frac{1}{2}\left(-1\pm\sqrt{1+4m(m+1)(m^2+1)}\right)=\frac{1}{2}\left(-1\pm\sqrt{4m^4+4m^3+4m^2+4m+1}\right)$

$\Delta=a^2\Longrightarrow m=0, m=2~~~ (m=0$ not acceptable: see "positive integer")

$x_1, x_2=\frac{1}{2}\left(-1\pm \sqrt{121}\right) \Longrightarrow x_1, x_2 = 5, -6$


Other possible positive values for m, which render "$\Delta$" a square: None because

For $m>2, (2m^2+m)^2=4m^4+4m^3+m^2<4m^4+4m^3+4m^2+4m+1<(2m^2+m+1)^2=4m^4+4m^3+5m^2+2m+1$

Solving $\rightarrow x_1=5, x_2=-6$

$x_1^2 + x_2^2 -x_1x_2=25+36+30=91=7*13$

The discriminant is apparently a square only if $m=0 (*), m=2$, as possibly shown by "entrapment"

(*) to be discarded by the wording "positive integer"
    }{%
    https://artofproblemsolving.com/community/c4t173f4h3049636_vietas_formulas
}

\end{shortque*}



\end{document}
