\pitem[]{%
    Let $A_1,A_2,\dots,A_{2016}$ be sets saisfying$:$
    \begin{enumerate}
        \item Each set $A_i$ contains exactly $20$ elements.
        \item Each intersection $A_i\cap A_j,$ $i\neq j,$ of two distinct sets contains exactly one element.
    \end{enumerate}
    Prove that there exists an element belonging to all $2016$ sets$.$
    }{%
    We observe that if an element belongs to all $2016$ sets, all of the pairwise intersections of those sets will be the set containing only that element, and vice versa.

    We claim that it must be the case that all of the pairwise intersections of the $2016$ sets are the same.

    We see that if we compare $A_2,A_3...A_{2016}$ to $A_1$, by the Pidgeonhole principle, at least one element $E_1$ will be shared among at least $101$ sets, including $A_1$. Let this set of sets be $S$. For the sake of contradiction, let $A_2$ be a set that does not share this element. There are $20$ possible intersections between $A_2$ and any other set $A_i$ as $A_2$ has $20$ elements, so there must be some element $E_2$ in $A_2$ that is the intersection between $A_2$ and at least two sets in $S$. As $E_2$ cannot be $E_1$ as $E_1$ is not in $A_2$, these sets in $S$ share more than one element with each other which contradicts part $b$ of their definition.

    Thus, our assumption that a set $A_i$ exists that does not contain $E_1$ is incorrect, and all sets $A_i$ contain $E_1$.
    }{%
    https://artofproblemsolving.com/community/c6t275f6h2328689_set_theory
}
