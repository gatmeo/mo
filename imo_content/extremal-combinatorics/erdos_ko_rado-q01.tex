\pitem[Roumanian JBMO team selection test 2013]{%
    A set that contains $7$ elements is given. All its subsets containing $3$ elements are colored with $n$ colors such that every two disjoint subsets have different colors. Find the least value of $n$ for which such a coloring its possible.
    }{%
    Let's color first three element subsets $A$ of $\{1,2,3,4,5,6\}$ with red and blue:

    If $1\in A$ then $A$ is red, othervise blue.

    Now color all other three element subset of $\{1,2,3,4,5,6,7\}$ with green (they all have $7$ in it).

    So $n\leq 3$. Suppose $n=2$.

    Since for each pair $A,B$ of subsets in monocolor family satisfy $|A\cap B|\geq 1$ it follow from Erdos-Ko-Rado Theorem that cardinality of this family is no more then ${7-1\choose 3-1} = 15 $. But then we can color at most $30$ subsets. Contradiction, since there is ${7\choose 3} = 35 $ three element subsets in $\{1,2,3,4,5,6,7\}$.
    }{%
    https://artofproblemsolving.com/community/c6h535946p3076610
}
