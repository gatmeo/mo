\pitem[]{%
    Ana and Bruno decide to play a game with the following rules. Ana has cards $1, 3, 5,7,..., 2n-1$, and Bruno has cards $2, 4,6, 8,...,2n$.
    During the first turn and all odd turns afterwards, Bruno chooses one of his cards first and reveals it to Ana, and Ana chooses one of her cards second. Whoever's card is higher gains a point. During the second turn and all even turns afterwards, Ana chooses one of her cards first and reveals it to Bruno, and Bruno chooses one of his cards second. Similarly, whoever's card is higher gains a point. During each turn, neither player can use a card they have already used on a previous turn. The game ends when all cards have been used after $n$ turns. Determine the highest number of points Ana can earn.
    }{%
    We claim that Anna can gain a maximum of $\left\lfloor \frac{n}{2} \right\rfloor$ points.  
    We prove by induction that this maximum score is possible.

    \textbf{Base case:} $n=1$ is trivial.
    For $n=2$, if Bruno plays $4$, Anna plays $1$ and she wins the next point. If Bruno plays $2$ first, Anna plays $3$ and wins the first point and loses the second.

    \textbf{Induction:} Now assume the maximum score is possible for $n-2$. For $n$, if Bruno plays $2n$, Anna plays $1$, and the next turn no matter what Bruno plays, Anna plays the number above. If Bruno plays anything lower than $2n$ first, Anna plays the next number higher and thus wins that point. Anna then plays $1$ and loses the second point. At this point, when Anna and Bruno's cards are considered together from largest to smallest, Bruno has the highest card and every other card after that, and Anna has the second highest card and every other card after that, so the case is identical to the $n-2$ case and since Anna has one more point going into that case, the maximum is always possible by induction.

    We now prove that no higher score is possible. This is equivalent to proving that Bruno can always gain $\left\lceil \frac{n}{2} \right\rceil$ points.
    Assume Bruno plays $2n$ on his first move, guaranteeing him that point. Then Anna optimally plays $1$; since she will lose the point anyways, it is obviously optimal for her to maximize the cards she has left. But here, the roles of Bruno and Anna are reversed: Anna, having the highest card and every other card after that, plays first this turn, and Bruno has the second highest card and every other card after that. So this case is identical to the previous case with Anna. Thus Bruno can gain another $\left\lfloor \frac{n-1}{2} \right\rfloor$ points, giving him $\left\lfloor \frac{n+1}{2} \right\rfloor$ points in total, equal to $\left\lceil \frac{n}{2} \right\rceil$ points as $n$ is an integer. So thus no higher score for Anna is possible and the result is proved.
    }{%
    https://artofproblemsolving.com/community/c6t224166f6h2682590_2_player_card_game_with_2n_cards
}
