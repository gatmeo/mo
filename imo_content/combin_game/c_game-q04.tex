\pitem[Putnam 2002 B2]{%
    Consider a polyhedron with at least five faces such that exactly three edges emerge from each of its vertices. Two players play the following game: Each, in turn, signs his or her name on a previously unsigned face. The winner is the player who first succeeds in signing three faces that share a common vertex. Show that the player who signs first will always win by playing as well as possible.
    }{%
    This was a game played on "a polyhedron with at least five faces such that exactly three edges emerge from each of the vertices." It has been pointed out to me that this problem was ill-stated: it should have been a convex polyhedron, or otherwise have the stipulation that no two faces can share two edges. If you start with a tetrahedron and cut a "notch" into one edge, creating two new triangular faces, three new edges, splitting one original edge into two separated pieces, and turning two of the original triangles into non-convex hexagons that meet along two edges, then you create a figure upon which the second player can always force a draw.

    So let us assume that the statement has been changed to prohibit this example - assume that a pair of faces meets along at most one edge. We first prove that the polyhedron must contain at least one face that has at least four sides. The polyhedron has $f$ faces. The $i$th face has $n_i$ edges, and let $n_i = 3 + x_i$, thinking of $x_i$ as then number of "excess" edges per face. The total number of edges is $e = \frac12\sum_{i=1}^f n_i$ , and the total number of vertices is $v = \frac23 e = \frac13\sum_{i=1}^f n_i $. The Euler characteristic says that $f - e + v = 2,$ or $f - \frac16\sum_{i=1}^f n_i = 2.$ Insert the definition of $n_i$ to get
    $f - \frac12 f - \frac16 \sum_{i=1}^f x_i = 2,$ which rearranges to $\sum_{i=1}^f x_i = 3f - 12.$ Since $f\ge 5,$ we then have that the sum of the number of "excess above triangle" edges must be at least three, so there must be at least one face that has at least four edges.

    Player $A$ starts the game by signing in a face that has at least four edges. No matter where B signs (best would be an adjacent face), there must be a face adjacent to $A$'s original face that has both of its vertices that are in common with $A$'s first face not shared with $B$'s first face. (Note that if there is a face that shares at least two edges with $A$'s first face, $B$ may choose that and negate this strategy.) Then $A$ has two chances to win on the next move and $B$ can only block one of them.
    }{%
    https://artofproblemsolving.com/community/c7t46107f7h469262_putnam_2002_b2
}
