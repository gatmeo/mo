\pitem[Putnam 2013 B6]{%
    Let $n\ge 1$ be an odd integer. Alice and Bob play the following game, taking alternating turns, with Alice playing first. The playing area consists of $n$ spaces, arranged in a line. Initially all spaces are empty. At each turn, a player either

    \begin{enumerate}
        \item places a stone in an empty space, or
        \item removes a stone from a nonempty space $s,$ places a stone in the nearest empty space to the left of $s$ (if such a space exists), and places a stone in the nearest empty space to the right of $s$ (if such a space exists).
    \end{enumerate}

    Furthermore, a move is permitted only if the resulting position has not occurred previously in the game. A player loses if he or she is unable to move. Assuming that both players play optimally throughout the game, what moves may Alice make on her first turn?
    }{%
    We show that the only winning first move for Alice is to place a stone in the central space. We start with some terminology.

    %Divide the playing area into the \emph{left half}, the \emph{central %space}, and the \emph{right half}.
    By a \emph{block} of stones, we mean a (possibly empty) sequence of stones occupying consecutive spaces. By the \emph{extremal blocks}, we mean the (possibly empty) maximal blocks adjacent to the left and right ends of the playing area.

    We refer to a legal move consisting of placing a stone in an empty space as a move of \emph{type 1}, and any other legal move as being of \emph{type 2}.
    For $i=0,\dots,n$, let $P_i$ be the collection of positions containing $i$ stones. Define the \emph{end zone} as the union $Z = P_{n-1} \cup P_n$. In this language, we make the following observations.
    \begin{itemize}
        \item
            Any move of type 1 from $P_i$ ends in $P_{i+1}$.
        \item
            Any move of type 2 from $P_n$ ends in $P_{n-1}$.
        \item
            For $i < n$, any move of type 2 from $P_i$ ends in $P_i \cup P_{i+1}$.
        \item
            At this point, we see that  the number of stones cannot decrease until we reach the end zone.
        \item
            For $i < n-1$, if we start at a position in $P_i$ where the extremal blocks have length $a,b$, then the only possible moves to $P_i$ decrease one of $a,b$ while leaving the other unchanged (because they are separated by at least two empty spaces). In particular, no repetition is possible within $P_i$, so the number of stones must eventually increase to $i+1$.
        \item
            From any position in the end zone, the legal moves are precisely to the other positions in the end zone which have not previously occurred. Consequently, after the first move into the end zone, the rest of the game consists of enumerating all positions in the end zone in some order.
        \item
            At this point, we may change the rules without affecting the outcome by eliminating the rule on repetitions and declaring that the first player to move into the end zone loses (because $\# Z = n+1$ is even).
    \end{itemize}

    To determine who wins in each position, number the spaces of the board $1,\dots,n$ from left to right. Define the \emph{weight} of a position to be the sum of the labels of the occupied spaces, reduced modulo $n+1$. For any given position outside of the end zone, 
    for each $s=1,\dots,n$ there is a unique move that adds $s$ to the weight:
    if $s$ is empty that a move of type 1 there does the job.
    Otherwise, $s$ inhabits a block running from $i+1$ to $j-1$ with $i$ and $j$ empty (or equal to $0$ or $n+1$), so the type 2 move at $i+j-s$ (which belongs to the same block) does the job.

    We now verify that a position of weight $s$ outside of the end zone is a win for the player to move if and only if $s \neq (n+1)/2$. We check this for positions in $P_i$ for $i = n-2, \dots, 0$ by descending induction. For positions in $P_{n-2}$, the only safe moves are in the extremal blocks; we may thus analyze these positions as two-pile Nim with pile sizes equal to the lengths of the extremal blocks. In particular, a position is a win for the player to move if and only if the extremal blocks are unequal, in which case the winning move is to equalize the blocks. In other words, a position is a win for the player to move unless the empty spaces are at $s$ and $n+1-s$ for some $s \in \{1,\dots,(n-1)/2\}$, and indeed these are precisely the positions for which the weight equals $(1 + \cdots + n) - (n+1) \equiv (n+1)/2 \pmod{n+1}$.
    Given the analysis of positions in $P_{i+1}$ for some $i$, it is clear that if a position in $P_i$ has weight $s \neq (n+1)/2$, there is a winning move of weight $t$ where $s+t \equiv (n+1)/2 \pmod{n}$,
    whereas if $s = (n+1)/2$ then no move leads to a winning position.

    It thus follows that the unique winning move for Alice at her first turn is to move at the central space, as claimed.
    }{%
    https://artofproblemsolving.com/community/c7t219f7h566373_putnam_2013_b6
    https://kskedlaya.org/putnam-archive/
}
