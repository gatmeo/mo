\pitem[]{%
    Each side and diagonal of a regular $n-$gon $(n \ge 3)$ is colored blue or green. A move consists of choosing a vertex and switching the color of each segment incident to that vertex (from blue to green or vice versa). Prove that regardless of the initial coloring, it is possible to make the number of blue segments incident to each vertex even by following a sequence of moves. Also show that the final configuration obtained is uniquely determined by the initial coloring.
    }{%
    The statement suffers from a major flaw; the OP failed to mention that $n$ must be odd.

Let us rephrase it in graph theoretical terminology. We are given a graph $G=(V,E)$ on $|V|=n$ vertices (where the edges in $E$ are the blue segments from the original). A move on a vertex $v\in V$ erases all edges $vw$ for $w\in N(v)$ and creates new edges $vu$ for $w\not \in N(v)\cup \{v\}$. The effect is that in the new graph $G'$ we get $\deg_{G'} v = n-1-\deg_G v$, $\deg_{G'} w = \deg_G w - 1$ for $w\in N(v)$, and $\deg_{G'} u = \deg_G u + 1$ for $u\not \in N(v)\cup \{v\}$.

Clearly a sequence of moves is commutative, and moves are toggles (performing a move twice is equivalent to not performing the move); thus there are precisely $2^n$ sequences of moves, corresponding to the $2^n$ parts of $V$. The effect on the degrees of vertices in the graph $G'$ obtained after a sequence of moves on vertices in a part $\emptyset \subseteq S \subseteq V$ is therefore that $\deg_{G'} s \equiv \deg_G s + |S| + n \pmod{2}$ for $s\in S$ and $\deg_{G'} t \equiv \deg_G t + |S| \pmod{2}$ for $t\not \in S$.

Since we are asked for a sequence $S$ of moves such that all degrees become even, we need $\deg_{G} s \equiv |S| + n \pmod{2}$ for $s\in S$ and $\deg_{G} t \equiv |S| \pmod{2}$ for $t\not \in S$.

For $n$ even, that means we must have $\deg_{G} v \equiv |S| \pmod{2}$ for all $v\in V$, i.e. all degrees of the vertices in $G$ must be of same parity. If they are even, we've done, while if they are odd, we may apply a move on just one vertex. But of course, it is possible that not all degrees have same parity, and then the task required cannot be performed; for example for $n=4$ and $G$ having just one edge.

For $n$ odd, that means the degrees of the vertices in $S$ must all be of same parity, different from the parity of the degrees of the vertices in $V\setminus S$. Thus $S$ must either be the set of all odd vertices from $G$ (thus having even cardinality), or else it must be the set of all even vertices from $G$ (thus having odd cardinality); in both cases the degrees of all the vertices in $G'$ will become even, and applying one or the other leads to the same final configuration. This is because applying the full sequence $V$ of moves clearly leaves the graph unchanged; thus for any sequence $S$, applying $S$ then $V$ is equivalent to applying $V\setminus S$.
    }{%
    https://artofproblemsolving.com/community/c6t220f6h1060984_even_number_of_edges
}
