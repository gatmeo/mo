\pitem[]{%
    Let $S$ be a set of $n$ persons such that:
    \begin{enumerate}
        \item any person is acquainted to exactly $k$ other persons in $S$;
        \item any two persons that are acquainted have exactly $l$ common acquaintances in $S$;
        \item any two persons that are not acquainted have exactly $m$ common acquaintances in $S$.
    \end{enumerate}
    Prove that $$m(n - k) - k(k - l) + k - m = 0.$$
    }{%
    We double count triplets $(P_1, P_2, C)$, where $P_1$ and $P_2$ are two people in $S$ and $C$ is a common acquaintance of the two.

    Count 1: Pick a person $P$. Then, there are $k$ people $P$ is acquainted to. If we choose any two of these people, they clearly form a triplet of the form we want. Since there are $n$ people total, there are $$n\binom{k}{2}$$such triplets.

    Count 2: We now count by cases: when $P_1$ and $P_2$ are acquainted, and when they are not acquainted. Fix $P_1$. If $P_2$ knows $P_1$, there are $k$ choices for $P_2$. Since the order in which we choose these people is irrelevant and there are $n$ choices for $P_1$, there are $\frac{nk}{2}$ such pairs $P_1, P_2$. We are given that $P_1$ and $P_2$ have $l$ common acquaintances in $S$, so there are $\frac{nkl}{2}$ such triplets where $P_1$ and $P_2$ are acquainted.

    If $P_1$ and $P_2$ are not acquainted, after fixing $P_1$, there are $n - k - 1$ choices for $P_2$ (make sure you see why this isn't $n - k$ - we don't want $P_1$ to be counted in the set of such $P_2$ ). Since there are $n$ choices for $P_1$, and order is irrelevant, there are $\frac{n(n - k - 1)}{2}$ such pairs $P_1$ and $P_2$. We are given that $P_1$ and $P_2$ have $m$ common acquaintances in $S$, so there are $\frac{n(n - k - 1)l}{2}$ such triplets where $P_1$ and $P_2$ are not acquainted.

    It follows that there are $$\frac{nkl}{2} + \frac{n(n - k - 1)l}{2}$$such triplets.

    Thus,
    \begin{eqnarray*} \frac{nkl}{2} + \frac{n(n - k - 1)l}{2} & = & n\binom{k}{2} \\ kl + (n - k - 1)l & = & k(k - 1) \\ m(n - k) - k(k - l) + k - m & = & 0. \hspace{0.5in} \Box \end{eqnarray*}
    }{%
    https://artofproblemsolving.com/community/c474725h1471295_combinatorics_post__1_double_counting
}
