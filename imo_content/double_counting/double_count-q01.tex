\pitem[China TST 1992, problem 1]{%
    16 students took part in a competition. All problems were multiple choice style. Each problem had four choices. It was said that any two students had at most one answer in common, find the maximum number of problems.
    }{%
    Let the students be $S_1, S_2, \ldots, S_{16}$ and the questions be $Q_1, Q_2, \ldots, Q_n$. Define $X$ as the number of triples $(S_i, S_j, Q_k)$ where $1 \le i < j \le 16$ such that $S_i$ and $S_j$ got the same answer for $Q_k$.

    Because any two students have at most one answer in common, we have $$X \le 1 \cdot \binom{16}{2} = 120.$$Now, we find a lower bound for $X$ by analyzing each question individually. 

    For each $Q_k$, suppose $a$ people answered "A" and so on. Since $a+b+c+d = 16$, Cauchy-Schwarz implies 
    \begin{alignat*}{1}
        \binom{a}{2} + \binom{b}{2} + \binom{c}{2} + \binom{d}{2} 
        &= \frac{a^2 + b^2 + c^2 + d^2 - (a+b+c+d)}{2}\\
        &\geq \frac{(a+b+c+d)^2}{8} - 8 = 24.
    \end{alignat*}
    Thus, there are at least $24$ triples of the form $(S_i, S_j, Q_k)$. Now, symmetry implies
    \[
        24n \le X \le 120 \implies n \le 5.
    \]
    }{%
    https://artofproblemsolving.com/community/c6t45317f6h42555_16_students_took_part_in_a_competition
}
