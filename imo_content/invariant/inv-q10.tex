\pitem[]{%
    Positive integers between $1$ to $100$ inclusive are written on a blackboard, each exactly once. One operation involves choosing two numbers $a$ and $b$ on the blackboard and erasing them, then writing the greatest common divisor of $a^2+b^2+2$ and $a^2b^2+3$. After a number of operations, there is only one positive integer left on the blackboard. Prove this number cannot be a perfect square.
    }{%
    Note that $-3$ is not a quadratic residue mod $9$, so $9$ can never divide $a^{2}b^{2} + 3$. Thus, as long as an operation is performed on some integer, the resulting integer will not be a multiple of $9$.

Now, we claim that the parity of the number of multiple of $3$s is invariant. If we choose $a, b$ not divisible by $3$, then $3 \nmid a^{2}b^{2} + 3$, so the gcd will not be divisible by $3$. If $a, b$ are both divisible by $3$, then $a^{2} + b^{2} + 2$ is not divisible by $3$. If exactly one of $a, b$ is divisible by $3$, then we can check that indeed $a^{2} + b^{2} + 2 \equiv 1 + 2 \equiv 0 \pmod{3}$ and $3 \mid a^{2}b^{2} + 3$. This proves the claim.

Since there are $33$ multiples of $3$ in the beginning, the final number must be a multiple of $3$. Since it's not divisible by $9$, it cannot be a perfect square, as desired.
    }{%
    https://artofproblemsolving.com/community/c6t219f6h1590925_gcd_of_numbers_on_blackboard
}
