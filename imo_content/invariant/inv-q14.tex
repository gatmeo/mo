\pitem[]{%
There are $50$ cards, two cards with number $k$ written on them, for every $k=1,2,...,25$ , Also, there are $25$ people, and they are seated around a table so that every person has exactly two cards. Every moment, every person passes the card with the lower number in their hand to the person at their right. Prove that eventually, a person will exist that has two cards of the same number at some moment.
    }{%
    Say in the beginning, there is not a person with two cards of the same number. Call a card unmovable if it is moved to a person's hand and never moved again. Note that at the beginning, both $25$s are unmovable. Now, if no person has a $24$ and a $25$, then both $24$s are unmovable. Otherwise, after at max two moves (because the max case is when in the beginning a $24$ can be in the same person as a $25$, then the $24$ gets moved to the right whose person also has a $25$, then the $24$ moves again), either the $24$s are held by the same person (in which case we're done) or all four of the $24$s and $25$s are in distinct persons' hands. This makes the $24$s unmovable because they can only move if they're in the same hand as a person with a $25$, which won't happen again.
For the general case $n$ where $n \ge 14$, when all of the cards $n+1,\cdots,25$ are unmovable, after at max $2(25-n)$ moves (because the max case is when the $n$ card moves from one card to the next in the set $\{n+1,n+1,\cdots,25,25\}$ before moving to a hand where the other card is less than or equal to $n$, and there are at max $2(25-n)<50$ cards in this set so the $n$ card won't come back in a circle), either the $n$s are held by the same person (in which case we're done) or all of the cards $\{n,n,\cdots,25,25\}$ are in distinct persons' hands, making the $n$ cards unmovable.
Eventually, all of the cards in $\{14,14,\cdots,25,25\}$ become unmovable, and one of the $13$ cards (call it $C_1$) also becomes unmovable by sliding into the last person who doesn't have an unmovable card yet. Then the cards $\{1,1,\cdots,12,12,C_2\}$ where $C_2$ is the other $13$ card are forced to continuously move in the circle. Eventually, $C_2$ will be in the same person's hand as $C_1$, and we have two $13$s held by a person. $\square$
    }{%
    https://artofproblemsolving.com/community/c6t220f6h1887763_invariants_and_monovariants_problem
}
