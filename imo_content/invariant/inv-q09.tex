\pitem[]{%
    Given $2004$ vertices $a_1,a_2,\cdots,a_{2004}$ circling a table in clockwise order. Initially, $a_1$ is labeled $0$ and $a_2,a_3,\cdots,a_{2004}$ are labeled $1$. In every step, we choose three vertices $a_{i-1},a_i,a_{i+1}$ with labels $a,b,c$ and change them into $1-a,1-b,1-c$ respectively. Is it possible to change all labels into $0$?
    }{%
    The answer is no; between any two numbers write down their absolute difference, and consider the circle containing only these absolute differences. To be more rigorous, let $b_i = |a_{i+1} - a_i|$, where we take $a_{2005} = a_1$. Then $b_1 = b_2 = 1$ and $b_3 = b_4 = \cdots = b_{2004} = 0$. Observe that an operation consists of choosing two numbers and flipping them. To be more rigorous, flipping vertices $a_{i-1}, a_i, a_{i+1}$ corresponds to flipping the numbers $b_{i-1}$ and $b_{i+2}$. We show that it's impossible to have $b_i = 0$ for all $1 \leq i \leq 2004$.

    Consider the numbers $b_1, b_4, b_7, \cdots, b_{2002}$, i.e. those with indices which are 1 mod 3. Initially only one of these is labeled $1$. Any operation preserves the parity of the number of 1s among these numbers. Thus we can never have $b_1 = b_4 = \cdots = 0$, which implies the result.
    }{%
    https://artofproblemsolving.com/community/c6t219f6h2269808_is_it_possible_to_make_all_labels_zero
}
