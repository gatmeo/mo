\pitem[AIMO 3, German Pre-TST 2009]{%
    Initially, on a board there a positive integer. If board contains the number $x,$ then we may additionally write the numbers $2x+1$ and $\frac{x}{x+2}.$ At some point 2008 is written on the board. Prove, that this number was there from the beginning.
    }{%
    Construct an infinite binary tree, with $2x+1$ the left child of $x$ and $\frac{x}{x+2}$ the right child of $x$ for every node $x$. Let $N$ be the root of this tree.

    For any rational number $x=p/q$ with $\gcd(p,q)=1$, define the parity $o(x)=(p\mod 2,q\mod 2)$.

    If $o(x)=(1,1)$, then every number in its subtree has parity $(1,1)$, so 2008 is not found in this subtree. From this it follows that $N$ must be even.

    If $o(x)=(0,1)$, then $o(2x+1)=(1,1)$, so its left subtree does not have a 2008. That means 2008 must be in its right subtree. Similarly, if $o(x)=(1,0)$, then $o\left(\frac{x}{x+2}\right)=(1,1)$, so 2008 must be in $x$'s left subtree.

    We now prove an invariant: For every node $x=p/q$ with $\gcd(p,q)=1$ on the path from $N$ to 2008, $p+q=N+1$.

    We prove this by induction. Certainly this is true for $N$ itself. Now let $x$ be a node for which this is true. If $o(x)=(1,0)$, then
    \[2x+1=\frac{2p+q}{q}=\frac{p+q/2}{q/2}.\]
    This fraction is irreducible and the sum of its numerator and denominator is $p+q=N+1$. On the other hand, if $o(x)=(0,1)$, then
    \[\frac{x}{x+2}=\frac{p}{p+2q}=\frac{p/2}{p/2+q}.\]
    The same thing applies to this.

    At the end of this path is 2008, so it must be that $N=2008$.
    }{%
    https://artofproblemsolving.com/community/c6t219f6h418596_at_some_point_2008_is_written_on_the_board
}
