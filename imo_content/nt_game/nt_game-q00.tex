\pitem[Tournament of Towns 2016 Fall Tour, A Senior, P6]{%
    Petya and Vasya play the following game. Petya conceives a polynomial $P(x)$ having integer coefficients. On each move, Vasya pays him a ruble, and calls an integer $a$ of his choice, which has not yet been called by him. Petya has to reply with the number of distinct integer solutions of the equation $P(x)=a$. The game continues until Petya is forced to repeat an answer. What minimal amount of rubles must Vasya pay in order to win?
    }{%
    This is a very cute problem, so let's hope I got this right for Anant's sake :P

    We claim that Vasya could do with $4$ rubles. We'll use the following notation: for an integer $a$, $n(a)$ is the number of distinct integer roots of $P(x)=a$. We'll first prove a lemma:

    Lemma: If $P(x)=a$ has $3$ or more distinct integer root for some $P(x)\in\mathbb Z[x]$ and $a\in \mathbb Z$, then both the equations $P(x)=a+1$ and $P(x)=a-1$ have no integer root.
    Proof: We can shift the origin appropriately and assume WLOG $a=0$. Then $$P(x)=(x-a_1)^{b_1}(x-a_2)^{b_2}(x-a_3)^{b_3}Q(x)$$for integers $a_i$'s , $b_i$'s and $Q\in\mathbb Z[x]$. Here $a_i$'s are all distinct and $b_i$'s are all positive. Then to have $|P(x)|=1$, we'll need $(x-a_i)=\pm 1$ for each $i$; but by PHP, this is impossible with distinct $a_i$'s. This proves our lemma. $\square$

    Now for the main problem. We'll first show that Vasya can win in at most $4$ moves. Let Vasya call the integers $0,-1,+1$ in the first three moves. If $n(0)\ge 3$, then $n(1)=n(-1)=0$ by the lemma and Vasya wins. If one of $n(1)$ and $n(-1)$ is $\ge 3$, say WLOG $n(1)=3$; then Vasya can call $2$ in the next move and win, because then $n(0)=n(2)=0$ by our lemma. So suppose $n(0), n(1),n(-1)$ are all $\le 2$. If some of two these were equal, Vasya would've won already, so let's say $n(0),n(-1),n(1)$ are $0,1,2$ in some order. Then at least one of $n(1)$ and $n(-1)$ is nonzero; say WLOG $n(1)\ne 0$. Then let Vasya call $2$ in the fourth move. $n(2)$ can't be $\ge 3$, or we would have $n(1)=0$; so $n(2)\in\{0,1,2\}$, so it has to collide with one of $n(0),n(1),n(-1)$, letting Vasya win.

    Now it remains to show that Vasya can't be sure of winning in $3$ moves or less. Indeed, we may WLOG assume the first integer called by Vasya is $0$. Then if he calls the integers $a$ and $b$ in the next two moves and hopes to win, then Petya could smash his hopes by conceiving the polynomial $$P(x)\equiv -a(x-1)(x-3)\left( (x-2)^2\left(x^{2016}+|b|+2017^{2017}\right)+1\right).$$Indeed, $P(x)=0$ has two roots $1$ and $3$; $P(x)=a$ has only one root $2$, because $$P(x)=a\implies |(x-1)(x-3)|=1\implies x=2;$$and $P(x)=b$ has no roots at all, because for $x\not\in \{1,2,3\}$, we have $|P(x)|>b$. Thus we're done. $\blacksquare$
    }{%
    https://artofproblemsolving.com/community/c6t45487f6h1435572_game_of_polynomials
}
