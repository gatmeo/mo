\section{Lucas' Theorem and Kummer's Theorem}

What is the remainder of ${m\choose n}$ mod $p$? And if ${m\choose n}$ is divisible by $p$, what is the $p$-adic valuation? These two questions are answered by \textbf{Lucas' Theorem} and \textbf{Kummer's Theorem} respectively.

\begin{theorem}[thm:]{Lucas' Theorem}
    Write the numbers $n$ and $k$ in base $p$: $n=\overline{n_dn_{d-1}\cdots n_0}_p$, $k=\overline{k_dn_{d-1}\cdots k_0}_p$.
    Then ${n\choose k}\equiv \prod_{i=0}^{k}{n_i\choose k_i}$, with ${n\choose k}=0$ when $n<k$. In particular, ${n\choose k}$ is divisible by $p$ iff at least one of the base-$p$ digits of $n$ is greater than the corresponding base-$p$ digit of $k$. 
\end{theorem}

\begin{lemma}[lem:]{}
    For $p$ prime and $x,r\in\mathbb{Z}$, $(1+x)^{p^r}\equiv 1+x^{p^r}\pmod{p}$.
    \begin{proof}
        For all $1\leq k \leq p-1$, $\binom{p}{k}\equiv 0 \pmod{p}$. Then we have
        \begin{eqnarray*}(1+x)^p&\equiv &\binom{p}{0}+\binom{p}{1}x+\binom{p}{2}x^2+\cdots+\binom{p}{p-1}x^{p-1}+\binom{p}{p}x^p\\ &\equiv& 1+x^p\pmod{p}\end{eqnarray*}
        Assume we have $(1+x)^{p^k}\equiv 1+x^{p^k}\pmod{p}$. Then
        \begin{alignat*}{1}
            (1+x)^{p^{k+1}} &\equiv\left((1+x)^{p^k}\right)^p\equiv\left(1+x^{p^k}\right)^p\\&\equiv1+x^{p^{k+1}}\pmod{p}
        \end{alignat*}
    \end{proof}
\end{lemma}

\begin{example}[exp:]{Lucas}
    Suppose $p=5, n=482=\overline{3412}_5, k=176=\overline{1201}_5$. By the lemma we have
    \begin{alignat*}{1}
        (1+x)^{482}&\equiv (1+x)^{125(3)}\cdot (1+x)^{25(4)}\cdot (1+x)^{5(1)}+(1+x)^2\pmod 5\\
                   &\equiv (1+x^{125})^3\cdot (1+x^{25})^4\cdot (1+x^5)^1\cdot (1+x)^2.
    \end{alignat*}
    Since the only way to get $x^{176}$ is by three $x^{125}$ terms, two $x^{25}$ terms, zero $x^{5}$ terms, and one $x^1$ terms on the RHS. Hence the number of ways to get such term is
    \[{482\choose 176}\equiv {3\choose 1}{4\choose 2}{1\choose 0}{2\choose 1}.\]
\end{example}

\begin{proof}(Lucas')
    The proof of Lucas' theorem follow a similar structure of the example. Another way to proof the theorem is by writting $n=Np+i, k=Kp+j$, and proceed by MI with some pascal's triangle properties.
\end{proof}

\begin{example}[exp:]{lucas}
Prove that the number of odd binomial coefficients in
$n$-th row of Pascal triangle is equal to $2^r$,
where $r$ is the number of 1's in the binary expansion of $n$.
\begin{proof}
    Write $n=\overline{n_dn_{d-1}\cdots n_0}$, ${n\choose k}$ is odd iff $k_i\leq n_i\ \forall\ i$, and hence there are $2$ choices for $k_i$ if $n_i=1$ and $1$ choice otherwise, i.e. the number of odd entries equal $2^r$.
\end{proof}
\end{example}

\begin{definition}[def:]{$p$-adic valuation}
    Given a non-zero integer $n$ and a prime $p$, we define the $p$-adic valuation of $n$ as the highest power of $p$ that divides $n$, and write it as $v_p(n)$.
\end{definition}

\begin{theorem}[thm:]{Kummer's Theorem}
    Given a prime $p$ and integers $m,n$, $v_p\left({m\choose n}\right)$ equals the number of carries in adding $m-n$ and $n$ in base $p$. 
\end{theorem}

\begin{theorem}[thm:]{Legendre's Formula}
    Let $n\in \mathbb{N}$, $p$ be prime. Let $s_p(n)$ be the sum of the digits of $n$ in base $p$. Then
    \[v_p(n!)=\frac{n-s_p(n)}{p-1}.\]
    \begin{proof}
        Let $n=\overline{a_ka_{k-1}\cdots a_0}_p$ be base $p$ representation. We have $s_p(n)=\sum_{i=0}^{k}a_k$, then
        \begin{alignat*}{1}
            v_p(n!)
            &= \lfloor\frac{n}{p}\rfloor + \lfloor\frac{n}{p^2}\rfloor + \lfloor\frac{n}{p^3}\rfloor + \cdots \\
            &= (a_kp^{k-1}+\cdots +a_1) + (a_kp^{k-2}+a_{k-1}p^{k-3}+\cdots +a_2)+\cdots +(a_kp+a_{k-1})+a_k\\
            &= a_k\frac{p^k-1}{p-1}+a_{k-1}\frac{p^{k-1}-1}{p-1}+\cdots +a_1\\
            &= \frac{1}{p-1}\left((a_kp^k+a_{k-1}p^{k-1}+\cdots +a_0) - (a_k+a_{k-1}+\cdots +a_0)\right)\\
            &= \frac{n-s_p(n)}{p-1}.
        \end{alignat*}
    \end{proof}
\end{theorem}

\begin{proof}(Kummer's)
    Write $m=\overline{a_ra_{r-1}\cdots a_0}_p, m-n=\overline{b_rb_{r-1}\cdots b_0}_p, n=\overline{c_rc_{r-1}\cdots c_0}_p$. Define carry func $\gamma$: 
    \[\gamma_i=
        \begin{cases}
            1&\textnormal{ if }b_i+c_i+\gamma _{i-1}\geq p,\\
            0&\textnormal{ if }b_i+c_i+\gamma _{i-1}<p.
        \end{cases}\ \forall\ 0\leq i<r, \gamma _{^{-1}}=0.
    \]
    Since $a_r$ is the leading coeff of $m$, $\gamma _r=0$. And observe that $a_i=b_i+c_i+\gamma _{i-1}-p\gamma _i\ \forall\ 0\leq i\leq r$. By Legendre, 
    \begin{alignat*}{1}
        v_p\left({m\choose n}\right)
            &= v_p(m!)-v_p((m-n)!)-v_p(n!)\\
            &= \frac{m-s_p(m)}{p-1}-\frac{m-n-s_p(m-n)}{p-1}-\frac{n-s_p(n)}{p-1}\\
            &= \frac{\sum_{i=0}^{r}b_i+c_i-a_i}{p-1}\\
            &= \frac{p\gamma _0+(p\gamma _1-\gamma _0)+\cdots +(p\gamma _{r-1}-\gamma _{r-2})-\gamma _{r-1}}{p-1}\\
            &= \gamma _0+\gamma _1+\cdots +\gamma _{r-1}.
    \end{alignat*}
\end{proof}

\newpage
\section{Some other Theorem}

\begin{theorem}[thm:]{Wilson's Theorem}
    For any prime $p$ (and for primes only), $(p-1)!\equiv -1\pmod{p}$.
\end{theorem}

\begin{theorem}[thm:]{Wolstenholme's Theorem}
    If $p\geq 5$, then ${2p\choose p}\equiv 2\pmod{p^3}$, or equivalently, ${2p-1\choose p-1}\equiv 1\pmod{p^3}$.
    \begin{proof}
        We have ${2n\choose n}=\sum_{k=0}^{n}{n\choose k}^2$. Since ${p\choose 0}={p\choose p}=1$, we need to prove $\sum_{k=1}^{p-1}{p\choose k}^2\equiv 0\pmod{p^3}$. Since all the binomial coeffcients are divisble by $p$, we can rewrite it as
        \[\sum_{k=1}^{p-1}\left(\frac{1}{p}{p\choose k}\right)^2\equiv 0\pmod{p}.\]
        But we have
        \[\frac{1}{p}{p\choose k}=\frac{(p-1)\cdots (p-(k-1))}{k(k-1)\cdots 1}\equiv \frac{1}{k}\frac{(-1)\cdots (-(k-1))}{(k-1)\cdots 1}\equiv \pm\frac{1}{k}\pmod(p),\]
        thus
        \[\sum_{k=1}^{p-1}\left(\frac{1}{p}{p\choose k}\right)^2\equiv \sum_{k=1}^{p-1}\left(\pm\frac{1}{k}\right)^2\equiv \sum_{k=1}^{p-1}\left(\frac{1}{k}\right)^2=\sum_{k=1}^{p-1}k^2=\frac{p(p-1)(2p-1)}{6}\equiv 0\pmod{p}\]
        as when $1/k$ go through $1$ to $p-1$, so does $k$.
    \end{proof}
\end{theorem}

\begin{remark}
    An equivalent formulation is the congruence ${ap\choose bp}\equiv {a\choose b}\pmod{p^3}$ for $p\geq 5$, which we will not cover the proof here.
\end{remark}
