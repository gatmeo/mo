\pitem[]{%
Find all pairs of non-negative integers $(n,m)$ and primes $p$ satisfying:
$\displaystyle{\binom{2n}{n}+n!=p^m}$
    }{%
    The only valid answers are $(m,n,p)=(1,0,2),(1,1,3),(3,2,2)$.

First of all we dispose of the case where $n=0,1$ obtaining the solution $(m,n,p)=(1,0,2) $ and $(1,1,3) $. Now assume that $n>1$. Define $s_a (b)$ to be the sum of digits of $b $ in base $a$ for all $a,b\in\mathbb Z_+$. Then by Lucas Theorem, $v_p (n!)=\frac {n-s_p (n)}{p-1} $ for all primes $p $ and positive integers $n $. Using $p=2$, we obtain $v_2 (n!)=n-s_2 (n) $. Observe that $s_2 (2n)=s_2 (n) $. Thus $v_2\left (\binom {2n}{n}\right)=v_2 ((2n)!)-2v_2 (n!)=(2n-s_2 (n))-2 (n-s_2 (n))=s_2 (n)>0$. So $2|\binom {2n}{n} $. Since $n>1, 2|n!\implies 2|\binom {2n}{n}+n!=p^m\implies p=2$. Now its time for a small claim.

$\text {Claim :} $ Let $a+b=p^m $ for some positive integers $a,b,m $ and prime $p $. Then $v_p (a)=v_p (b) $.

$\text {Proof :} $ Assume the contrary and WLOG let $v_p (a)>v_p (b) $. Let $a=p^{v_p (a)}a_0,b=p^{v_p (b)}b_0$. Then we get $p^m=(a+b)=p^{v_p (b)}(p^{v_p (a)-v_p (b)}a_0+b_0)\implies p|p^{v_p (a)-v _p (b)}a_0 +b_0$. But since $v_p (a)>v_p (b), p|p^{v_p (a)-v_p (b)}\implies p|b_0$, a contradiction. Thus our claim is proven.

Applying our claim to the situation $2^m=\binom {2n}{n}+n! $, we get $v_2\left (\binom {2n}{n}\right) =v_2 (n!)\implies s_2 (n)=n-s_2 (n)\implies n=2s_2 (n) $. Let $n=\sum_{i=1}^k2^{b_i}, b_i <b_{i+1} $ be the binary representation of $n $. Since $n=2s_2 (n),2|n $ hence $b_1>0$. Write $n=2s_2 (n) $ as $\sum_{i=1}^k (2^{b_i}-2)=0$, but $2^{b_i}>2\forall i>1$ and $2^{b_1}=2$ if and only if $b_1=1$. Thus $k=1,b_1=1,n=2$ giving us the final solution $(m,n,p)=(3,2,2) $.
    }{%
    https://artofproblemsolving.com/community/c6h1465157p8483600
}
